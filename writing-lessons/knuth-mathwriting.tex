\magnification\magstep1
\font\sixrm=cmr6
\font\mflogo=logo10
\def\MF{{\mflogo META}\-{\mflogo FONT}}
\def\narrower{\advance\leftskip20pt \advance\rightskip20pt }
\nopagenumbers
\pageno=-1

% To produce only a subset of pages, put the page numbers on separate
% lines in a file called pages.tex
\let\Shipout=\shipout
\newread\pages \newcount\nextpage \openin\pages=pages
\def\getnextpage{\ifeof\pages\else
 {\endlinechar=-1\read\pages to\next
  \ifx\next\empty % in this case we should have eof now
  \else\global\nextpage=\next\fi}\fi}
\ifeof\pages\else\message{OK, I'll ship only the requested pages!}
 \getnextpage\fi
\def\shipout{\ifeof\pages\let\next=\Shipout
 \else\ifnum\pageno=\nextpage\getnextpage\let\next=\Shipout
  \else\let\next=\Tosspage\fi\fi \next}
\newbox\garbage \def\Tosspage{\deadcycles=0\setbox\garbage=}

\leftline{\phantom{Computer Science Department}}
\vfill
\centerline{\bf Mathematical Writing}
\smallskip
\centerline{by}
\smallskip
\centerline{Donald E. Knuth, Tracy Larrabee, and Paul M. Roberts}
\vfill
This report is based on a course of the same name given at Stanford
University during autumn quarter, 1987. Here's the catalog description:
\medskip
{\narrower {\bf CS\thinspace209. Mathematical Writing---}Issues of
technical writing and the effective presentation of mathematics
and computer science. Preparation of theses, papers, books, and
``literate'' computer programs. A term paper on a topic of your
choice; this paper may be used for credit in another course.
\medskip}

The first three lectures were a ``minicourse'' that summarized the
basics. About two hundred people attended those three sessions,
which were devoted primarily to a discussion of the points in
\S1 of this report. An exercise (\S2) and a suggested solution
(\S3) were also part of the minicourse.

The remaining 28 lectures covered these and other issues in depth.
We saw many examples of ``before'' and ``after'' from manuscripts
in progress. We learned how to avoid excessive subscripts and
superscripts. We discussed the documentation of algorithms, computer
programs, and user manuals. We considered the process of refereeing
and editing. We studied how to make effective diagrams and tables,
and how to find appropriate quotations to spice up a text.
Some of the material duplicated some of what would be discussed in
writing classes offered by the English department, but the vast
majority of the lectures were devoted to issues that are specific to
mathematics and/or computer science.

Guest lectures by Herb Wilf (University of Pennsylvania),
Jeff Ullman (Stanford), Leslie Lamport (Digital Equipment Corporation),
Nils Nilsson (Stanford), Mary-Claire van Leunen (Digital Equipment
Corporation), Rosalie Stemer (San Francisco Chronicle), and
Paul Halmos (University of Santa Clara), were a special highlight
as each of these outstanding authors presented their own perspectives
on the problems of mathematical communication.

This report contains transcripts of the lectures and copies of
various handouts that were distributed during the quarter. We think the
course was able to clarify a surprisingly large number of
issues that play an important part in the life of every
professional who works in mathematical fields. Therefore we hope
that people who were unable to attend the course might still
benefit from it, by reading this summary of what transpired.

The authors wish to thank Phyllis Winkler for the first-rate technical
typing that made these notes possible.

Caveat: These are transcripts of lectures, not a polished set of
essays on the subject. Some of the later lectures refer to mistakes
in the notes of earlier lectures; we have decided to correct some
(but not all)
of those mistakes before printing this report. References to such
no-longer-existent blunders might be hard to understand. Understand?

Videotapes of the class sessions are kept in the Mathematical \& Computer
Sciences Library at Stanford.

The preparation of this report was supported in part by NSF grant CCR-8610181.

\eject
% Table of Contents

\centerline{\bf Table of Contents}
\bigskip
\begingroup\advance\baselineskip 0pt plus 1pt
\tabskip=0pt plus 100pt
\halign to\hsize{\hfil\S$\oldstyle#$.\enspace\tabskip0pt&
#\quad\leaders\hbox to1em{\hss.\hss}\hskip8em plus 1fill
 &\enspace\hfil#\tabskip0pt plus 100pt\cr
1&Minicourse on technical writing&1\cr
2&An exercise on technical writing&7\cr
3&An answer to the exercise&8\cr
4&Comments on student answers (1)&9\cr
5&Comments on student answers (2)&11\cr
6&Preparing books for publication (1)&14\cr
7&Preparing books for publication (2)&15\cr
8&Preparing books for publication (3)&18\cr
9&Handy reference books&19\cr
10&Presenting algorithms&20\cr
11&Literate Programming (1)&22\cr
12&Literate Programming (2)&26\cr
13&User manuals&28\cr
14&Galley proofs&30\cr
15&Refereeing (1)&31\cr
16&Refereeing (2)&34\cr
17&Hints for Referees&36\cr
18&Illustrations (1)&37\cr
19&Illustrations (2)&40\cr
20&Homework: Subscripts and superscripts&40\cr
21&Homework: Solutions&43\cr
22&Quotations&47\cr
23&Scientific American Saga (1)&49\cr
24&Scientific American Saga (2)&51\cr
25&Examples of good style&54\cr
26&Mary-Claire van Leunen on `hopefully'&57\cr
27&Herb Wilf on Mathematical Writing&59\cr
28&Wilf's first extreme&61\cr
29&Wilf's other extreme&62\cr
30&Jeff Ullman on Getting Rich&66\cr
31&Leslie Lamport on Writing Papers&69\cr
32&Lamport's handout on unnecessary prose&71\cr
33&Lamport's handout on styles of proof&72\cr
34&Nils Nilsson on Art and Writing&73\cr
35&Mary-Claire van Leunen on Calisthenics (1)&77\cr
36&Mary-Claire's handout on Composition Exercises&81\cr
37&Comments on student work&89\cr
38&Mary-Claire van Leunen on Which vs.\ That&93\cr
39&Mary-Claire van Leunen on Calisthenics (2)&98\cr
40&Computer aids to writing&100\cr
41&Rosalie Stemer on Copy Editing&102\cr
42&Paul Halmos on Mathematical Writing&106\cr
43&Final truths&112\cr}
\endgroup
\eject
% macros for the report layout

\font\sc=cmcsc10 %use lower case as (Monthly)
\font\cm=cmbxsl10
\font\eightsl=cmsl8
\font\boldsy=cmbsy10
\font\boldold=cmmib10

\def\TeXbook{{\sl 
The T\hbox{\hskip-.1667em\lower.424ex\hbox{E}\hskip-.125em X}\kern1pt book}}

\footline={\tenrm\kern-2em[\ifodd\pageno \runhead\hfil\tenit\folio\/%
 \else\tenit\folio\hfil\runhead\/\kern.5pt\fi\tenrm]\kern-2em}

\def\runhead{{\eightsl\uppercase\expandafter{\botmark}}}

\def\beginsection #1. [#2] #3\par
  {\vskip0pt plus.3\vsize\penalty-250
    \vskip0pt plus-.3\vsize\bigskip\vskip\parskip
    \mark{\S{#1}. #2}\message{S#1}%
    \line{\bf{\boldsy\char'170}{\boldold#1}.\quad#3\hfil}
    \nobreak\smallskip\noindent}

\let\sectionsign=\S
\def\S#1{\sectionsign$\oldstyle#1$}
\def\<#1>{\leavevmode\hbox{$\langle$#1\/$\rangle$}} % syntactic quantity
\def\LaTeX{L\kern -.36em\raise.6ex\hbox{\sixrm A}\kern-.15em\TeX}
\def\\#1{\hbox{\it#1\/\kern.05em}} % italic type for identifiers

\def\tll #1 #2 {Excerpts from class, #1 #2\hfill[notes by TLL]}
\def\pmr #1 #2 {Excerpts from class, #1 #2\hfill[notes by PMR]}

\pageno=1

\def\yskip{\penalty-50\vskip 3pt plus 3pt minus 2 pt} 
\def\dskip{\nobreak\vskip 1.5pt\relax} 
\def\thbegin #1. #2\par{\yskip\noindent{\bf#1.}\xskip{\sl#2}\par\yskip}
\def\proofbegin #1. {{\sl#1.}\xskip}
\def\adx#1:#2\par{\par\halign{\hskip #1##\hfill\cr #2}\par}
\def\yyskip{\penalty-100\vskip 6pt plus 6pt minus 4pt}
\def\upsidea{\mathop{\forall}} %  as math operator in formula
\def\backe{\mathop{\exists}} %  as math operator in displayed formula
\let\suchthat=\ni % lower case backwards 
\def\therefore{\mathinner{\mskip2mu\raise1pt\vbox{\kern7pt\hbox{.}}
 \mskip2mu\raise7pt\hbox{.}
 \mskip2mu\raise1pt\hbox{.}\mskip6mu}}
\def\{\Rightarrow}
\def\ldotss{\ldots}
\def\leftv{\left|}
\def\rightv{\right|}
\def\leftset{\,\{}
\def\rightset{\}\,}
\def\relv{\mid}
\def\per{\mathop{\rm per}\nolimits}
\def\blackslug{\hbox{\hskip 1pt \vrule width 4pt height 6pt depth 1.5pt \hskip 1pt}}
\def\ctrline{\centerline}
\def\lft#1{{#1}\hfill}
\def\disleft#1:#2:#3\par{\par\hangindent#1\noindent
			 \hbox to #1{#2 \hfill \hskip .1em}\ignorespaces#3\par}
\def\display#1:#2:#3\par{\par\hangindent #1 \noindent
			\hbox to #1{\hfill #2 \hskip .1em}\ignorespaces#3 \par}
\def\xskip{\hskip .7em plus .3em minus .4em}
\def\point{\vfil\penalty0\vfilneg\disleft 20pt: }

\begingroup
\raggedbottom

\beginsection 1. [Minicourse on technical writing] Notes on Technical Writing

Stanford's library card catalog refers to more than 100 books about technical
writing, including such titles as {\sl The Art of Technical Writing},
{\sl The Craft of Technical Writing}, {\sl The Teaching  of Technical Writing}.
There is even a journal devoted to the subject, the {\sl IEEE Transactions
on Professional Communication}, published since 1958.  The Amer\-ican Chemical
Society, the American Institute of Physics, the American Mathematical
Society, and the Mathematical Association of America
have each published ``man\-uals of~style.'' The last of these,
{\sl Writing Mathematics Well\/} by Leonard Gillman, is one of the required
texts for CS\thinspace209.

The nicest little reference for a quick tutorial is {\sl The
Elements of Style}, by Strunk and White (Macmillan, 1979).  Everybody should
read this 85-page book, which tells about English prose writing in general.
But it isn't a required text---it's merely recommended.

The other required text for CS\thinspace209 is
{\sl A Handbook for Scholars\/} by
Mary-Claire van Leunen (Knopf, 1978).  This well-written book is a real
pleasure to read, in spite of its unexciting title.  It tells about footnotes,
references, quotations, and such things, done correctly instead of the 
old-fashioned ``op.\ cit.''\ way.

Mathematical writing has certain peculiar problems that have rarely been
discussed in the literature.  
Gillman's book refers to the three previous classics in the field:
An article by Harley Flanders, 
{\sl Amer.\ Math.\ Monthly}, 1971, pp.\ 1--10;  another
by R.~P. Boas in the same journal, 1981, pp.\ 727--731. There's also a
nice booklet called {\sl How to Write Mathematics}, published by the
American Mathematical Society in 1973, especially the delightful essay by
Paul~R. Halmos on \hbox{pp.\ 19--48}.

The following points are especially important, in your instructor's view:

\yskip
\point 1.: Symbols in different formulas must be separated by words.

\yskip
\display 70pt: {Bad:}: Consider $Sq$, $q < p$.

\dskip
\display 70pt:	{Good:}: Consider $Sq$, where $q<p$.

\yskip
\point 2.: Don't start a sentence with a symbol.

\yskip
\display 70pt: {Bad:}: $x^n-a$ has $n$ distinct zeroes.

\dskip
\display 70pt: {Good:}: The polynomial $x^n-a$ has $n$ distinct zeroes.

\yskip
\point 3.: Don't use the symbols $\therefore$,
        $\$, $$, $$, $\suchthat$; replace
	them by the corresponding words.\xskip (Except in works on logic, of course.)

\yskip
\point 4.: The statement just preceding a theorem, algorithm, etc.,
	should be a complete sentence or should end with a colon.

\yskip
\display 70pt: {Bad:}: We now have the following

\display 70pt:: {\bf Theorem}.   $H(x)$ is continuous.

\yskip
\disleft 20pt:: This is bad on three counts, including rule 2.  It should be
	rewritten, for  example, like this:

\yskip
\display 70pt: {Good:}: We can now prove the following result.

\display 70pt:: {\bf Theorem.} The function $H(x)$ defined in (5) is continuous.

\yskip
\disleft 20pt:: Even better would be to replace the first sentence by a more
	suggestive motivation, tying the theorem up with the previous discussion.

\yskip
\point 5.: The statement of a theorem should usually be self-contained,
	not depending on the assumptions in the preceding text.\xskip  (See the
	restatement of the theorem in point~4.)

\yskip
\point 6.: The word ``we'' is often useful to avoid passive voice; the
	``good'' first sentence of example 4 is much better than ``The
	following result can now be proved.''  But this use of ``we'' should be
	used in contexts where it means ``you and me together'', {\sl not\/}
	a formal equivalent of ``I''.  Think of a dialog between author and
	reader.

\vskip 2pt
\disleft 20pt:: In most technical writing, ``I'' should be avoided, unless the
author's persona is relevant.

\yskip
\point 7.: There is a definite rhythm in sentences.  Read what you have
	written, and change the wording if it does not flow smoothly.  For example,
	 in the text {\sl Sorting and Searching\/} it was sometimes better to say
	``merge patterns'' and sometimes better to say ``merging patterns''.
	There are many ways to say ``therefore'', but often only one has the
	correct rhythm.

\yskip

\point 8.: Don't omit ``that'' when it helps the reader to parse the sentence.

\yskip
\display 70pt: {Bad:}: Assume $A$ is a group.

\dskip
\display 70pt: {Good:}: Assume that $A$ is a group.

\yskip
\disleft 20pt:: The words ``assume'' and ``suppose'' should usually be followed
	by ``that'' unless another ``that'' appears nearby.  But {\sl never\/}
	say ``We have that $x=y$,'' say ``We have $x=y$.''  And avoid unnecessary
	padding ``because of the fact that'' unless you feel that the reader
	needs a moment to recuperate from a concentrated sequence of ideas.

\yskip

\point 9.: Vary the sentence structure and the choice of words, to avoid
	monotony.  But use parallelism when parallel concepts are being discussed.
	For example (Strunk and White \#15), don't say this:

\yskip
\disleft 70pt:: Formerly, science was taught by the textbook method, while
	now the laboratory method is employed.

\disleft 20pt:: Rather:

\disleft 70pt:: Formerly, science was taught by the textbook method; now it is
	taught by the laboratory method.

\yskip
\disleft 20pt:: Avoid words like ``this'' or ``also'' in consecutive sentences;
	such words, as well as unusual or polysyllabic utterances, tend to stick
	in a reader's mind longer than other words, and good style will keep
	``sticky'' words spaced well apart.\xskip  (For example, I'd better not say
	``utterances'' any more in the rest of these notes.)

\yskip
\point 10.: Don't use the style of homework papers, in which a sequence
	of formulas is merely listed. Tie the concepts together with a running
	commentary.
 
\yskip
\point 11.:  Try to state things twice, in complementary ways, especially
	when giving a definition.  This reinforces the reader's understanding.
	(Examples, see \S2 below: $N^n$ is defined twice,
        $An$ is described as ``nonincreasing'', $L(C,P)$ is characterized as
	the smallest subset of a certain type.)  All variables must be defined,
	at least informally, when they are first introduced.

\yskip
\point 12.: Motivate the reader for what follows. In the example of\/~\S2,
	Lemma 1 is motivated by the fact that its converse is true. 
	Definition 1 is motivated only by decree; this is somewhat riskier.

\vskip 2pt
\disleft 20pt:: Perhaps the most important principle of good writing is to keep
	the reader uppermost in mind:  What does the reader
	know so far?  What does the reader
	expect next and why?

\vskip 2pt
\disleft 20pt:: When describing the work of other people it is sometimes safe to
	provide motivation by simply stating that it is ``interesting'' or
	``remarkable''; but it is best to let the results speak for themselves
	or to give {\sl reasons\/} why the things seem interesting or remarkable.

\vskip 2pt
\disleft 20pt:: When describing your own work, be humble and don't use
	superlatives of praise, either explicitly or implicitly, even if you
	are enthusiastic.

\yskip
\point 13.: Many readers will skim over formulas on their first reading of
	your exposition. Therefore, your sentences should flow smoothly when all
	but the simplest formulas are replaced by ``blah'' or some other grunting
	noise.

\yskip

\point 14.: Don't use the same notation for two different things.
	Conversely, use consistent notation for the same thing when it appears
	in several places.  For example, don't say ``$Aj$ for $1jn$'' in
	one place and ``$Ak$ for ${1kn}$'' in another place unless there is a
	good reason.  It is often useful to choose names for indices so that
	$i$ varies from $1$ to $m$ and $j$ from $1$ to $n$, say, and to stick to
	consistent usage.  Typographic conventions (like lowercase letters
	for elements of sets and uppercase for sets) are also useful.

\yskip
\point 15.: Don't get carried away by subscripts, especially when
	dealing with a set that doesn't need to be indexed; set element
	notation can be used to avoid subscripted subscripts.  For example,
	it is often troublesome to start out with a definition like 
	``Let $X=\{x1,\ldotss,xn\}$'' if you're going to need subsets of~$X$,
	since the subset will have to defined as $\{x{i1},\ldotss,x{im}\}$,
	say.  Also you'll need to be speaking of elements $xi$ and $xj$ all
	the time.  Don't name the elements of $X$ unless necessary. Then you can
	refer to elements $x$ and $y$ of $X$ in your subsequent discussion,
	without needing subscripts; or you can refer to $x1$ and $x2$ as
	specified elements of $X$.

\yskip
\point 16.: Display important formulas on a line by themselves.  If you
	need to refer to some of these formulas from remote parts of the text,
	give reference numbers to all of the most important ones, even if they
	aren't referenced.

\yskip
\point 17.: Sentences should be readable from left to right without
	ambiguity.  Bad examples:  ``Smith remarked in a paper about the 
	scarcity of data.''  ``In the theory of rings, groups and other
	algebraic structures are treated.''

\yskip
\point 18.: Small numbers should be spelled out when used as adjectives,
	but not when used as names (i.e., when talking about numbers as numbers).

\yskip
\display 70pt: {Bad:}: The method requires 2 passes.

\dskip
\display 70pt: {Good:}: Method 2 is illustrated in Fig.~1; it requires  17 passes.
	The count was increased by~2.  The leftmost 2 in the sequence was changed
	to~a~1.

\yskip

\point 19.: Capitalize names like Theorem 1, Lemma 2, Algorithm 3, Method 4.

\vfill\eject

\point 20.: Some handy maxims:

\yskip
\adx 70pt: Watch out for prepositions that sentences end with. \cr
	When dangling, consider your participles. \cr
	About them sentence fragments. \cr
	Make each pronoun agree with their antecedent. \cr
	Don't use commas, which aren't necessary. \cr
	Try to never split infinitives. \cr

\yskip
\point 21.: Some words frequently misspelled by computer scientists:

$$\vcenter{\halign{\lft{#}\hskip30pt
&\lft{#}\hskip30pt
&\lft{#}\cr
implement&not&impliment \cr
complement&not&compliment \cr
occurrence&not&occurence \cr
dependent&not&dependant \cr
auxiliary&not&auxillary \cr
feasible&not&feasable \cr
preceding&not&preceeding \cr
referring&not&refering \cr
category&not&catagory \cr
consistent&not&consistant \cr
PL/I&not&PL/1 \cr
descendant (noun)&not&descendent \cr
its (belonging to it)&not&it's (it is)\cr}}$$

\vskip 2pt
The following words are no longer being hyphenated in current literature:

\yskip
\adx 70pt: nonnegative \cr
	nonzero \cr

\yskip
\point 22.: Don't say ``which'' when ``that'' sounds better.  The general
	rule nowadays is to use ``which'' only when it is preceded by a comma
	or by a preposition, or when it is used interrogatively.  Experiment to
	find out which is better, ``which'' or ``that'', and you'll understand
	this rule.

\yskip
\display 70pt: {Bad:}: Don't use commas which aren't necessary.

\display 70pt: {Better:}: Don't use commas that aren't necessary.
\yskip
\disleft 20pt:: Another common error is to say ``less'' when the proper
word is ``fewer''.

\yskip

\point 23.: In the example at the bottom of\/~\S2 below, note that the text
	preceding displayed equations (1) and (2) does not use any special
	punctuation.  Many people would have written

\yskip
\disleft 70pt:: $\ldots$ of ``nonincreasing'' vectors:
$$An = \leftset(a1,\ldotss,an) \in N^n \relv a1  \cdots  an\rightset. 
	\eqno(1)$$
\disleft 70pt:: If $C$ and $P$ are subsets of $N^n$, let:
$$L(C,P)=\ldots$$
\disleft 20pt:: and those colons are wrong.

\yskip
\point 24.: The opening paragraph should be your best paragraph, and its first
	sentence should be your best sentence.  If a paper starts badly, the
	reader will wince and be resigned to a difficult job of fighting with
	your prose.  Conversely, if the beginning flows smoothly, the reader
	will be hooked and won't notice occasional lapses in the later parts.

\vskip 2pt
\disleft 20pt:: Probably the worst way to start is with a sentence of the form
	``An $x$ is $y$.'' For example,

\yskip
\display 70pt: {Bad:}: An important method for internal sorting is quicksort.

\display 70pt: {Good:}: Quicksort is an important method for internal sorting,
	because $\ldots$

\vskip 2pt
\display 70pt: {Bad:}: A commonly used data structure is the priority queue.

\display 70pt: {Good:}: Priority queues are significant components of the data
	structures needed for many different applications.

\yskip
\point 25.: The normal style rules for English say that commas and 
	periods should be placed inside quotation marks, but other punctuation
	(like colons, semicolons, question marks, exclamation marks) stay outside
	the quotation marks unless they are part of the quotation.  It is
	generally best to go along with this illogical convention about commas
	and periods, because it is so well established, except when you are
	using quotation marks to describe some text as a specific string of 
	symbols.  For example,

\yskip
\display 70pt: {Good:}: Always end your program with the word ``end''.

\yskip
\disleft 20pt:: On the other hand, punctuation should always be strictly logical
	with respect to parentheses and brackets.  Put a period inside
	parentheses if and only if the sentence ending with that period is entirely
	within the parentheses.  The punctuation within parentheses should be
	correct, independently of the outside context, and the punctuation outside
	the parentheses should be correct if the parenthesized statement would be
	removed.

\yskip
\display 70pt: {Bad:}: This is bad, (although intentionally so.)

\yskip
\point 26.: Resist the temptation to use long strings of nouns as
	adjectives:  consider the packet switched data communication network
	protocol problem.

\vskip 2pt
\disleft 20pt:: In general, don't use jargon unnecessarily.  Even specialists
	in a field get more pleasure from 
	papers that use a nonspecialist's vocabulary.

\yskip
\display 70pt: {Bad:}: ``If $\hbox{\bf L}^{\scriptscriptstyle +}(P,N0)$ 
	is the set of functions $f\colon PN0$ with the property that

$$\backe{n0 \in N0} \;\suchthat\; \upsidea{p\in P} p
	  n0 \ f(p)=0$$

\display 70pt:: then there exists a bijection $N1  
	\hbox{\bf L}^{\scriptscriptstyle +}(P,N0)$ 
	such that if	$n\mapsto f$ then

$$n = \prod{p\in P} p^{f(p)}.$$

\display 70pt:: Here $P$ is the prime numbers and $N1=N0 \sim \{0\}$.''

\yskip
\display 70pt: {Better:}: ``According to the `fundamental theorem of arithmetic'
	(proved in ex.\ \hbox{1.2.4--21}), each positive integer $u$ 
	can be expressed in the form 

$$u=2^{u2}3^{u3}5^{u5}7^{u7}11^{u{11}}\ldotss 
  = \prod{p \rm\;prime}p^{up},$$

\display 70pt:: where the exponents $u2,u3,\ldotss$ are uniquely determined
	nonnegative integers, and where all but a finite number of the exponents
	are zero.''

\vskip 2pt
\disleft 20pt:: [The first quotation is from Carl Linderholm's neat satirical
  	book {\sl Mathema\-tics Made Difficult\/}; the second is from D. Knuth's
	{\sl Seminumerical Algorithms}, Section 4.5.2.]

\yskip
\point 27.: When in doubt, read {\sl The Art of Computer Programming\/}
	for outstanding examples of good style.

\vskip 2pt
\disleft 20pt:: [That was a joke.  Humor is best used in technical writing when
	readers can understand the joke only when they also understand
	a technical point that is being made.  Here is another example from
	Linderholm:

\yskip
\disleft 70pt:: ``... $\emptyset D = \emptyset$ and $N\emptyset =N$, which we
	may express by saying that $\emptyset$ is absorbing on the left and 
	neutral on the right, like British toilet paper.''

\yskip
\disleft 20pt:: Try to restrict yourself to jokes that will not seem silly on
	second or third reading.  And don't overuse exclamation points!]

\vfill\eject
\endgroup
\beginsection 2. [An exercise on technical writing] An Exercise on Technical Writing

In the following excerpt from a term paper, $N$ denotes the nonnegative integers,
$N^n$ denotes the set of $n$-tuples of nonnegative integers, and
$An = \{(a1,\ldotss,an)\in N^n \relv a1  \cdots  an\}$.  
If $C, P \subset N^n$, then $L(C,P)$ is defined to be 
$\{c+p1+\cdots +pm \relv c\in C$, $m  0$, and $pj\in P$ for ${1jm}\}$.
We want to prove that $L(C,P) \subseteq An$ implies $C,P \subseteq An$.

The following proof, directly quoted from a sophomore term paper, is mathematically
correct (except for a minor slip) but stylistically atrocious:

\begingroup\advance\baselineskip2pt
\vskip 2pt
\adx 20pt: $L(C,P) \subset An$ \cr
 $C \subset L \ C \subset An$ \cr
 Spse $ p\in P, \;p\notin An \ pi < pj$ for $i<j$ \cr
 $c+p \in L \subset An$ \cr
 $\therefore  ci+pi  cj+pj$ but $ci cj0, pjpi \therefore 
  (ci-cj)  (pj-pi)$ \cr
 but $$ a constant $k \suchthat  c+kp \notin An$ \cr
 let $k = (ci-cj)+1 \qquad c+kp \in L \subset An$ \cr
 $\therefore  ci+kpi  cj+kpj \ (ci-cj)  k(pj-pi)$ \cr
 $\ k-1  k \cdot m \qquad k,m  1$ \qquad Contradiction \cr
 $\therefore  p \in An$ \cr
 $\therefore  L(C,P) \subset An \ C,P \subset An$ and the \cr
 lemma is true. \cr

\endgroup
\vfill
\noindent {\sl A  possible way to improve the quality of the writing:}

\yskip
Let $N$ denote the set of nonnegative integers, and let 
$$N^n = \{\,(b1,\ldotss,bn) \relv bi \in N \hbox{ for }1in\,\}$$ 
be the set of $n$-dimensional vectors
with nonnegative integer components.  We shall be especially interested in the
subset of ``non\-increas\-ing'' vectors,
$$An = \leftset(a1,\ldotss,an) \in N^n \relv a1  \cdots  an\rightset. 
	\eqno(1)$$

If $C$ and $P$ are subsets of $N^n$, let
$$L(C,P) = \leftset c+p1+ \cdots +pm \relv c \in C, m0,\,\, 
	\hbox{and}\,\, pj \in P \,\, \hbox{for}\,\, 1jm\rightset \eqno(2)$$
be the smallest subset of $N^n$ that contains $C$ and is closed under the
addition of elements of $P$.  Since $An$ is closed under addition, $L(C,P)$
will be a subset of  $An$ whenever $C$ and~$P$ are both contained in $An$.
We can also prove the converse of this statement.

\thbegin Lemma 1.  If\/\ $L(C,P) \subseteq An$ and\/\ $C\emptyset$,
then $C \subseteq An$ and\/\ $P \subseteq An$.

\noindent\proofbegin Proof.  (Now it's your turn to write it up beautifully.)

\vfill
\eject
\beginsection 3. [An answer to the exercise] An Answer

Here is one way to complete the exercise in the previous section. (But
please try to {\sc work it yourself before reading this}.)  Note that
a few clauses have been inserted to help keep the reader synchronized with the
current goals and subgoals and strategies of the proof.  Furthermore the
notation $(b1,\ldotss,bn)$ is used instead of $(p1,\ldotss,pn)$, in the
second para\-graph below, to avoid confusion with formula (2).

\yyskip
\proofbegin Proof.  Assume that $L(C,P) \subseteq An$.  Since $C$ is always
contained in $L(C,P)$, we must have ${C \subseteq An}$; 
therefore only the condition
$P \subseteq An$ needs to be verified.

If $P$ is not contained in $An$, there must be a vector $(b1,\ldotss,bn) \in P$
such that $bi < bj$ for some $i<j$.  We want to show that this leads to a
contradiction.

Since the set $C$ is nonempty, it contains some element $(c1,\ldotss,cn)$.
We know that the components of this vector satisfy $c1\cdotscn$, 
because ${C\subseteq An}$.

Now $(c1,\ldotss,cn)+k(b1,\ldotss,bn)$ is an element of $L(C,P)$ for all
$k0$, and by hypothesis it must therefore be an element of $An$.  But if we
take $k=ci-cj+1$, we have $k1$ and
$$ci+kbicj+kbj,$$
hence
$$ci-cjk(bj-bi).\eqno(3)$$
This is impossible, since $ci-cj=k-1$ is less than $k$, yet $bj-bi1$.
It follows that $(b1,\ldotss,bn)$ must be an element of~$An$.~\blackslug

\yskip
Note that the hypothesis $C\emptyset$ is necessary in Lemma 1, for if $C$ is empty
the set $L(C,P)$ is also empty regardless of $P$.

\yskip
\ctrline{[This was the ``minor slip.'']}

\vfill
BUT $\ldots$ don't always use the first idea you think of.  The proof above actually
commits another sin against mathematical exposition, namely the unnecessary
use of proof by contradiction.  It would have been better to use a direct proof:

\yskip
\noindent Let $(b1,\ldotss,bn)$ be an arbitrary element of $P$, and let $i$
and $j$ be fixed subscripts with $i<j$; we wish to prove that $bibj$.  Since
$C$ is nonempty, it contains some element $(c1,\ldotss,cn)$.  Now the vector
$(c1,\ldotss,cn)+k(b1,\ldotss,bn)$ is an element of $L(C,P)$ for all 
$k0$, and by hypothesis it must therefore be an element of $An$.  But this
means that $ci+kbi  cj+kbj$, i.e.,
$$ci-cj  k(bj-bi),\eqno(3)$$
for arbitrarily large $k$.  Consequently  $bj-bi$ must be zero or negative.

We have proved that $bj-bi0$ for all $i<j$, so the vector $(b1,\ldotss,bn)$
must be an element of~$An$. \blackslug

\yskip
This form of the proof has other virtues too:  It doesn't assume that the $bi$'s
are integer-valued, and it doesn't require stating that $c1\cdots cn$.

\vfill\eject
\beginsection 4. [Comments on student answers (1)] \tll October 7

Our first serious business involved examining ``the worst abusers of the
`Don't use symbols in titles' rule.''  Professor Knuth (hereafter known as
Knuth) displayed a paper by Gauss that had a long displayed formula
in the title. He
showed us a bibliography he's preparing that references not only that paper but
another with even more symbols in the title.  (Such titles make more than
bibliographies difficult; they make bibliographic data retrieval systems
and keyword-in-context
produce all sorts of hiccups.)

\parindent 0pt
\parskip 5pt plus1pt minus1pt

In his bibliography Knuth has tried to keep his citations true to the
original sources.  The bibliography contains mathematical formulas, full
name spellings (even alternative spellings when common), and completely
spelled-out source journal names. (This last may be unusual enough that
some members of a field may be surprised to see the full journal name
written out, but it's a big help to novices who want to find it in
the library.)

We spent the rest of class going over some of the solutions that students
had turned in for the exercise of\/~\S2 (each
sample anonymous).  He cautioned us that while he was generally pleased 
by the assignments, he was going to be pointing out things that could be
improved. The following points were all made in the process of going
through these samples.

\smallskip
\disleft 30pt::
    In certain instances, people did not understand what constitutes a proof.
    Fluency in mathematics is important for Computer Science students but
    will not be taught in this class. 
    
\smallskip
\disleft 30pt::
    Not all formulas are equations. Depending on the formula, the terms
    `relation', `definition', `statement', or `theorem' might be used. 
    
\smallskip
\disleft 30pt::
    Computer Scientists must be careful to distinguish between mathematical
    notation and programming language notation.  While it may be appropriate
    to use $p[r]$ in a program, in a
    formal paper it is probably better to use~$p$ with a subscript of~$r$.
As another 
    example, it is not appropriate to use a star ($\ast$) to denote multiplication 
    in a paper about mathematics. Just say~$xy$.

\smallskip
\disleft 30pt::
    Some people called $p$ an element of~$P$ and $pr$ an element of~$p$.
Everything was an ``element.''  It's better to call $pr$ a ``component''
of~$p$, thus distinguishing two kinds of subsidiary relationships.

    
\smallskip
\disleft 30pt::
    It is natural in mathematics to hold off some aspects of your
    definition---to ``place action before definition'' (as in 
    `$p(x) < p(y)$  for some $x < y$').
But it is possible to carry this too 
    far, if too much is being held back.
 The best location for certain definitions is a subjective matter.
    
\smallskip
\disleft 30pt::
    Remember to put words between adjacent formulas.
    
\smallskip
\disleft 30pt::
    When you use ellipses, such as $(P1,\ldots, Pn)$, remember to put commas
    before and after the three dots.  When placing ellipses between commas the
    three dots belong on the same level as the commas, but when the ellipsis
    is bracketed by symbols such as `$+$' or `$<$' the dots should be at
    mid-level.  
    
\smallskip
\disleft 30pt::
    Be careful with the spacing around ellipses.  The character
    string `\dots$\!\!$;' looks strange (it should have more space after the last
    dot).  All kinds of accidents happen concerning spaces in formulas.
Typesetting systems like \TeX\ have built-in
 rules to cover 99\% of the cases, but if you write a lot of mathematics
    you will get bitten.
    
\smallskip
\disleft 30pt::
    Linebreaks in the middle of formulas are undesirable.  There are ways
    to enforce this with \TeX\ (as well as other text formatting systems).
    People who use \TeX\ and wish to use the vertical bar and the empty set
symbol in notation like `$\{c\mid c\in\emptyset\}$'
 should be aware of the \TeX\ commands 
{\tt{\char'134}mid}
and 
{\tt{\char'134}emptyset}.
    
\smallskip
\disleft 30pt::
    Comments such as, ``We demonstrate the second conclusion by contradiction,''
    and ``There must be a witness to the unsortedness of P,'' are useful because
    they tell the reader what is going on or bring in new and helpful
    vocabulary.
    
\smallskip
\disleft 30pt::
    Numbering all displayed formulas is usually a bad idea; number
    the important ones only.  Extraneous parentheses can also be distracting.
    For example, in the phrase ``let $k$ be $(ci-cj)+1$,'' the parentheses
    should be omitted.
    
\smallskip
\disleft 30pt::
    You can overdo the use of any good tool.  For instance, you could overuse
    typographic tools by having 20 different fonts in one paper.
    
Two more topics were touched on (and are sure to be discussed further):
the use of `I' in technical writing, and the use of past or present tense
in technical writing.

Knuth says that Mary-Claire van Leunen defends the use of `I' in scholarly
articles, but that he disagrees (unless the identity of the author is
important to the reader).  Knuth likes the ``teamwork'' aspect of using `we'
to represent the author and reader together.  If there are multiple authors,
they can either ``revel in the ambiguity'' of continuing
to use `we', or they can use added disambiguating text. If one author needs to
be mentioned separately, the text can
say `one of the authors (DEK)', or `the first
author', but not `the senior author'.

Knuth (hereafter known as Don)
recommends that one of two approaches be used with respect to tenses
of verbs: Either use present tense throughout the entire paper, or write
sequentially.  
Sequential writing means that you say things like,
``We saw this before.  We will see this
later.''  The sequential approach is more appropriate for lengthy
papers. You can use it even more effectively by using words of duration:
``We observed this long ago. We saw the other thing
recently. We will prove
something else soon.''

\beginsection 5. [Comments on student answers (2)] \tll October 9

``I'm thinking about running a contest for the best Pascal program that is
also a sonnet,'' was the one of the first sentences out of Don's mouth on
the topic of the exact definition of ``Mathematical Writing.''  He admitted
that such a contest was ``probably not the right topic for this course.''
However, a program (presumably even an iambic pentameter program) is among
the documents that he will accept as the course term paper.  He will
accept articles for professional journals, chapters of books or theses,
term papers for other courses, computer programs, user manuals or parts thereof:
anything that falls into a definite genre where you have a specific
audience in mind and the technical aspect is significant.

We spent the rest of class continuing to examine the
homework assignment.  In the interest of succinct notes, I~have replaced
many literal phrases by their generic equivalents.  For example, I~might
have replaced  `$A > B$' by `\<relation>'.  This time I have divided the
comments into two sets: those dealing with what I will call ``form''
(parentheses, capitalization, fonts, etc.) and those dealing with
``content'' (wording, sentence construction, tense, etc.).

First, the comments concerning form:

\smallskip
\disleft 30pt::
    Don't overdo the use of colons.  While the colon in  `Define it as 
    follows:' is fine, the one in  `We have: \<formula>' should be
    omitted since the formula just completes the sentence.
Some papers had more colons than periods.
    
\smallskip
\disleft 30pt::
    Should the first word after a colon be capitalized? Yes, if the phrase
    following the colon is a full sentence. No, if it is a sentence fragment.
(This is not ``yet'' a standard rule, but Don has been trying it for several
years and he likes~it.)
    
\smallskip
\disleft 30pt::
    While too many commas will interfere with the smooth flow of a sentence, 
    too few can make a sentence difficult to read.  As examples, a sentence
    beginning with `Therefore, ' does not need the comma following
    `therefore'. But `Observe that if \<symbol> is \<formula> then so
    is \<symbol> because \<reasoning>' at least needs a comma before `because'.
    
\smallskip
\disleft 30pt::
    Putting too many things in parentheses is a stylistic thing that can get
    very tiring. (When Don moves from his original, handwritten draft to a
    typed, computer-stored version his most frequent change is to remove extra
    parentheses.)
    
\smallskip
\disleft 30pt::
    Among the parentheses most in need of removal are nested parentheses. To
    this end, it is better to write `(Definition~2)' than `(definition (2))'.
Unfortunately, however, you can't use the former if the definition was given
in displayed formula~(2). Then it's probably best to think of a way to
avoid the outer parentheses altogether.
    
\smallskip
\disleft 30pt::
    In some cases your audience may expect nested parentheses.  In this case
    (or in any other case 
when you feel you must have them), should the outer pair
    be changed to brackets (or curly-braces)?  This was once the
    prevailing convention, but it is now not only obsolete but potentially 
    dangerous; brackets and curly braces have semantic content
for many scientific professionals.
(``The world is short of delimiters,'' says Don.)  
    Typographers help by using slightly larger parentheses for the outer pair 
    in a nested set.
    
\smallskip
\disleft 30pt::
    An entire paper or proof in capital letters is distracting. It gives the
    impression of sustained shouting. Same goes for boldface, etc.
    
\smallskip
\disleft 30pt::
    Paul Halmos introduced the handy convention of placing a
    box at the end of a proof; this box serves the same function as the 
    initials `Q.E.D.'.  If you use such a box, it seems best to leave a space
    between it and the final period.

\smallskip
\disleft 30pt::
    Try to make it clear where new paragraphs begin.  When using displayed
    formulas, this can become confusing unless you are careful.
    
\smallskip
\disleft 30pt::
    Using notational or typographic conventions can be helpful to your readers
    (as long as your convention is appropriate to your audience).  Boldface
    symbols or arrows over your vectors are each appropriate in the correct
    context.  When using a raised `st' in phrases such as `the $i + 1^{st}$
    component', it's better to use roman type: `$i+1^{\rm st}$'. Then it's
    clear that you aren't speaking of ``$1$~raised to the power~$st$.''
    
\smallskip
\disleft 30pt::
    Avoid ``psychologically bad''
 line breaks.  This is  subjective, but you can catch many such awkward breaks
by  not letting the final symbol lie on a line 
    separate from the rest of its sentence.
 If you are using \TeX, a tilde ({\tt{\char'176}})
 in place of a space will 
    cause the two symbols on either side of the tilde to be tied together.  
    (Other text processors also have methods to disallow line breaks at 
    specific points.)

\smallskip
\disleft 30pt::
    Some of us are much better at spelling than others of us.  Those of us who
    are not naturally wonderful spellers should learn to use spelling checkers.
    
\smallskip
\disleft 30pt::
    Allowing formulas to get so long that they do not format well or are
    unnecessarily confusing ``violates the principle of `name and conquer' that
    makes mathematics readable.''  For example, 
`$v = c + u(ci - cj + 1)$' should be 
`$v = c + ku$, where $k = ci - cj + 1$', if you're going to do a lot
of formula manipulation in which $ci-cj+1$ remains as a unit.
 
\smallskip
\disleft 30pt::
    Be stingy with your quotation marks.  ``Three cute things in quotes is
    a little too cute.''
    
\smallskip
\disleft 30pt::
    Remember to minimize subscripts.  For example, `$pi$~is an element 
of~$P$' could more easily be `$p$~is an element of~$P$'.
    
\smallskip
\disleft 30pt::
    Remember to capitalize words like theorem and lemma in titles like
    Lemma~1 and Theorem~23.
    
\smallskip
\disleft 30pt::
    Remember to place words between adjacent formulas. A particularly bad
example was, ``Add $p\;k$ times to~$c$.''
    
\smallskip
\disleft 30pt::
    Be careful to define symbols before you use them (or at least to define
    them very near where you use them).

\smallskip
\disleft 30pt::
    Don't get hung up on one or two styles of sentences. The following
    sort of thing can become very monotonous:

\halign{\qquad\qquad\qquad #\hfill\cr
Thus, -- -- -- -- .\cr
Consequently, -- -- -- -- .\cr
Therefore, -- -- -- -- .\cr
And so, -- -- -- -- .\cr}
\smallskip
\disleft 30pt::
   On the other hand, parallelism should be
   used when it is the point of the sentence.

\smallskip
Now the comments involving content:
    
\smallskip
\disleft 30pt::
    Try to make sentences easily comprehensible from left to right.  For
    example, ``We prove that \<grunt> and \<snort> implies \<blah>.'' It would be
    better to write ``We prove that the two conditions \<grunt> and \<snort>
    imply \<blah>.'' Otherwise it seems at first that \<grunt> and \<snort>
    are being proved.
    
\smallskip
\disleft 30pt::
    While guidelines have been given for the use of the word `that', the final
    placement must be dictated by cadence and clarity.  Read your words aloud
    to yourself.
    
\smallskip
\disleft 30pt::
The word `shall' seems to be a natural word for definitions to many
    mathematical readers, but it is considered formal by younger members of the 
    audience.
    
\smallskip
\disleft 30pt::
    Be precise in your wording.  
If you mean ``not nonincreasing,'' don't say ``increasing''; the
former means that $pj<p{j+1}$ for {\sl some\/}~$j$, while the latter
that $pj<p{j+1}$ for {\sl all\/}~$j$.
    
\smallskip
\disleft 30pt::
    Mixed tenses on the same subject are awkward.  For example, ``We assume now
   \<grunt>, hoping to show a contradiction,'' is better than, ``We assume
    now \<grunt>, and will show that this leads to a contradiction.''
    
\smallskip
\disleft 30pt::
    Many people used the ungainly phrase ``Assume by contradiction that \<blah>.''
    It is better to say, ``The proof that \<blah> is by contradiction,'' and even
    better to say ``To prove \<grunt>, let us assume the opposite and see
    what happens.''
    
\smallskip
\disleft 30pt::
    In general, a conversational tone giving signposts and clearly written
    transition paragraphs provides for pleasant reading.  One especially 
easy-to-read proof contained the phrases ``The operative word is zero,'' ``The
    lemma is half proved,'' and ``We divide the proof into two parts, first
    proving \<blah> and then proving \<grunt>.''
    
\smallskip
\disleft 30pt::
You can give relations in two ways, either saying `$pi<pj$' or `$pj>pi$'.
The latter is for ``people who are into dominance,'' Don says, but the former
is much easier for a reader to visualize after you've just said
`$p=(p1,p2,\ldots,pn)$ and $i<j$'. Similarly, don't say
`$i<j$ and $pj<pi$'; keep $i$ and~$j$ in the same relative position.

\beginsection 6. [Preparing books for publication (1)] \tll October 12

Don opened class by saying that up until now he has been criticizing our
writing; now he will show us what he does to his own.  Perhaps apropos
showing us his own writing he quoted Dijkstra: ``A good teacher will teach
his students the importance of style and how
to develop their own style---not how to mimic his.''

First he showed us a letter from Bob Floyd. The letter opened by saying
`Don,
Please stop using so many exclamation points!' and closed with at least
five exclamation points.  After receiving this letter he looked in {\sl The
Art of Computer Programming\/} and found about two exclamation points per
page.  (Among the other biographical tidbits we learned at this class were
that Don went to secretarial school, types 80 words per minute, and once knew
two kinds of shorthand.)

Don is writing a book with Oren Patashnik and Ron Graham.  The book is
entitled {\sl Concrete Mathematics\/} and is to be used for CS~260.  He
showed us two copies of Chapter Five of this book: one copy he called
``Before'' and one he called ``After''.

The Before copy actually came into existence long after the work on the
book began.  Oren wrote several drafts using the \LaTeX\ book style, and
then the authors availed themselves of the services of a book designer.
The designer decided how wide the text was, what fonts were to be used,
what chapter headings looked like, and a host of other things.  The
designer, at the authors' request, has left room in the inner margins for
``graffiti.''  That is, for informal snappy comments from the peanut
gallery. (This idea was ``stolen'' from the booklet {\sl Approaching Stanford}.)

The After copy is just another formally typeset revision of the Before
copy. Neither copy has yet been through a professional copy editor. Having
now mentioned copy editors and book designers, Don said, ``In these days of
author self-publishing, we must not forget the value of professionals.''
(Actually, the copy editor was first mentioned when an error in punctuation
was displayed on the screen.) 

Upon receiving a question from the audience concerning how many times he
actually rewrites something, Don told us (part of) his usual rewrite
sequence:

His first copy is written in pencil.  Some people compose at a terminal,
but Don says, ``The speed at which I write 
by hand is almost perfectly synchronized
with the speed at which I think.  I~type faster than I think so I have to
stop, and that interrupts the flow.''

In the process of typing his handwritten copy into the computer he edits
his composition for flow, so that it will read well at normal reading speed.
 Somewhere around here the text gets \TeX ed, but
the description of this stage was tangled up with the description of the
process of rewriting the composition. Of course, rewriting does not all
occur at any one stage.  As Don said, ``You see things in different ways on
the different passes. Some things look good in longhand but not in type.''

While discussing his own revisions, he mentioned those of two other
Computer Science authors. Nils Nilsson had 
at least
five different formal drafts of
his ``Non-Monotonic Reasoning'' chapter.  Tony
Hoare revised the algorithm in his paper on ``Communicating Sequential
Processes'' more than a dozen times over the course of two years.  

Don, obviously a fan of rewriting in general, told us that he knows of
many computer programs that were improved by scrapping everything after
six months and starting from scratch.  He said that this exact approach
was used at Burroughs on their Algol compiler in 1960 and the result was what Don
considers to be one of the best computer programs he has ever seen.  On
the limits of the usefulness of rewriting, he did say, ``Any writing can
be improved.  But eventually you have to put something out the door.''

The last part of class was spent discussing the font used in the coming
book: Euler.  The Euler typeface
 was designed by Hermann Zapf (``probably the greatest
living type designer'') and is an especially appropriate font to use in a
book that is all about Euler's work.  The idea of the face is
to look a bit handwritten.  For example, the zero to be used for
mathematics has a point at the apex because ``when people write zeros, they
never really close them.'' This zero is different from the zero used
in the text (for example, in a date), so book preparation with Euler
needs more care than usual. You have to distinguish mathematical numerals
from English-language numerals in the manuscript.

Somebody asked about `all' versus `all~of'. Which should it be? Answer:
That's a very good question. Sometimes one way sounds best, sometimes
the other. You have to use your ear. Another tricky business is the position
of `only' and `also'; Don says he keeps shifting those words around when he edits
for flow.

\beginsection 7. [Preparing books for publication (2)] \pmr October 14

Don discussed the labours of the book designer and showed us specimen
``page plans'' and example pages.  The former are templates for the page
and show the exact dimensions of margins, paragraphs, etc. His designer
also suggested
a novel scheme for equations: They are to be indented
much like paragraphs rather than being centered in the traditional
way.  We also saw conventions for the display of algorithms and
tables.  Although Don is doing his own typesetting, he is using the
services of the designer and copy editor. These professionals are well
worth their keep, he said.  Economists in the audience were not
surprised to hear that the prices of books bear almost no relation to
their production costs. Hardbacks are sometimes cheaper to produce
than paperbacks. For those interested in such things, Don recommended
a paperback entitled {\sl One Book\thinspace/\thinspace Five Ways\/}
(available in the Bookstore)
that describes the entire production process by means of actual documents.

Returning to the editing of his Concrete Maths text, Don went through
more of the Before and After pages he began to show us on
Monday, picking out specific examples that illustrate points of general
interest.

He exhorted 
writers to try to put themselves in their readers' shoes: ``Ask yourself what
the reader knows and expects to see next at some point in the text.''
Ideally, the finished version reads so simply and smoothly that one
would never suspect that it had been rewritten at all. 
For example, part of the Concrete Maths draft said

\display 80pt:(Before):
The general rule is ($\,$\dots$\,$) and it is particularly valuable because
($\,$\dots$\,$). The transformation in (5.12) is called ($\,$\dots$\,$). It is
easily proved since ($\,$\dots and \dots$\,$).

Reading this at speed and in context made it clear that readers would
be hanging on their chairs wondering why the rule was true; so we
should first tell them why, before stressing the rule's significance:

\display 80pt:(After):
The general rule is ($\,$\dots$\,$) and it is easily proved since ($\,$\dots 
and \dots$\,$).

\display 80pt::
[new paragraph] 
Identity (5.12) is particularly valuable because ($\,$\dots$\,$). It is called
($\,$\dots$\,$).

Don's favorite
dictionary was of no help on the question of `replace
with' vs.\ `replace by'. The phrase `by replacing -- -- by -- --~' is
bad (due to the repetition), and `by replacing -- -- with -- --~' seems worse.
In this case the solution is to choose another word:
`by changing --~--~to~--~--~'.

As a very general rule, try reading at speed. You will often get a
much better sense of the rhythm of the sentence than you did when you
wrote it.

It is a bad idea to display false equations. The reader's eye is apt
to alight upon them in the text and treat them as gospel. It is much
better to put them into the text, as in ``So the equation `$\,$\dots$\,$' is
always false!''

Be sure that the antecedent of any pronoun that you use is clear.
For example, the previous paragraph has two sentences beginning
`It is \dots$\,$'; they are fine. But sometimes such a sentence structure
is troublesome because `it' seems to be referring to an
object under discussion. For example, 

\display 80pt:(Before):
Two things about the derivation are worthy of note. First, it's a
great convenience to be summing \dots$\,$.
\display 80pt:(After):
Two things about this derivation are worthy of note. First, we see
again the great convenience of summing \dots$\,$.

Towards the end of the editing process you will need to ensure that
you don't have a page break in the middle of a displayed formula. Often
you'll simply have to think up something else to say to fill up the
page, thus pushing the displayed formula entirely onto the next page.
Try to think of this as a stimulus to research!

Let proofs follow the same order as definitions, e.g., where you have
to deal with several separate cases.

Hyphens, dashes, and minus signs are distinct and should not be used
interchangeably. The shortest is the hyphen. The next is the en-dash,
as in `lines 10--18'. Longer still is the minus sign, used in formul{\ae}:
`$10-18=-8$'.
The longest of all is the em-dash---used in sentences.

When proofreading you may catch technical errors as well as stylistic
errors. Think about the mathematics too, not just the prose. For
example, the book was discussing a purported argument that $0^0$
should be undefined ``because the functions $x^0$ and $0^x$ have
different limiting values when $x0$.'' Don revised this statement
to ``\dots when $x$ decreases to~0,'' because $0^x$ is undefined
when $x0$ through negative values.

\vfill\eject

When you use the word `instead', be clear about the contrast you are
drawing. The reader should
immediately understand what you are referring~to:

\display 80pt:(Before):
And when $x=-1$ instead, \dots
\display 80pt:(After):
And when $x=-1$ instead of $+1$, \dots

Notice the helpful use of a redundant `$+$' sign here.

Use the present tense for timeless facts. Things that  we
proved some time ago are nevertheless still true.

Try to avoid repeating words in a sentence. 

\display 80pt:(Before):
-- -- approach the values -- -- fill in the values -- -- . 
\display 80pt:(After):
-- -- approach the values -- -- fill in the entries -- -- .

In answer to a question from the class, Don suggested giving page
numbers only for remote references (to equations, say). Usually it is
enough to say `using Equation 5.14' or whatever. It becomes unwieldy
to give page numbers for every single such reference. A~member of the
class suggested the `freeway method' for numbering tables; number
them with the page number on which they appear. Don confessed that he
hadn't thought of this one. Sounds like a neat idea.

The formula

\display 80pt:(Before):
$\displaystyle{\sum{km}{r\choose k}\left(k-{r\over 2}\right)
=-{m+1\over 2}{r\choose m+1}}$

\display 80pt::
looks a bit confusing because of the minus sign on the right, so 
Don changed it to

\display 80pt:(After):
$\displaystyle{\sum{km}{r\choose k}\left({r\over 2}-k\right)
={m+1\over 2}{r\choose m+1}\,.}$

There may be many ways to write a formula;  you have the freedom
to select the best. (This change also propagated into the subsequent
text, where a reference to `the factor $(k-r/2)$' had to be changed to
`the factor $(r/2-k)$'.)

Somebody saw an integral sign on that page and asked about the relative
merits of
$$\int{-}^af(x)\,dx$$
versus other notations like
$$\int\limits{-}^af(x)\,dx\qquad\qquad
\int\limits{x=-}^{x=a}f(x)\,dx\,.$$
Don said that putting limits above and below, instead of at the right,
traded vertical space for horizontal space, so it depends on how wide
your formulas are. Both forms are used. Whichever form you adopt should
be consistent throughout an entire book. Somebody suggested
$$\int{\hskip-32pt x=-\,}^{x=a}f(x)\,dx$$
but Don pooh-poohed this.

He said that major writing projects each have their own style; you get
to understand the style that's appropriate as you write more and more
of the book, just as novelists learn about the characters they are
creating as they develop a story. In {\sl Concrete Mathematics\/}
he is learning how to use the idea of ``graffiti in the margin''
as he writes more. One nice application is to quote from the first
publications of important discoveries; thus famous mathematicians
like Leibniz join the writers of 20th century graffiti.

\beginsection 8. [Preparing books for publication (3)] \tll October 16

We continued to examine Before and After pages from
the book of which Don is a coauthor. The following points were made in
reference to changes Don decided to make.

\smallskip
\display 30pt::
    When long formulas don't fit, try to break the lines logically. In some
    cases the author can even change some of the math (perhaps by introducing
    a new symbol) to make the formula placement less jarring.  Such a change
is best made by the author, since the choice of how to display a complex
expression is an important part of any 
mathematical exposition.

\smallskip
\display 30pt::
    Sometimes moving a formula from embedded text to one separately displayed
    will allow the formula to be more logically divided. The placement of the
    equals sign ($=$) is different for line breaks in the middle of displayed
    versus embedded formulas: The break comes after the equals sign in an
    embedded formula, but before the equals sign in a display.
    
\smallskip
\display 30pt::
    While editing for flow, sentences can be broken up by changing semicolons
    to periods; or if you want the sentences to join into a quickly moving
    blur, you can change periods to semicolons.  Breaking existing
    paragraphs into smaller paragraphs can also be helpful here.
    
\smallskip
\display 30pt::
    While making such changes make sure to preserve clarity.  For example,
    make sure that any sentences you create that begin with conjunctions are
    constructed clearly, and that words like `it' have clear antecedents.
    (Sentences that begin with the word `And' are not always evil.)
    
\smallskip
\display 30pt::
    Make sure your variable names are not misleading.  Variable names that
    are too similar to conceptually unrelated variables can be confusing.
Systematic  variable renaming is one of the advantages of text editors.
    
\smallskip
\display 30pt::
We noted last time that present tense is correct for facts
    that are still true; but it is okay to use past tense for ``facts'' that 
have turned     out to be in error.
    
\smallskip
\display 30pt::
    One of the most common errors that mathematicians make when they get their
    own typesetting systems is to over-use the form of fraction with a
    horizontal bar $\bigl({1+x\over y}\bigr)$ rather than a slash 
$\bigl((1+x)/y\bigr)$.
The stacked form can lead to tiny little
    numbers---especially when they are used in exponents.  One of the most
    common changes that mathematical copy editors make is to slash 
a mathematician's fractions.
(They even know that they have to add parentheses when
    they do this.)
    
\smallskip
\display 30pt::
    Exercises are some of the most difficult parts of a book to write.
    Since an exercise has
very little context, ambiguity can be especially deadly; a bit of carefully chosen
    redundancy can be especially important.  For this reason, exercises are also
    the hardest technical writing to translate to other languages.
    
\smallskip
\display 30pt::
    Copyright law has changed, making it technically necessary to give credit to
    all previously published exercises.  Don says that crediting sources is
    probably sufficient (he doesn't plan to write every person referenced in
    the exercises for his new book, unless the publisher insists).
Tracing the
    history of even well-known theorems can be difficult, because mathematicians
have tended to omit citations.
He recently spent four
    hours looking through the collected works of Lagrange trying to find the
source of ``Lagrange's inequality,'' but he was unsuccessful.
Considering the benefit to future authors and readers,
 he's not too unhappy with the new law.
    
\smallskip
\display 30pt::
    We can dispense with some of our rhetorical guidelines when writing the
    answers to exercises.  Answers that are quick and pithy,
and answers that start
    with a symbol, are quite acceptable.

\beginsection 9. [Handy reference books] \tll October {16 (continued)}

From esoteric mathematics we moved on to reference books.  Don showed us
six such books that he likes to have next to him when he writes.
[And he added a seventh later.]

\smallskip
\display 30pt:1.:
{\sl       The Oxford English Dictionary\/} (usually called the OED). 
	He showed us the two volume ``squint print'' edition rather than
	the 16-volume set.  This compact edition is often offered as a bonus
	given to new members upon joining a book club.  (There is 
	a project in Toronto that will soon have the entire OED 
	online.)

\smallskip
\display 30pt:2.:
           The {\sl OED Supplement}.  The supplement brings the OED up to date.
 	The supplement comes in four volumes, each of which costs \$100 
or more, so you
           may have to go to the library for this one.

\smallskip
\display 30pt:3.:
           {\sl The American Heritage Dictionary}.  Don likes this dictionary 
	because of the usage notes and the Appendix containing
	Indo-European root words.  (For example, the usage notes will
	help you choose between `compare to' or `compare with' in 
	a specific sentence.)
	
\smallskip
\display 30pt:4.:
          {\sl The Longman Dictionary of Contemporary English}.  Instead of the
	historical words found in the previously mentioned dictionaries, 
	this one has the words used on the street.  Current slang and
	popular usage are explained in very simple English.  (For example,
	the nuances of `mind-bending' versus `mind-blowing' versus 
	`mind-boggling' are explained.)
	
\smallskip
\display 30pt:5.:
	{\sl Webster's New Word Speller Divider}.  Don said that people who
	don't spell well find this book to be quite useful.  [I~saw no
	indication that {\sl he\/} actually uses it, though.]
	
\smallskip
\display 30pt:6.:
	{\sl Roget's Thesaurus}. This book is a synonym dictionary.  Don says
	that he owns two, 
one for home and one for his Stanford office, and he
 uses them in many different ways: when he
	knows that a word exists but has forgotten it; when he wants to 
	avoid repetition;  when he wants to define a new 
technical term or a new name for a 	paper or program.
	
\smallskip
\display 30pt:7.:
	{\sl Webster's Dictionary of English Usage}.
A wonderful new (1989) resource, which goes well beyond the
{\sl American Heritage\/} usage notes. It's filled with choice
examples and is a joy to read.

\smallskip
 The issue of British versus American dialect
came up.  When writing for international audiences, should we use
British or American spellings and conventions?
  Don says  he agrees
with the rule that Americans should write with their own spellings 
and the British should do the same.  The two styles should be mixed only
when, say,  an American
writes about the `labor of the British Labour Party'.
(Readers of these classnotes will now understand why TLL and PMR
spell some words differently.)
\beginsection 10. [Presenting algorithms] \tll October 19

Should this course have been named ``Computer Scientifical Writing'' or
``Informatical Writing'' rather than ``Mathematical Writing''?  The Computer
Science Department is offering this class, but until now we have been
talking about topics that are generally of concern to all writers who use
mathematics. Today we begin to discuss topics
specific to the writing of Computer Science.  

We are not abandoning mathematical concerns; Don says that a technical
typist in Computer Science must know all that a Math department typist
must know plus quite a bit more.  He showed us two examples where
mathematical journals had trouble presenting programs, algorithms, or
concrete mathematics in papers he wrote.
In order to solve the first problem,  Don had to convince the typesetters
at {\sl Acta Arithmetica\/} to create ``floor'' and ``ceiling'' functions by carving
off small pieces of the metal type for square brackets.  The second
problem had to do with typographic conventions for computer programs; {\sl The
American Mathematical Monthly\/} was using different fonts for the same
symbol at different points in a procedure,  was interchangeably using
``:$=$'', ``:~$=$'', and ``$=$:'' to represent an assignment symbol.

Stylistic conventions for programming languages originated with Algol~60.
Prior to 1960, FORTRAN and assembly languages were
displayed using all uppercase letters in variable-width fonts that did not
mix letters and numbers in a pleasant manner. Fortunately, Algol's visual
presentation was treated with more care: Myrtle Kellington of ACM worked from the
beginning with Peter Naur 
(editor of the Algol report)
to produce a set of conventions concerning,
among other things, indentation and the treatment of reserved words.

Don found the prevailing variable-width fonts unacceptable for use in the
displayed computer programs in Volume~1 of {\sl The Art of Computer
Programming},  and he insisted that he needed fixed-width type. The publishers
initially said that it wasn't possible, but they eventually found a way to
mix {\tt typewriter} style with roman, {\bf bold}, and {\it italic}.

Don says he had a difficult time trying to
decide how to present algorithms.  He could have used a specific
programming language, but he was afraid that such a choice would alienate
people (either because they hated the language or because they had no
access to the language). So he decided to write his algorithms in English.

His Algorithms are presented rather like
Theorems with labeled steps; often they
have accompanying (but very high-level)
flow charts (a technique he first saw in Russian
literature of the 1950s).  The numbered steps have parenthetical remarks
that we would call comments; after 1968 these parenthetical remarks are often
invariant relations
 that can be used in a formal proof of program
correctness.

Don has received many letters complimenting him on his approach, but he
says it is not really successful. 
Explaining why, he said, ``People keep saying,
`I'm going to present an algorithm in Knuth's style,' and then they 
completely botch it by ignoring the conventions I~think are most important.
This style must just be a personal style that works
for me.  So get a personal style that works for you.''  In recent papers he has
used the pidgin Algol style introduced by Aho,
Hopcroft,  and Ullman; but he will
not change his style for the yet-unfinished volumes of {\sl The Art of
Computer Programming\/} because he wants to keep the entire series
consistent.

Don says that a computer program is a piece of literature. (``I look forward to the
day when a Pulitzer Prize will be given for the best computer program of 
the year.'')  He
says that, apart from the benefit to be gained for the readers of our
programs, he finds that treating programs in this manner actually helps
to make them  run smoothly on the computer. (``Because you get it
right when you have to think about it that way.'') 

He gave us a reprint of ``Programming Pearls'' by Jon Bentley, from
{\sl Communications of the ACM\/ \bf29} (May 1986), pages 364--369,
and told us we had best read it by Wednesday since it will be an important
topic of discussion.  Don, who was `guest oyster' for this installment of
``Programming Pearls,'' warned us that ``this represents the best thing to
come out of the \TeX\ project.  If you don't like it, try to conceal your
opinions until this course is over.''

Bentley published that article only after Don had first published the idea
of ``literate programming'' in the British {\sl Computer Journal}. (Don says that
he chose the term in hopes of making the originators of the term
``structured programming'' feel as guilty when they write illiterate
programs as he is made to feel when he writes unstructured programs.) When
Bentley wanted to know why Don did not publish this in America, Don said
that Americans are illiterate and wouldn't care anyway.  Bentley seems to
have disagreed with at least part of that statement.  (As did many of his
readers: The article was so popular that there will now be three columns a
year devoted to literate programming.)

As Don began explaining the ``{\tt WEB}'' system, he restated two previously
mentioned principles: The correct way to explain a complex thing is to
break it into parts and then explain each part; and things should be
explained twice (formally and informally).  These two principles lead
naturally to programs made up of modules that begin with text (informal
explanation) and finish with Pascal (formal explanation).

The {\tt WEB} system allows a programmer to keep one source file that can
produce either a typesetting
file or a programming language source file, depending
on the transforming program used.

Monday's final topic was the ``blight on the industry'': user manuals.  Don
would like us to bring in some really stellar examples of bad user
manuals.  He tried to find some of his favorites but found that they had
been improved (or hidden) when he wasn't looking.  While he could have
brought in the improved manuals, bad examples are much more fun.

He showed a brand-new book, {\sl The AWK Programming Language}, to illustrate
a principle often used by the writers of user manuals: Try to write for
the absolute novice.  He says that many manuals say just that, but then
proceed to use jargon that even some experts are uncomfortable with.
While the AWK book does not explicitly state this goal, the authors
(Aho, Weinberger, Kernighan $=$ AWK) told him that they had this goal in mind.

But the book fails to be comprehensible by novices. It
fails because, as Don says, ``If you are a person who has been in the
field for a long time, you don't realize when you are using jargon.''
However, Don says that just because the AWK book fails to meet this goal
does not mean that it isn't a good book. (``Perhaps the best book in
Computer Science published this
year.'') He explains this by saying, ``If you try to write for the
novice, you will communicate with the experts---otherwise you communicate
with nobody.''

\beginsection 11. [Literate Programming (1)] \pmr October 21

Don opened class with the good news that Mary-Claire van Leunen
has agreed to help read the term papers and drafts thereof, despite
the fact that her name was incorrectly capitalized in last week's
notes. 

Returning to the subject of ``Literate Programming,'' Don said that it
takes a while to find a new style to suit a new system like {\tt WEB}. When
he was trying to write the {\tt WEB} program in its own language he tore up
his first 25~pages of code and started again, having finally found a comfortable
style. He digressed to talk about the vicious circle involved in
writing a program in its own language. To break it, he hand-simulated the
program on itself to produce a Pascal program that could then be used
to compile {\tt WEB} programs. The task was eased because there is obviously
no need for error-handling routines when dealing with code that you have
to debug anyway. But
there is also another kind of bootstrapping going on; you can evolve a
style to write these programs only by sitting down and writing
programs.  Don told us that he wrote {\tt WEB} in just two months,
as it was never intended to be a polished product like \TeX.

We spent the rest of the class looking at {\tt WEB} programs that had been
written by undergraduates doing independent research with Don
during the Spring. We saw how
they had (or had not) adapted to its style. Don said that he had got a
lot of feedback and sometimes found it hard to be dispassionate about
stylistic questions, but that some things were clearly wrong.  He
showed us an example that looked for all the world just like a Pascal
program; the student had obviously not changed his ways of thinking or
writing at all, and so had failed to make any use of the features of
the system. The English in his introductory paragraph also left a lot
to be desired.

Don showed us his thick book {\sl \TeX: The Program\/}---a listing of the
code for \TeX, written in {\tt WEB}. It consists of almost 1400 modules.  The
guiding principle behind {\tt WEB} is that each module is introduced at the
psychologically right moment. This means that the program can be
written in such a way as to motivate the reader, leaving {\tt TANGLE} to
sort everything out later on. 
[The {\tt TANGLE} processor converts {\tt WEB} programs to Pascal
programs.]
After all, we don't need to worry about
motivating the compiler. (Don added the aside that contrary to
superstition, the machine doesn't spend most of its time executing
those parts of the code that took us the longest to write.) It seems to
be true that the best way in which to present program constructs to
the reader is to use the same order in which the creator of the program
found himself making decisions about them. Don himself always felt it
was quite clear what had to be presented next, throughout the entire
composition of this huge program. There was at all points a natural
order of exposition, and it seems that the natural orderings for
reading and writing are very much the same.

The first student hadn't used this new flexibility at all; he had
essentially just used {\tt WEB} to throw in comments here and there.

A general problem of exposition arose: How are we to describe the
behavior of a computer program? Do we see the program as essentially
autonomous, ``running itself,'' or are we participants in the action?
Our attitude to this determines whether we are going to say `we insert
the element in the heap' or `it inserts the element \dots'. Don favours
`we'; at any rate one should be consistent.

Students used descriptors and imperatives for the names of their modules;
Don said he favours the latter, as in 
$\langle\,$Store the word in the dictionary$\,\rangle$,
which works much better than $\langle\,$Stores the word in the
dictionary$\,\rangle$.
On the other hand,
where a module is essentially a piece of text with a declarative 
function---a~list of declarations, say---we should use a descriptor to name it:
$\langle\,$Procedures for sorting$\,\rangle$.

Incidentally, it is natural to capitalize the first letter of a module name.

One student used the identifier `{\it FindInNewWords\/}'. This looks 
comparatively bad in
print: Uppercase letters were not designed to appear immediately
following lowercase ones. Since the use of compound nouns is almost
inevitable, {\tt WEB} provides a neat solution. It allows a short underscore
to be used to conjoin words like \\{get\_word}. (Since the Pascal compiler
will not accept identifiers like this, {\tt TANGLE} quietly removes the
underscore.) Don told us that Jim Dunlap of
Digitek, who made some of the best
early compilers, invariably used identifiers forty-or-so characters
long. The meaning was always quite clear although no comments
appeared in any of Dunlap's code.

Each module should contain an informal but clear description of what
it actually does. A play-by-play account of an algorithm, a~simple
stepping through of the process, does not qualify. We are trying to
convey an intuition of what is going on, so a high-level account is
much more helpful.

We saw several modules that were much too long. Don thinks that a
dozen lines of code is about the right length for a module. Often he
simply recommended that the students cut the offending specimens into
several pieces, each of more manageable size. The whole philosophy
behind {\tt WEB} is to break a complex thing into tractable parts, so the
code should reflect this. Once you get the idea, you begin writing
code piecewise, and it's easier.

We saw an example in which the student had slipped into ``engineerese''
in his descriptive text---all conjunctions and no punctuation. This
worked for James Joyce, but it doesn't make for good documentation.
One student had apparently managed to break {\tt WEB}---the formatting of
{\bf begin}s and {\bf end}s came out all wrong. Heaven knows what he did.

One student put comments after each {\bf end} to show what was being
ended, as {\bf end} $\{${\bf while}$\}$. This is a good idea when writing
ordinary Pascal, but it's unnecessary in {\tt WEB}. 
Thus it's a good example of a convention
that is no longer appropriate to the new style; when you change
style you needn't carry excess baggage along.

Don had more to say about the anthropomorphization of computer
systems.  Why prompt the user with `{\tt Name of file to process?}' when we
can have the computer say `{\tt What file should I process?}'? Don
generally likes the use of~`{\tt I}' by the computer when referring to
itself, and thinks this makes it easier for users to conceptualize
what is going on. Perhaps humans can think of complex processes best in
terms of demons in boxes, so why not acknowledge this? Eliza, the AI~program
 that simulates a certain type of psychiatrist, managed to fool
virtually everyone by an extension of this approach. Eliza may or may
not be a recommendation for anthropomorphisms, or for psychiatry. There are
those, such as Dijkstra, who think such use of~`I' to be a bad thing.

As in the case of maths, don't start a sentence with a symbol. So
don't say `{\it data\/} assumes that \dots'---it can easily
be rewritten.

We saw several programs by one student who had developed a very
distinctive and (Don thought) colourful style. His prose is littered
with phrases like ``Oooops! How can we fix this?'' and ``Now to get down
to the nitty-gritty.''  This  stream-of-consciousness style really
does seem to motivate reading, and helps infect the reader with the
author's obvious enthusiasm. There were a few small nits to pick with
this guy though: His descriptions could often be more descriptive.
Why not call a variable \\{caps\_range} instead of just {\it range\/}? Don also
had to point out to him that `complement' and `component' are in fact
two different words.

  In {\tt WEB} you can declare your variables at any point in the program.
Don thinks it is always a good idea to add some comment when you do
so, even if only a very cursory explanation is needed.

A note about asterisks: Be warned that typeset asterisks tend to appear
higher above the line than typewritten ones, so your multiplication formul\ae\
may come out looking strange. Better to use $\times$ for multiplication,
and to use a typewriter-style font with body-centered `{\tt *}' symbols
instead of the `*' in normal typographic fonts.

Another freshman was digitizing the Mona Lisa for reasons best known
to insiders of Don's research project.
 Don pointed out that since the program uses a somewhat
specialized data structure (the heap) that might be unknown to the
readers, the author should keep all the heap routines together in the text so
that they can be read as a group while fresh in the reader's mind.
In {\tt WEB} we are not constrained by top-down, bottom-up, or any other
order.

This student capitalized the first letter of every word in titles of modules,
 even `And' and the like. This looks rather unnatural---it
is better to follow the newspaper-headlines convention by leaving such
words entirely in lowercase, and even better to capitalize only the first word.

Don thought it a good idea to use typewriter type for hexadecimal
numbers, for instance when saying `{\tt 3F} represents 63'. But leave the
`63' in normal type. This convention looks appropriate and provides a~kind 
of subliminal type-checking.

The words used in the documentation should match the words used in the
formal program---you will only confuse the reader by using two
different terms for the same thing.

It's a good idea to develop the habit of putting your {\bf begin}s and
{\bf end}s inside the called modules, not putting them in the calling
module. That is, do it like this:

\smallskip
\halign{\qquad\qquad#\hfil\cr
{\bf if} \\{down} $=4$ {\bf then} $\langle\,$Punt$\,\rangle$;\cr
\qquad $\vdots$\cr
\noalign{\smallskip}
$\langle\,$Punt$\,\rangle =$\cr
\qquad {\bf begin} snap;\cr
\qquad place;\cr
\qquad kick;\cr
\qquad {\bf end}\cr
}

\smallskip\noindent
Not like this:

\smallskip
\halign{\qquad\qquad#\hfil\cr
{\bf if} {\it down\/} $=4$ {\bf then}\cr
\qquad {\bf begin} $\langle\,$Punt$\,\rangle$;\cr
\qquad {\bf end}\cr
\qquad $\vdots$\cr
$\langle\,$Punt$\,\rangle =$\cr
\qquad snap;\cr
\qquad place;\cr
\qquad kick\cr
}

\smallskip\noindent
Incidentally, appalling bugs will occur if we mix the two conventions!
\beginsection 12. [Literate Programming (2)] \pmr October 23

One of the chief aspects of {\tt WEB} is to
encourage better programming, not just better exposition of programs.
For example, many people say that around 25\%
of any piece of software should be devoted to error handling and
user guidance. But this will typically mean that a subroutine might
have 15 lines of `what to do if the data is faulty' followed by one or
two lines of `what to do in the normal course of events'. The subroutine then looks
very much like an error-handling routine. This 
fails to motivate the writer to do a good job; his heart just isn't in
the error handling. {\tt WEB} provides a solution to this. The procedure can
have a single line near the beginning that says $\langle\,$Check if the data
is wrong ${\scriptstyle 28}\,\rangle$ 
and points to another module. Thus the proper focus is
maintained: In the main module we have code devoted to handling the
normal cases, and elsewhere we have all the error-case instructions.
The programmer never feels that he's writing a whole lot of stuff
where he'd really much rather be writing something else; in module~28,
it feels right to do the best error detection and recovery.
Don showed us
an example of this from his undergraduate class in which a routine had
two references of the form

\smallskip
\halign{\qquad\qquad #\hfil\cr
{\bf if} \dots {\bf then} \dots {\bf else} \\{char\_error}\cr
}

\smallskip\noindent
pointing to a very brief error-reporting module. 

We looked at a program written by another student who had the temerity to
include some comments critical of {\tt WEB}. Don struck back with the following:

\smallskip
\display 30pt::
  It is good practice to use italics for the names of variables when they
appear in comments.

\smallskip
\display 30pt::
  Let the variables in the module title correspond to the local
  parameters in the module itself.

\smallskip
\display 30pt::
  According to this student's comments, his algorithm uses `tail recursion'.
  This is an impressive phrase, helpful in the proper context;
but unfortunately that is not the kind of recursion his program uses.

\smallskip
However, Don did grant that his exposition was good, and said that
it gave a nice intuition about the functions of the modules.

We saw a second program by the same student. It had the usual
sprinkling of ``wicked whiches''---`which's that should have been
`that's. The purpose of the program was to ``enforce'' the triangle
inequality on a table of data that specified the distances between
pairs of large cities in the US\null. Don commented here that 
his project (from which these programs came) intends to publish
interesting data sets so that researchers in different places
can replicate each other's results.
He also observed that a program running on a table of ``real data,'' as
here (the actual ``official'' distances between the cities in question)
is a lot more interesting than the same program running on ``random
data.''  Returning to the
nitty-gritty of the program, Don observed that the student had made a
good choice of variable names---for instance `{\it villains}' for those
parts of the data that were causing inconsistencies. This fitted in
nicely with the later exposition; he could talk about `cut~throats' and
so forth. (Don added that we nearly always find villainy
pretty unamusing in real life, but the word makes for a witty exposition
in artificial life; the English language has lots of vocabulary just 
waiting for such applications.)

Don wondered aloud why it is that people talk about ``the $n^{\rm th}$ and 
$m^{\rm th}$
positions'' (as this student had) thereby reversing the natural (or at
any rate alphabetical) order?

He also pointed to an issue that arises with the move from typewriters
to computer typesetting---the fact that we now distinguish between
opening and closing quotes. We saw an example where the student had written
''main program''. To add to the confusion, different languages have
different conventions for quotes; in German they appear like this:
\lower1ex\hbox{''}The Name of the Rose``.
How to represent this in a standard ASCII file remains a mystery.

Back to the triangle inequality. Don pointed out that one obvious
check for bad data in the distance table follows from the fact that
the road distance can not be less than a Great Circle route. (``It
could, if you had a tunnel'' commented a New Yorker in the audience.)
The student had written a nice group of
modules based on this fact, and it
illustrated the {\tt WEB} facility of being able to put displayed equations
into comments.

``So {\tt WEB} effectively just does macro substitution?''\ asked another
member of the class. Exactly, said Don. 
In fact the macros
he uses are not very general---they really allow only one parameter.
This means he doesn't need a complex parser, but in fact one can do a
great deal within this restriction. For instance, it is not difficult
to simulate two-parameter macros if we wish.

Someone in the class commented that it seemed a little strange to put
variable declarations in a different module from their use. Don said
that this was OK as long as they are close to their use, but large procedures
should have their local variables ``distributed'' as the exposition
proceeds.

Don recalled that older versions of Algol allowed you to
declare a variable in the middle
of a block.  This fits in nicely with the {\tt WEB} philosophy, but
unfortunately cannot be done in modern Pascal. Indeed, Don became
painfully aware of the limitations of Pascal for system programming
when he was writing {\tt WEB}---you can't have an array of file names, for
example. He got around them, though, with macros.

One example of improving Pascal via macros is to define (in {\tt WEB})

\smallskip
\halign{\qquad\qquad #\hfill\cr
\\{string\_type}({\tt{\#}}) $$ {\bf packed array} $[1\,.\,.\,${\tt{\#}}$]$ {\bf of} \\{char}\cr
}

\smallskip\vskip-\parskip\noindent
so that you can say things like

\smallskip
\halign{\qquad\qquad #\hfill\cr
\\{name\_code}: \\{string\_type}(2)\cr
}

\smallskip\vskip-\parskip\noindent
when declaring a two-letter string variable.

At this point, prompted by a note from Tracy, Don announced that 23
copies of the {\sl Handbook for Scholars\/} had arrived in the Bookstore,
with more to come. A resounding cheer echoed throughout Terman.

Don commented that the student had given a certain variable
the name `{\it scan}'.
Since this variable was essentially a place marker, Don thought that a
\vadjust{\eject}% preserve old page break
noun would be much better than a verb---`{\it place}', perhaps. Let the function
determine the part of speech; think of it as a kind of Truth in
Naming. Verbs are for procedures, not data.

The last student had written a program to handle graph structures based
on encounters between the characters in novels. He too had made the ''quote
mistake''.  The student gave a nice
characterization of the input and output of the program, using the
typewriter font to illustrate data as it appears in a file.

This student also showed a bit of inconsistency in the use of `it'
and `we' as the personification of his program. We seem to be
finding the same old faults over and over now, Don said, so perhaps
that indicates that we have found them all. Discuss.
\beginsection 13. [User manuals] \pmr October 26

We moved  on to the subject of user manuals. Don was disappointed that
nobody had responded to his request in a previous lecture
to give him glaring examples of bad ones---either they are
being much better written these days or we hadn't taken him seriously.
So instead, Don produced mini-sized user manuals written by
CS~graduate students for his class CS\thinspace304 earlier this year. The
students had had to tackle five weird and wonderful problems in ten weeks;
 one of the problems had been to design and implement some
software and to write a one-page user manual for the `Digiflash'
display system. This is the kind of thing you see in Times Square,
and increasingly in bars and post offices, in which news and advertising
flows across a sort of dot-matrix screen. In this case, the screen
was to be a simple array of 8 by 256 pixels. The students had only
two weeks in which to write the system and manual, which were then subject
to the ultimate test, the truly {Na\"\i ve} User. The idea was that
the user would need no understanding of computers or of writing,
but should still be able to use the system to produce a variety of
visual effects. The students divided themselves into four teams and
so we saw four solutions to the problem.

A common failing was that terminology that seemed perfectly transparent
meant nothing to Don's wife: ``Menu'' and ``Scrolling'' for example.
Such terms are so familiar to CS~people, it never occurs to them that
these are actually technical terms.

Don went through the solutions in ascending order of competence. The
class reaction to this discussion might almost have led one to believe
that some of the authors were sitting amongst us.

Don digressed on the subject of `i.e.'. Is it formal, he asked, or is
it part of the language? He confessed that he was considering taking
all the `i.e.'s out of his new book. One thing he does know: You should
always put a comma after `i.e.'. (Except in this instance.)

The first solution could be described as a very {\it logical\/} approach,
almost an archetypically CS solution. The manual was essentially
a hierarchy of definitions. The writers talked about MESSAGES (or
MESSAGEs---consistency was not their watchword) when they wanted
to say: `here are objects that have a special meaning for us and
whose definition you ought to know'. But, said Don, formal
definitions are not the way to explain something to a novice. 

This write-up apparently thought the phrase LEFT-INDENTED to be
self-explanatory, although it meant `flush with the left margin'.
(Left unindented?)
The user was prompted to enter data by the words
`{\tt Type of message (1-6):}'.
Why should there be numeric types? Sentences like ``And now you
should ENTER the data'' do nothing to help the user relax---the capitals
look too much like DANGER SIGNS.

Don's wife commented that one thing she always needs to know is
``How do I get out of a mess if I do something wrong?'' Don said that
this is something manuals almost never explain---perhaps it never
occurs to their authors that somebody will eventually want to stop
playing with their program. The solution we were looking at did have
a one-line description of how to EXIT, but Don said
even this is jargon.

The second solution was Digiflash$^{\scriptscriptstyle\rm TM}$. It had a
good introductory and motivational paragraph, albeit with a whole crowd
of `which's that should have been `that's. Unfortunately it claimed
that the system was very easy to use and understand---a~claim that can
rebound by making the user feel stupid. There was a major flaw in the
program in that one was expected to hit OPTION-B to enter `bold' mode,
and then OPTION-B again to leave it. Don thought it would be far more
natural to type OPTION-N (for `normal') instead. Option-V was
``reverse video''---another jargon word, and why wasn't this
OPTION-R? There were some cute options though: `M'~for `slowly materializing'
text, and an assortment of small animal logos that could be made to appear.

The third solution was the DiJKSTra system, so named to keep it
sufficiently Dutch (obscure in-joke, please ignore). The authors had
a nice use of the phrase `flashing bar' instead of the more technical
`cursor' (though for some reason they still felt impelled to define
the latter as the former), and likewise they said `hit return' instead
of `enter' (or worse, ENTER). They also kept their sentences nice and
short. Another good idea was that the manual invited the user
to type~`{\tt ?}'\ to get an online demonstration, thus sparing us a painful
description of such arcane concepts as boldface italic reverse video
fade-in mode and incidentally helping to keep the manual concise. If a
picture is truly worth a thousand words, said somebody, then an animated
demonstration must be worth at least a paragraph. One problem with this
system was that the user is prompted for five or so parameters every time
he enters a new line, and the defaults are fixed. Wouldn't it be better,
asked Don, to default to the style used for the preceding line?

The last solution (though not even typeset, much less \TeX ed) Don declared
to be the best. There was a good overview and a step-by-step description
of the system; very friendly looking. Crisp sentences. Easy to skim.
Helpful redundancy and diagrams. Don said that there's really nothing
much you can do about the reader who insists on starting at a random
point in the middle of a text. When he surreptitiously watches people
looking at his books in the bookstore, he notices that they always start
in the middle somewhere, not at the preface where he wanted them to read
first.\footnote*{``As for those readers who do not know how to study my
composition, no author can accompany his book wherever it goes and allow only certain persons to study it.''
{\hfill--- Maimonides\parfillskip=0pt\par}}


There was a good use of a symbol in the text to indicate the control key,
and likewise diagrams of the keyboard to explain which keys to use for
left, right, next message, previous message, etc. It was also good to 
\vadjust{\goodbreak}%
emphasize that the control key must actually be held down while another
key is typed (that is, they are not simply typed successively). Perhaps the
main flaw was that the user was expected to realize that `up' meant, 
in effect, `go to the previous message'; and `down', `go to the next
message'. To those unfamiliar with full-screen editors, this mightn't
be obvious. There was a nice use of icons to describe scrolling up, down,
left, or right though. One obscurity was the advice

\medskip
\def\boxit#1{\vbox{\hrule\hbox{\vrule\kern3pt
 \vbox{\kern3pt#1\kern3pt}\kern3pt\vrule}\hrule}}
\setbox1=\vbox{\hsize 9pc\noindent\strut
{\tt THIS IS VISIBLE HERE}
\strut}

\halign{\qquad\qquad#\hfil\quad&#\hfil\cr
\boxit{\box1}&\raise7pt\hbox{\tt BUT NOT HERE}\cr
}

\vskip-3pt
Don declared that he didn't know what this was supposed to mean; it
would be a lot better to say `Extra long messages can be seen if you
make them move'.

It's good to have plenty of comments like `Good luck!' and `Enjoy!'
scattered here and there. But Don thought the phrase
`this system has been carefully redesigned not to bite' hardly
reassuring.
\vskip 0pt minus 8pt
\beginsection 14. [Galley proofs] \pmr October {26 (continued)}

In Don's mailbox today he found galley proofs from the ACM, to be corrected
and returned within 48~hours of this time two weeks ago. Unfazed
by this injunction he went over the text with us. The Algol programs
seemed to be laid out properly. There were occasional cryptic marginal
notes: `Bad proof, Camera copy OK'. He took this to mean that his copy
was made by a laser printer instead of a phototypesetting machine.
We learned that `Au' means not gold but `author' in the copy-editing
world. The copy editor had substituted `cleverer' for Don's `more
clever', citing Fowler. Don sighed and recalled the occasion that
{\sl Scientific American\/} had replaced his `more common' with
`commoner'. It was noticeable that the copy editor was not going to
change anything without Don's specific approval---not even removing
the first `of' in `\dots several possible of values of the variable~$n$
\dots'. 
Don told us that at the moment all papers are re-typed by the publishers,
except for one or two AI journals that have used \TeX\ for several
years. But next year a math journal will be adopting a policy in which
the author's text is manipulated electronically throughout the
whole process. This should speed publication and reduce errors and
costs. 

Some of the notes in the galley were signed `Ptr', that is
`printer', and asked `OK?'. Don answers affirmatively by circling
the `OK'. At one point he was asked to sanction the insertion
of a whole new sentence. Apparently he had made reference
to Figure~14 before Figure~13, and his approval was sought to make
an extra comment first about `Figures 13--16'. (The extra comment
was wrong but fixable.)

The publishers also insisted on more details in his bibliography.
They wanted to know, for example, exactly where and when a conference
had taken place. Someone in the class pointed out that Mary-Claire
van Leunen recommends omitting 
the location of conferences. Don replied that libraries often
nowadays index conferences by city for those poor souls
who can remember nothing else about them; so such information was useful.
He observed that people have a great tendency to copy citation information
blindly into their own papers, and so errors propagate unchecked.
When Elwyn Berlekamp wrote his book on coding theory, he found that nearly
half the information in bibliographies of papers was wrong! Don wrapped
up the galley proof discussion by showing us a few tables of (somewhat)
standardized proof-readers' symbols.
\beginsection 15. [Refereeing (1)] \pmr October 30

Today Don spoke about the refereeing process. A paper submitted to an
academic journal is usually passed to one or more
referees by the editor of the
journal. Each referee is intended to be an expert in the relevant
field, and thus in a position to tell the editor whether or not the
paper merits publication. Don pointed out that many of us will one day
find our papers being subject to just this scrutiny; and some of us
will certainly be asked to assess other people's papers ourselves.

Don talked about his now-famous research on ``The Toilet Paper Problem.''
This was first published in the {\sc Monthly},  and
as Don pointed out to the Editor in his cover letter, many of its
readers probably keep their copies in the bathroom anyway. The editor (Halmos)
replied a little gravely that ``jokes are dangerous in our journal,''
and asked Don to think twice about
 the scatological references. Don did agree to
change the section names---which originally continued the pun with
such headings as `An absorbing barrier',
`A~process of elimination', and `Residues'---to
innocuous equivalents, but kept the title intact. In justification of
this, Don pointed out to the editor that two talks had already been
given on his results under this title, and that the material had been
widely circulated and discussed. ``Your toilet paper is accepted,''
replied Halmos. Don confessed that he still has occasional doubts
when he catches sight of the title amongst his papers, but the deed is
done now. Still, it did get reasonably good reviews, even in Russia.

Don showed us an article entitled `Rules for Referees' by Forscher,
published in {\sl Science\/} (October~15, 1965).
 These rules constitute a rather traditional
view, Don said, and emphasize the legal rights and responsibilities of
all concerned. Don thought that this seems a lot more oriented to the
advancement of careers rather than of science as such; the right
reason to publish is to build upon the results of others and provide a
foundation for future research. It is a sad truth, said Don, that an
editor can all too easily find himself spending a great deal of time
dealing with those authors whose papers don't merit publication, for
it is usually very hard to convince them of the fact. Rebuttals are
followed by counter-rebuttals, and so on. But fortunately this doesn't
happen so often that the whole business of science gets bogged down.

The referee is conventionally regarded as a sort of ``expert witness,''
whose task is to tell the editor whether the paper deserves to be
published or not. The first criterion should be originality;
is the material presented a genuine advance on previous work?

Don urged referees to see their primary responsibility as being to
authors and readers, not just to editors. Don himself decided long ago
that he would put more of his efforts into refereeing papers before
their publication than into reviewing published papers. Don hoped that
he could thus do his bit to encourage high standards of writing in
Computer Science and help the field win respect. These days there are
more good people to go around, both in refereeing and reviewing.

In the 1960s Don circulated a list of `Hints to referees' to try to
encourage good practice. He would like to show us that list, but not a copy
can be found. Don has written to some of the people to whom he sent
it, so it is possible that a copy will turn up before the end of the
quarter.

Don disagreed with our guest speaker, Herb Wilf, who had said that he
would tolerate more stylistic lapses in the {\sl Journal of Algorithms\/} than
in the {\sc Monthly}. Authors, thought Don, should always be
encouraged to do better; he could recall only a single occasion when,
as a referee or editor,
 he could recommend no improvements at all. (The author in
this case was George Collins writing for the ACM journal.) Let us
publish journals to be proud of, he said. This was sadly not true of
Computer Science in the early 60s. Some published results were just
plain wrong; or a correct result was incorrectly proved; or a paper
simply contained no results at all! Contrast this state of affairs,
said Don, to the math journals that were published in the 20s and~30s---leafing
 through them at random we see a host of familiar names
and theorems, because so much of what was written then was polished,
significant, and worth reprinting in textbooks. The same could not be
said of today's efforts---perhaps we have grown increasingly tolerant
of substandard work.

Referees should try to be teachers, said Don. The author you criticize
today will be writing another paper tomorrow, so try to help him
improve his writing. Unfortunately, referees will often be
over-critical and make quite tasteless comments on papers, knowing
that they do so under a cloak of anonymity. This only angers the
author and he learns nothing. Try to supply constructive criticism,
Don urged. These human issues are not discussed in Forscher's `Rules'.

In addition, the referee can contribute to the technical quality of a
paper by giving references to related work of which the author was
apparently unaware, or improving the results.
 Don himself has contributed results anonymously
to papers---more
than one author has had to add a footnote: ``My thanks to the
referee for Theorems 4, 5, and~6.'' Don was always pleased to feel
that by doing this the image of the journal was improved. A~journal
should be seen as a source of wisdom, so let us be cooperative toward
this end, not legalistic.

How should one choose a journal to which to submit a paper? Don
thought the answer is to look for the one with the best referees, not
the one with the least critical editor. After all, an author
presumably wants to know whether he has really made a contribution to
his field.  So find a journal that has handled papers on related
subjects.

Someone asked whether the letters that appear in journals are also
refereed. Don said that sometimes they are, sometimes not. There is
often nothing to distinguish letters from short papers.

Some journals do not use referees at all. Their readership must be
willing to wade through a great deal of nonsense. The ACM did at one
time have plans to publish an unrefereed journal, but to Don's relief those plans
never came to fruition.

At this point Don confessed to a sneaky trick he had pulled way back
in the 60s. At that time he had just begun to edit
 the programming languages
material for the {\sl Communications of the ACM\/} and the {\sl Journal 
of the ACM}.
He had no way of knowing which of his referees were any good, so
in an effort to calibrate them he sent all a copy of the same paper
and solicited their opinions. Don had already refereed the paper
himself, of course, and found it a very badly written exposition of a
very interesting algorithm (due to someone besides the author).
 As such, it was certainly worthy of the
referees' study. 

We looked at some of the results.
One commentator simply went through line by line, listing his
complaints point by point. Another made much more general comments:
``A~paper with this title should contain (1)~a complete
algorithm; (2)~a proof or at least a convincing explanation of
correctness; (3)~a statement of limitations on the algorithm's
applicability. None of these can be found here.''
A~third said that the paper contained little
that was new, and supplied a substantial bibliography for the author
to go away and study.  The next referee liked the algorithm and
recommended the paper for publication. Don was  surprised; he
had mistakenly 
thought that this referee had originally invented the algorithm
himself! Another critic dismissed the paper as `incredibly poorly written'.
Another commented it was not a paper to be read, but rather
a puzzle to be solved.

Don told us that as a result of his experiment, the algorithm actually
became quite well-known.

On one occasion Don ripped into a paper with a long report on its
failings, and was later told by the author that those constructive
comments had changed his life: The author had resolved that from then
on he was going to study writing and give a lot of attention to exposition.
This nameless individual
went on to become a renowned professor at a great (but here equally
nameless) university, and an editor of a fine journal.

In answer to a question, Don said that if the content of a paper was
obviously bad, he would not spend time reviewing the grammar. But in
studying a paper that really has something to say, then he would also
try to ensure that it was said as well as possible.

Don showed us some referees' reports on one of his recent papers. The
editor had told him that these were `mostly positive'---in fact two
were in favour and one against. The referees in this case had been
asked to answer a specific list of questions about his paper---Don
said that this tedious format might at least cause a referee to consider
issues he might otherwise have forgotten about. The referees did
agree that Don hadn't made enough reference to earlier work in the
subject. This didn't surprise him; the paper was his first venture into
an unfamiliar field. The referees were helpful enough to comment now
and then that they had particularly enjoyed certain sections, and they
provided a whole slew of references to other work---references that
Don said had led to some new ideas. They were also able to point out subtle
technical errors; Don had to write a program to convince
himself that one in particular of these criticisms was valid. Finally, we were
amused to see that the referees were asked to assign an overall rating
to the paper by checking one of a series of boxes, ranging from (as
the most lavish praise) `accept', down through `accept with major
modifications' and `accept with minor changes' to `Reject: submit to
\_\_\_\_\_\_\_\_'. When checking this last box (the most damning
indictment), the referee was asked to suggest a less prestigious
journal that might publish such inferior work. By such a downward
filtering, even the most appalling paper stands some chance of finding
its place in the pages of what Dijkstra has characterised a
``Write-Only Journal.'' With four new scientific papers being published
every minute throughout the world, we can rest assured that many do~so.
\beginsection 16. [Refereeing (2)] \tll November 2

Today's handout, ``Hints for Referees'' by D. Knuth (see \S{17} below),
 could have been
subtitled ``Ask and ye shall receive.'' Last Friday Don mentioned in class
that he could find no copy of this document, but when he returned to his
office immediately after class he found it sitting on his desk.  (To be truthful, he
thinks this copy has gone through a few revisions since it left his hands;
he no longer recognizes the style of all the comments.)

Before demonstrating to us how highly he esteems referees and the lengths
to which he will go in order to get referees, Don told us to note an
important date on our calendar: On Wednesday, November~18, we are to
turn in the first drafts of our Term Papers (``The closer to the final
version, the better'').

The identities of the referees for a journal paper are usually hidden from
the author.  Is the identity of the author ever hidden from the referees?
In a few journals, yes.
 Don is well aware that the name written just below
the title of a paper can strongly affect the reader's reaction, so he
submitted a journal paper using the sobriquet, Ursula~N. Owens.  (Those of
us who have read Agatha Christie's {\sl And Then There Were None\/} realize
that his near-use of the name U. N. Owen is a pleasant allusion.)

Don doesn't always resort to pseudonyms, but neither does he always get
his papers refereed.  On occasion he has recruited his own referees when
he found out that his target publication was supplying none.  As an
example, his paper on Literate Programming for the British {\sl Computer
Journal\/} generated no referee reports (and no
feedback of any kind); they went right into print.

Don repeatedly stated how invaluable he found ``feedback from a motivated
reader.'' He showed us the comments that ``Ursula'' got on her paper, and
they were pertinent in more than one way.  The referee found typographic
errors and suggested notation changes, as well as finding errors where
there were none present.  The last set of comments were more important
than they might at first seem because they pointed out where Don's
presentation was misleading or overly subtle.  

In another example, the referee significantly improved one of the theorems
while remaining anonymous.  Instead of being content with an
acknowledgment to an anonymous contributor, such a referee could be
jealous and publish his own competing paper.  

In contrast to such substantive contributions, Don showed us another
example with suggestions that he called ``facile generalizations''
(terminology attributed to P\'olya): generalizations that are merely
mechanical manipulations of a given argument without creating new insight.

Don says that refereeing is a ``cooperative effort---a correspondence
between tens of thousands of
world authorities,'' and he is perfectly willing to exploit the system
by letting referees improve his papers as he helps with theirs.

He showed us a series of letters passing between himself and the {\sl Journal
of Number Theory}.  He had produced a result that seemed novel (could not
be found by exploring the standard pathways in the Math Library), but
since Number Theory is not his field of expertise, Don was unwilling to
claim that the result was not a duplication.  He told this to the editors
of the journal and asked for feedback.  (``I~put in a lot of time reviewing
other people's papers.  This is my chance to get some of that time back.'')

The referee reports on that paper found references that Don ``couldn't have
found in a million years.''  The results were similar but not identical, so
the referee offered to check with a famous Russian expert.  As Don was
availing himself of this offer, someone else was publishing on the same
subject. (``You have to decide, do you want speedy publishing or rigorous
checking?'')

Finally, he showed us two examples that dealt with ambiguity.  In the
first, he and David Fuchs had written a paper entitled ``Optimal Font
Caching.''  One of the referees pointed out that this paper could be about
the caching of optimal fonts, or the best of all possible caching
mechanisms for fonts.  An analogous title ``Common Sense Amplifiers''
was cited.
(Don and Dave solved this problem by changing the
title to ``Optimal Prepaging and Font Caching.'')  In the second, he had to
cope with the {\sl IEEE Journal on Coding Theory}'s penchant for writing out
the words `one' and `zero' for the symbols~`1' and~`0'.  Since `one' is an
English pronoun, he was forced to use the word `unity' in one place to make
the text unambiguous.

\vfill\eject
\beginsection 17. [Hints for Referees]

\vfill\eject
\beginsection 18. [Illustrations (1)] \pmr November 4

Today, Don said, we are going to talk about the use of pictures and
illustrations in mathematical writing, 
and about the problem of ``getting across the feeling of complicated
algorithms.''

But first,
by popular demand, Don showed
us his first publication. This was a description of a system of
weights and measures known as the Potrzebie System, which appeared
in the pages of {\sl MAD\/} magazine in 1957. Any resemblance to the Metric
System is purely coincidental. It is an extremely natural and logical
system, Don told us.  For example, the units of time were named after
the editors of {\sl MAD} (the new editors substituted their own names). He
felt there was also a need for new units of counting, and so coined the
MAD; 48~things constitute one MAD (or~49, a~baker's MAD). Don didn't
publish a better illustrated work until 
\TeXbook, he claimed, nor
another paid one until he wrote for ACM
{\sl Computing Surveys\/} some 12~years
later. {\sl MAD\/} forked over no less than \$25 for this research paper,
no mean sum thirty years ago.  
`The Potrzebie System' still heads the list of publications on his~C.V.

{\sl MAD\/} inexplicably declined Don's second article, ``RUNCIBLE:
Algebraic Translation on a Limited Computer,'' 
which was picked up by {\sl Communications of the ACM\/} in 1959.
Perhaps this was
because it contained what even Don admits is probably one of the worst
``spaghetti''
flow charts ever drawn. Not only does the chart attempt to illustrate
the entire algorithm, but it contains an error (a misdirected arrow).
The article included a play-by-play account of the algorithm,  which 
helped ameliorate the obscurity of the chart. Back in
those days, Don now admits, he didn't know any better. Likewise, full
of youthful enthusiasm at 
being able to communicate improvements
 on a previously  published  algorithm
(Don was a Junior then), he failed to mention his coauthors in the
paper; Don did the  writing but other students contributed
illustrations and most of the ideas of the algorithm.
 At the time he had no notion there was
academic prestige to be gained through publication, Don confessed.
This is, he said, a common mistake among young authors who frequently
overlook proper acknowledgments in their haste to get the news out.
At the other extreme, he recalled, Paul Erd{\H o}s once cited a railroad
car porter as a coauthor.

Diagrams are good if they are kept small, said Don. As an example of
a useful
one that is {\it not\/} small, he showed us a fold-out syntax chart for a
slightly extended version of the Algol~60 language. It does convey
quite a good impression of what the language is, and gives
computer scientists something to hang on the wall where chemists put
their Periodic Tables.

Don's ``Programming Pearls'' article came up again. He had ended that
paper with the observation that the only fair test of his {\tt WEB} system
would be this: Someone should provide a challenge problem, and Don
would use {\tt WEB} to write an ultra-elegant solution to it. Jon Bentley
responded to this challenge; he devised such a problem and invited Don
to submit his solution for publication. Holding Don to his claim that
{\tt WEB} programs should be works of literature, Jon then published the
solution along with a literary critique. In this review Jon commented
that Don could have eased the exposition of his data structure with a
suitable diagram. Don agreed that this would have helped the reader
get a handle on it (he had described the data structure in words
only). He told us that diagrams were actually quite easy to do in {\tt WEB},
a~claim that was greeted with a certain skeptical laughter from the
class (all doubtless recalling hours spent wasted trying to get
tables {\it just\/} right).

Referring again to his `optimal prepaging' paper (which included a diagram in
which two approximately diagonal lines crawled across the page,
touching occasionally to indicate a page fault) Don told us that the
referee had complained that the figure was too detailed. Don disagreed
with this, saying that the detail was there for those who want to see
it, but could easily be ignored by those who don't. Don confessed that
he always has been very concerned with the minuti{\ae} of his subject, and
seldom thought any detail too trifling to bother with.

Don discussed a paper he had written with Michael Plass on \TeX's
algorithm for placing line-breaks in a paragraph [{\sl Software---Practice
\& Experience\/ \bf11} (1981), 1119--1184].
The main difficulty in writing the paper was: 
How to describe the problem and the new
algorithm? First of all, they chose a paragraph from one of Grimms'
Fairy Tales as ``test data'' with which to illustrate the process. As Don
remarked once before, it is better to use ``real'' data than ``sample
data'' that have in fact been cooked up solely to use as an example.
[{\sl Grimms' Fairy Tales}, along with the text of {\sl Harold and Maude}, are
kept online on {\tt SAIL}, an ancient and eccentric CSD computer.]
Corresponding to each line of any right-and-left-justified paragraph
is a real number, positive or negative, indicating the degree to which
the line had to be stretched or compressed to fit the space exactly.
In his paper, Don prints these numbers in a column beside his typeset
paragraph. Don used a couple of lines of the paper itself to show how
bad it looks if these adjustments are too extreme (and of course had
to tell the printers that this was a deliberate mistake, lest they
``correct'' it).\looseness=-1

Don outlined three basic algorithms: first fit (which essentially
packs the text as tightly as possible one line at a time);
 best fit (which can loosen it
up if this works better, but still works line by line);
 and optimum fit (optimal in the sense that
it minimizes the sum of the ``demerits'' earned by the various
distortions of each line,
taken over the paragraph as a whole).
 To describe this last algorithm, Don drew a
diagram. It is essentially a graph, each node on level~$p$ corresponding
to a different word after which the $p^{\rm th}$ line might be broken. Edges
run between nodes on successive levels, and are labeled by the
demerits scored by the line of text they define. The problem of finding
an optimal fit thus reduces to finding a least-cost path from the
top to the bottom node; well-understood search techniques can be used
for this. Don commented that certain ``demerit-cutoffs'' will
limit the number of nodes on each level and thus speed the algorithm.
This means that a solution in which one very distorted line permits
all the rest to be displayed perfectly might be missed.

If the above account is opaque, it only goes to show why diagrams can be
so useful.

The article includes
 histograms to illustrate how frequently \TeX\ generates
more-or-less distorted lines of text. As he explained, this was biased
by the fact that he would usually re-write any particularly ugly
paragraph. A~second histogram confirmed that the text was considerably
more distorted when it hadn't been hand-crafted to the line width that
\TeX\ was generating, yet the new algorithm was significantly better
than Brands~$X$ and~$Y$.

Finally, we saw an old Bible whose printers were so keen to fill out
the page width that they inserted strings of o's to fill up any gaps.

Don found many illustrative illustrations in the book {\sl The Visual
Display of Quantitative Information\/} by Tufte. He also recommended
{\sl How to Lie with Statistics\/} by Huff, which advises (for example)
that if you would impress your populace with the dazzling success of
the Five-Year Plan in increasing wheat production by 17\%, then draw
two sacks, the first 6~cm and the second 7~cm tall. The perceived
increase, of course, corresponds to the apparent volumes of the sacks,
and $7^3$ is 58\% larger than $6^3$. \dots

Don referred to Terry Winograd's book {\sl Language as a Cognitive
Process}.  Algorithms for parsing English sentences are there
illustrated as charts defining augmented transition networks or 
ATN's---nodes correspond to internal states, edges are transitions between
states and correspond to individual words. Winograd also has a nice
use of nested diagrams---boxes within boxes---to replace the more
traditional tree diagrams.

We saw a scattergram of smiley-faces of somewhat indeterminate
significance; a wit in SITN projected Don's amongst them.
The idea is that several dimensions of numeric data can be used to
control features on these faces; humans are supposedly wired to
read nuances in facial expressions quite easily.

Don showed us a table from his {\sl Art of Computer Programming\/} that
listed the many, many states of the Caltech elevator. He said he
wished that he'd been able to dream up a diagram to capture that example
more neatly: A~listing of events is the best way he knows to convey
the essential features of asynchronous processes.

 The third Volume of this tome does contain a large fold-out
illustration comparing the performances of various sort-on-tapes
algorithms.  Certain subtleties arise from overlaps, rewinds, and
buffering that tend to elude conventional algorithmic analysis. Don's
diagram neatly captures these, and clearly shows that certain
sophisticated algorithms---one was even patented by its 
author---are
in fact slower than traditional methods. Unanticipated rewind
times can cause significant slow-downs, and the chart shows why.

\beginsection 19. [Illustrations (2)] \tll November 6

We spent the first half of class examining the solutions to a homework
assignment (see \S{20} below).
Don says that the solutions were surprisingly good
(see \S{21}).

One of the proofs described in that section
 contains illustrations in four colors.  Don says
that color can be used effectively in talks, but usually not in papers
(for that matter, Leslie Lamport says that proofs should never be
presented in talks, but only in papers). Technical illustrations, even
without four colors, cause no end of trouble: Don says that the amount of
work involved in preparing a paper for publication is proportional to the
cube of the number of illustrations.  But they are indispensable in many
cases.

Don showed us several of the illustrations, charts, and tables from {\sl The
Art of Computer Programming}, Volume~3,
and recounted the difficulties in choosing
clear methods of presenting his ideas.  He also mentioned some technical
and artistic problems that he had with an illustration: At what angle
should the truncated octahedron on page~13 be displayed?

His books contain some numerical tables (``which are sometimes thought to
be unenlightening''); Don says that they can sometimes
present ideas that can't be
demonstrated graphically (such as numbers oscillating about~2 with period~$2\pi$,
page~41).  Diagrams with accompanying text are also used. Don made sure that
the final text was arranged opposite the diagrams to which it refers.

The book contains a running example of how 16 particular numbers are sorted
by dozens of different algorithms. Each algorithm leads to a different
graphical presentation of the sorting activities on those numbers
(pages 77, 82, 84,
97, 98, 106, 110, 113, 115, 124, 140, 143, 147, 151, 161, 165, 166, 172,
175, 205, 251, 253, 254,~359).
\beginsection 20. [Homework: Subscripts and superscripts] A Homework Problem

The Appendix to Gillman's book takes a paper that has horrible notation
and simplifies it greatly. Your assignment is to take Gillman's simplification
and produce something simpler yet. Aim for notation that needs no double
subscripts or subscripted superscripts.
This assignment will be graded! Please take time to do your best.

Here is a statement of Gillman's simplification. This is your starting
point. What is the best way to present Sierpi\'nski's theorem?

\proclaim Lemma. There is a one-to-one correspondence between 
the set of all real numbers
$\alpha$ and the set of all
pairs $(\langle nk\rangle, \langle tk\rangle)$, where
$\langle nk\rangle{k1}$ is an increasing sequence of positive integers
and $\langle tk\rangle{k1}$ is a sequence of real numbers. 

\noindent{\bf Notation.}
The sequences $\langle nk\rangle$ and $\langle tk\rangle$ corresponding
to~$\alpha$ are called $\langle nk^{\alpha}\rangle$ and
$\langle tk^{\alpha}\rangle$. The set of real numbers is called~{\bf R}.

\vfill\eject
\proclaim Theorem. Assume that $\langle A{\alpha}\rangle{\alpha\in{\bf R}}$ is
a family of countably infinite subsets of\/~{\bf R} such that, for
$\alpha\beta$, either $\alpha\in A{\beta}$ or $\beta\in A{\alpha}$.
Then there is a sequence of functions~$fn\!:{\bf R}{\bf R}$ such that,
if $S$ is any uncountable subset of\/~{\bf R}, we have $fn(S)={\bf R}$
for all but finitely many~$fn$.

\noindent
{\bf Proof.} Let the countable set $A{\alpha}$ consist of the real numbers
$$\{\alpha1,\alpha2,\alpha3,\ldots\,\}\,.$$
If $\alpha$ is any real number, define an increasing sequence of positive
integers $\langle lk^{\alpha}\rangle$ by letting $l1^{\alpha}=n1^{\alpha1}$
and then, after $l{k-1}^{\alpha}$ has been defined, letting $lk^{\alpha}$
be the least integer in the sequence $\langle n1^{\alphak},n2^{\alphak},
\ldots\,\rangle$  that is greater than $l{k-1}^{\alpha}$.

Let $fn$ be the function on real numbers defined by the rule
$$fn(\alpha)=\cases{tn^{\alphak}\,,&if $n=lk^{\alpha}$ for some $k1\,$;\cr
\noalign{\smallskip}
\alpha\,,&otherwise.\cr}$$
We will show that the sequence of functions $fn$ satisfies the theorem,
by proving that any set~$S$ for which infinitely many~$n$ have 
$fn(S){\bf R}$ must be countable.

Suppose, therefore, that $\langle nk\rangle$ is an increasing sequence of
integers and that $\langle tk\rangle$ is a sequence of real numbers such that
$$t{nk}\notin f{nk}(S)\,,\qquad\hbox{for all $k1$}\,.$$
Let $tj=0$ if $j$ is not one of the numbers $\{n1,n2,\ldots\}$.
By the lemma, there's a real number~$\beta$ such that $nk=nk^{\beta}$
and $tk=tk^{\beta}$ for all~$k$.

Let $\alpha$ be any real number $\beta$ such that $\alpha\notin A{\beta}$.
We will prove that $\alpha\notin S$;  this will prove the theorem,
because all elements of~$S$ must then lie in the countable set $A{\beta}\cup
\{\beta\}$.

By hypothesis, $\beta\in A{\alpha}$. Hence we have $\beta =\alphak$
for some~$k$. If we set $n=lk^{\alpha}$, we know by the definition
of~$fn$ that
$$fn(\alpha)=tn^{\alphak}=tn^{\beta}=tn\,.$$
But the construction of $lk^{\alpha}$ tells us that $n=nj^{\alphak}
=nj^{\beta}=nj$ for some~$j$. Therefore
$$f{nj}(\alpha)=t{nj}\,.$$
We chose $t{nj}\notin f{nj}(S)$, 
hence $\alpha\notin S$.\quad\blackslug

\vfill\eject
[Here are additional excerpts from TLL's classnotes for October~16,
when the homework problem was handed out:] \
The first thing that we learned in class today was that now would be a good time
to buy Leonard Gillman's book ({\sl Writing Mathematics Well\/}). Not only have
several copies (finally) arrived at the bookstore, but Don has given us a
homework assignment straight out of the Appendix of this book.

The assignment (which is due on Friday, October 30th) is to take the
``simplified version'' of the proof in 
Gillman's  case study 
 and to simplify it still further.  The main simplifying principle is to
minimize subscripts and superscripts.
When we are done, there should be
 no subscripted subscripts and no
subscripted superscripts. As Don said, ``Try to recast the proof so that
the idea of the proof remains the same, but the proof gets shorter.''

The original proof was written by Sierpi\'nski.  Don told us that
Sierpi\'nski was a
great mathematician who wrote several papers cited in {\sl Concrete Mathematics},
from the year 1909 as well as 1959.
But  the notation in Sierpi\'nski's original proof quoted by Gillman
was so complicated that it
confused even him: His proof contained an error that was found by another
mathematician (after publication). 

While the mathematics used in the proof is not trivial, it uses only
functions and sets and should be accessible to us. (This is not to say that it is
immediately obvious.) Anyone who is uncomfortable with what sets are, what
it means for a set to be countable, or what a one-to-one correspondence is,
may need some help with this assignment.  Don recommended visiting the TAs
during office hours as a good first step for those who feel they need
help.  (It might also help to remember that Don says, ``It's not
necessary to understand the proof completely in order to do this
assignment.'')

Don't worry if the hypothesis of the theorem seems pretty wild; it is
pretty wild. It implies the ``Continuum Hypothesis.'' The Continuum
Hypothesis states that there are no infinities between the countably
infinite (the cardinality of the integers) and the continuum 
(the cardinality of the real
numbers). From 1900 to 1960, the truth or falsity of the Continuum Hypothesis
was one of the most famous unsolved problems of mathematics;
Sierpi\'nski published his paper as a step toward solving that problem.
Kurt G\"odel proved in 1938 that the Continuum Hypothesis
is consistent with standard set theory; Paul Cohen of Stanford proved
25~years later that the negation of the Continuum Hypothesis is also consistent.
Thus we know now that the hypothesis  can be neither  proved nor disproved.

\bigskip
[Here are additional excerpts from PMR's classnotes for October~23:] \
The homework assignment is due a week from today, Don said; so do it
as well as possible, and let's not have any excuses!

\vfill\eject\null\vfill
\beginsection 21. [Homework: Solutions] Solutions to the Homework Assignment

Most students pleased the instructor by handling this assignment rather well.
Either you already knew a lot about writing, or you have learned something
this quarter; in any case the exercise seems to have been good practice.

Several answers or excerpts from answers are attached. First is Solution~A,
an unexpurgated draft that was written by your instructor before handing
out the assignment. The main idea here is to ``hold back'' before enumerating
the elements of a set; you can say that $S$ is countable without writing
$S=\{s1,s2,\ldots\,\}$. This solution also simplifies Sierpi\'nski's
proof in minor ways. For example, it's not necessary to have the hypothesis
$\alpha\beta$ to conclude that $\alpha\in A{\beta}$ or $\beta\in A{\alpha}$,
because the existence of a family~$A{\alpha}$ that satisfies
Sierpi\'nski's more complicated hypothesis is equivalent to the existence
of a family that satisfies the simplified one.

The grader objected to the last sentence in the first paragraph of my proof.
He asks, ``Has some `initialization' of~$L{\alpha}$ been omitted?''
He apparently wants $k=1$ to be singled out as a special case, for more
effective exposition. The sentence makes perfectly good sense to me, but
maybe there should be a concession to readers who are unaccustomed to
empty constraints.

Solution B introduces two nice techniques of a different kind. First, the
lemma becomes a sequence of ordered pairs instead of an ordered pair of
sequences. Second, the need for a notational correspondence between $\alpha$
and the corresponding sequence is avoided by just using English words,
saying that one is the {\sl counterpart\/} of the other. In other words,
we can hold back in giving notations for a correspondence, since plain
words are sufficient (even better at times).

Solution B also ``factors'' the proof into two parts, one that describes
a subgoal (the crucial property that the functions~$fn$ will possess)
and one that applies the coup de grace. Much less must be kept in mind
when you read a factored proof,
because the two pieces have a simple interface. Moreover, the reader is told
that the proof is ``essentially a diagonalization technique''; this statement
gives an extremely helpful orientation. It is no wonder that the grader
found Solution~B easier to understand than Solution~A.

Solution C is by another student who found words superior to notation in this case.

Solution D cannot be shown in full because it contains seven illustrations,
some of which are in four colors. But the excerpts that are shown
do capture its expository flavor.

A combination of the ideas from all these solutions would lead to a truly
perspicacious proof of Sierpi\'nski's theorem.

\eject

\line{\bf Solution A\hfill}

\proclaim Lemma. There is a one-to-one correspondence between the set of all
real numbers $\alpha$ and the set of all pairs $(N,T)$, 
where $N$ is a countable set of integers
and $T$ is a sequence of real numbers.

\noindent
{\bf Notation.} The set $N$ corresponding to $\alpha$ is called $N{\alpha}$,
and the sequence~$T$ is called $(\alpha1,\alpha2,\ldots\,)$.
The set of real numbers is called~{\bf R}.

\proclaim Theorem. Assume that there is an uncountable family of 
countable subsets~$A{\alpha}$, one for each real number~$\alpha$, with
the property that either $\alpha\in A{\beta}$ or $\beta\in A{\alpha}$
for all real $\alpha$ and~$\beta$. Then there exists a countable
family~$F$ of functions $f\!:{\bf R}{\bf R}$ such that, if $S$ is any
uncountable subset of\/~{\bf R}, we have $f(S)={\bf R}$ for all but
finitely many $f\in F$.

\noindent
{\bf Proof.} If $\alpha$ is any real number, we can construct a countable
set of integers~$L{\alpha}$ as follows: For $k=1,2,\ldots\,$, 
let $\beta$ be the $k^{\rm th}$ element of~$A{\alpha}$, in some
enumeration of this countable set. Include in~$L{\alpha}$ any element
of~$N{\beta}$ that's not already present in~$L{\alpha}$ because of
the first $k-1$ elements of~$A{\alpha}$.

Now let $F=\{f1,f2,\ldots\}$ be the countable set of functions defined
for all real~$\alpha$ as follows:
$$fn(\alpha)=\cases{\betan\,,&if $n\in L{\alpha}$ and $n$ corresponds
to $\beta\in A{\alpha}\,$;\cr
\alpha\,,&if $n\notin L{\alpha}\,.$\cr}$$
We will show that $F$ satisfies the theorem, by proving that any given set
$S\subseteq R$ is countable whenever $\{\,n\mid fn(S){\bf R}\,\}$ is
infinite.

Let $S$ be a set such that $N=\{\,n\mid fn(S){\bf R}\,\}$ is infinite,
and suppose that
$$tn\notin fn(S)\,,\qquad\hbox{for all $n\in N\,$}.$$
Let $tn=0$ for $n\notin N$. By the lemma, there is a real number~$\beta$
such that $N=N{\beta}$ and $(t1,t2,\ldots\,)=(\beta1,\beta2,\ldots\,)$.

Let $\alpha$ be any real number such that $\alpha\notin A{\beta}$. We will
prove that $\alpha\notin S$; this will prove the theorem, because
all elements of~$S$ then must lie in the countable set~$A{\beta}$.

By hypothesis, $\beta\in A{\alpha}$. Hence there is some $n\in L{\alpha}$
corresponding to~$\beta$, and $fn(\alpha)=\betan$ by definition of~$fn$.
Also $n\in N{\beta}=N$, by the construction of~$L{\alpha}$. But
$\betan=tn\notin fn(S)$, so $\alpha$ cannot be in~$S$.\quad\blackslug

\vfill\eject

\line{\bf Solution B\hfill}

\centerline{\bf Sierpi\'nski's Theorem}

\proclaim Lemma. There is a one-to-one correspondence between the set of all
real numbers~$\alpha$ and the set of all sequences of ordered pairs
$\langle(nl,tl)\rangle{l1}$, where the first components $\langle nl\rangle$
form an increasing sequence of positive integers and the second components~$\langle
tl\rangle$ form a sequence of real numbers.

We shall call the sequence of ordered pairs corresponding to~$\alpha$ the
{\it counterpart\/} of~$\alpha$, and vice versa.

\proclaim Theorem. Suppose that there exists a family of countably infinite
subsets of the reals~{\bf R}, denoted by 
$\langle A{\alpha}\rangle{\alpha\in{\bf R}}$,
with the property that $\alpha\beta$ implies either $\alpha\in A{\beta}$
or $\beta\in A{\alpha}$. Then there is a sequence of functions
$fn$:~${\bf R}{\bf R}$ such that for any uncountable subset~$S$ of\/~{\bf R},
we have $fn(S)={\bf R}$ for all but finitely many~$fn$.

\noindent
{\bf Proof:} Using the existence of $\langle A{\alpha}\rangle{\alpha\in{\bf R}}$,
we first construct a sequence of functions~$fn$ with the property that
for all~$\alpha$, and for all $\beta\in A{\alpha}$, there exists an
ordered pair $(n,t)$ in the counterpart of~$\beta$ such that $fn(\alpha)=t$.
The construction is essentially a diagonalization technique. For each~$\alpha$,
let the countable set~$A{\alpha}$ be enumerated as
$$\{\beta1,\beta2,\beta3,\ldots\,\}\,.$$
Start with $(n1,t1)$ being the first ordered pair in the counterpart
of~$\beta1$. Proceed inductively, and let $(nk,tk)$ be the first
ordered pair in the counterpart of~$\betak$ such that $nk>n{k-1}$.
This selection can be made because the first component of the counterpart
of~$\betak$ is unbounded. Thus, we have constructed a sequence of ordered
pairs $\langle(nk,tk)\rangle{t1}$ with $nk$ increasing and each
$(nk,tk)$ in the counterpart of~$\betak$. Using this sequence, we
then define the function~$fn$ by the rule
$$fn(\alpha)=\cases{tk\,,&if $n=nk$ for some $k$;\cr
\noalign{\smallskip}
\alpha\,,&otherwise.\cr}$$
Indeed, $fn$ is well-defined since $ninj$ for $ij$. Moreover, the
sequence $\langle fn\rangle$ has the desired property that for every~$\alpha$
and every~$\beta$ in~$A{\alpha}$, there is an ordered pair $(n,t)$ in the
counterpart of~$\beta$ such that $fn(\alpha)=t$.

Now we show that any subset $S$ of~{\bf R} for which infinitely many~$n$
have $fn(S){\bf R}$ must be countable, thereby proving the theorem.
If $fn(S){\bf R}$ then there exists a real $t\notin fn(S)$. So if there
are infinitely many~$fn$ such that $fn(S){\bf R}$, then there is a
sequence of ordered pairs $(n,t)$ with $n$ increasing and $t\notin fn(S)$.
Let the counterpart of this sequence of ordered pairs be~$\beta$. Thus, every
ordered pair $(n,t)$ in the counterpart of~$\beta$ has $t\notin fn(S)$.
Now consider all real $\alpha\notin A{\beta}\cup\{\beta\}$. By the
hypothesis, we must have $\beta\in A{\alpha}$. We constructed the 
sequence~$\langle fn\rangle$ in such a way that there is an ordered pair
$(n,t)$ in the counterpart of~$\beta$ with $fn(\alpha)=t$. But by the
choice of~$\beta$, we have $t\notin fn(S)$. Hence, $fn(\alpha)=t\notin fn(S)$
implies $\alpha\notin S$. Therefore  $S$~must be a subset of 
$A{\beta}\cup\{\beta\}$, a~countable set, implying that $S$ is also a
countable set.

\vfill\eject

\line{\bf Solution C\hfill}

\smallskip
%\centerline{\bf A consequence of the continuum hypothesis due to
%Sierpi\'nski}

\dots If the real number $\alpha$ corresponds to the pair $\bigl(\langle nk
\rangle,\langle tk\rangle\bigr)$, then we call $\langle nk\rangle{k1}$
the {\it integer sequence\/} of~$\alpha$ and $\langle tk\rangle{k1}$ the 
{\it real sequence\/} of~$\alpha$.

\smallskip
\dots {\bf Proof.} Note that a given real number $\alpha$ has associated with
it both integer and real sequences, as well as a set
 of reals $A{\alpha}=\{\alpha1,
\alpha2,\alpha3,\ldots\,\}$. We add to this list and construct an infinite set
of integers $L{\alpha}=\{l1,l2,l3,\ldots\,\}$ in which each~$li$ comes
from the integer sequence of~$\alphai$.

\smallskip
\dots
$$fn(\alpha)=\cases{tn\,,&if $n=li\in L{\alpha}$, where 
$\langle tk\rangle$ is the real sequence of $\alphai$;\cr
\noalign{\smallskip}
\alpha&otherwise.\cr}$$
With these functions we will establish the contrapositive of the theorem:
If $fn(S){\bf R}$ for infinitely many integers~$n$, then $S$ is countable.
\dots

\vfill

\line{\bf Solution D\hfill}

As a step toward proving the Continuum Hypothesis, which states that there
are no infinities between the countably infinite and the continuum,
Sierpi\'nski proposed the following theorem.

Suppose we have a function, ${\rm spec}(\alpha)$, that maps every real~$\alpha$
to a countably infinite subset of the reals (Figure~A). Now suppose we
make the additional hypothesis that for any two reals
 $\alpha\overline{\alpha}$, either
$\alpha\in{\rm spec}(\overline{\alpha})$ or $\overline{\alpha}\in{\rm spec}(\alpha)$
(Figure~B). Then we can draw the following conclusion. There exists \dots
$$\vcenter{\halign{$#$\hfil\quad&$#$\hfil\qquad
&$#$\hfil\quad%.
&$#$\hfil\quad%.
&$#$\hfil\qquad%.
&$#$\hfil$\,$%.
&$#$\hfil$\,$%.
&$#$\hfil\qquad%.
&$#$\hfil\qquad\qquad\qquad%.
&$#$\hfil\quad%.
&$#$\hfil\qquad\qquad\qquad%.
&$#$\hfil\qquad%.
&$#$\hfil\qquad%.
&$#$\hfil\cr
&&&&\alpha\cr
&\langle&&&\bullet&&&&&&&&&\rangle\cr
\noalign{\smallskip}
&&&&\Downarrow\cr
\noalign{\smallskip}
{\rm spec}(\alpha)%
&\langle&\bullet&\bullet&&\bullet&\bullet&\bullet&\bullet%
&\bullet&\bullet&\bullet&\bullet&\rangle\cr
}}$$

\centerline{{\bf Figure A.} Each real number $\alpha$ determines ${\rm spec}(\alpha)$,}
\centerline{a countably infinite subset of the reals.}

$$\vcenter{\halign{$#$\hfil\quad&$#$\hfil\qquad
&$#$\hfil\quad%.
&$#$\hfil\quad%.
&$#$\hfil\qquad%.
&$#$\hfil$\,$%.
&$#$\hfil$\,$%.
&$#$\hfil\qquad%.
&$#$\hfil\qquad\qquad\qquad%.
&$#$\hfil\quad%.
&$#$\hfil\qquad\qquad\qquad%.
&$#$\hfil\qquad%.
&$#$\hfil\qquad%.
&$#$\hfil\cr
&&&&\alpha\cr
{\rm spec}(\overline{\alpha})%
&\langle&&&\bullet&&&&&&&&&\rangle\cr
\noalign{\smallskip}
&&&&\Downarrow&&&&&&&\Uparrow\cr
\noalign{\smallskip}
{\rm spec}(\alpha)%
&\langle&\bullet&\bullet&&\bullet&\bullet&\bullet&\bullet%
&\bullet&\bullet&\bullet&\bullet&\rangle\cr
&&&&&&&&&&&\overline{\alpha}\cr
}}$$

\centerline{{\bf Figure B.} By hypothesis, either $\alpha\in{\rm spec}
(\overline{\alpha})$,}
\centerline{or $\overline{\alpha}\in{\rm spec}(\alpha)$. Here $\alpha$ is
not in ${\rm spec}(\overline{\alpha})$,}
\centerline{so $\overline{\alpha}$ must be in ${\rm spec}(\alpha)$.}

\eject
\beginsection 22. [Quotations] \pmr November 9

{\narrower\smallskip\noindent
      Quotation \dots\ a writer expresses himself in quoting 
      words that have been used before because they give his 
      meaning better than he can give it himself, or because 
      they are beautiful or witty, or because he expects them
      to touch a chord of association in his readers, or 
      because he wishes to show that he is learned and well-read.
      Quotation due to the last motive is invariably ill-advised;
      the discerning reader detects it and is contemptuous, the
      undiscerning is perhaps impressed, but even then is at the
      same time repelled, pretentious quotation being the surest
      road to tedium.
\smallskip}

\line{\hfill --- Fowler, {\sl Dictionary of Modern English Usage}.}

\smallskip
{\narrower\smallskip\noindent
	Mais malheur \`a l'auteur qui veut toujours instruire$\,$!
	Le secret d'ennuyer est celui de tout dire.
\smallskip}

\line{\hfill --- Voltaire, {\sl De la Nature de l'Homme.}}
 % 6th discours en vers sur l'homme

\smallskip
{\narrower\smallskip\noindent
	Il ne faut jamais qu'un prince donne dans les d\'etails.
	Il faut qu'il pense, et laisse et fasse agir~:
	il est l'\^ame, et non pas le bras.
\smallskip}

\line{\hfill --- Montesquieu, {\sl Mes Pens\'ees.}}
 % Pens\'ees le Spicil\`ege (1991 edition) #953

\medskip
\noindent
Don's secret delight, he confessed today, is to ``play a library as if
it were a musical instrument.'' Using the resources of a great library
to solve a specific problem---now {\it  that}, to him, is real living.
One of his favourite ways to spend an afternoon is amongst the
labyrinthine archives, pursuing obscure cross-references, tracking down
ancient and neglected volumes, all in the hope of finding the perfect
quotation with which to open or conclude a chapter. Don takes great
pleasure in finding a really good aphorism with which to preface a
piece of writing.  So many people have written so many neat things
down the ages, he said, that it behooves us to take every opportunity
to pass them on. Don has been known to take such a liking to a phrase
that he has written an article to publish along with it.

So how are we to find that wonderfully apposite quotation with which to
preface our term paper? Serendipity, said Don. Live a full and varied
life, read widely, keep your eyes and ears open, live long and prosper.
You will stumble across great quotations. For example, Webster defines
`bit' as ``a~boring tool''---Don was able to use this when
introducing a computer science talk.

Sometimes one needs to go about the search more systematically. For
example, Don's \TeX book consists of 27 chapters, 10~appendices, and a
preface. His format demands two relevant quotations at the end of each
of these. His \MF\ book  posed exactly the same problem. How did he go
about~it?

The first secret, he confided, is Bartlett. There are numerous
dictionaries of quotations [filed under PN 6000 in the reference
section of Green Library], of which Bartlett's {\sl Familiar Quotations\/} is
the most familiar. It was here, under the heading `technique'
in the index, that
Don found a quote from Leonard Bacon deriding Technique as the death
of true Art. 
Now `$\tau\epsilon\chi$',
in Greek, means both `technique' and
`art', so this seemed pretty appropriate for  \TeXbook\ where the (Greek)
name \TeX\ is explained.

When Bartlett fails, we can try the OED. This incomparable dictionary
lists every word along with contexts in which it has been used; very
often it prints a memorable quotation that incorporates the word in
question. Likewise, we can turn to concordances of Shakespeare or
Chaucer to find every single instance in which these authors used any
given word.

Leafing through \TeXbook, Don picked out some of his favourites:
Goethe on mathematicians (and why they are like Frenchmen); Paul
Halmos telling us that the best notation is no notation (write mathematics
as you would speak it!). Tacitus had something to say about the macro
(or rather, about the ancient politician of that name).

A stiffer challenge was provided by a book that listed the \MF\
 code defining each letter of the alphabet (as well as other
symbols) in a certain typeface; Don had to come up with quotes for
 individual letters of the alphabet.
  No problem: James Thurber had proposed the
abolition of~`O'; Ambrose Bierce had scathing things to say about~`M'
in his famous {\sl Devil's Dictionary}; Benjamin Franklin once wrote
to Bodoni concerning the exact form of the letter~`T';
a~technical report about statistical properties of the alphabet
deliberately made no use of the letter~`E'.

Some of the best quotations are taken entirely out of context. The
economist Leontief had something to say about (economic) output; Don
quoted him in his chapter on (computer) output. Galsworthy's comments
on Expressionists found their way into his section on expressions.

In a pinch, said Don, quote yourself. You could even find someone
famous and ask her to say something---anything!---on such-and-such a
subject.  In another desperate case, Don couldn't find anything much
that had been said about fonts. No matter, he quoted the explorer
Pedro Font writing about something else entirely (the discovery of
Palo Alto, as it happens).  If you are Don Knuth, you may even be able
to quote Mary-Claire van Leunen praising your use of quotation!

Computer technology now gives us another quote-locating resource. When
Albert Camus' {\sl The Plague\/} is available online, it will be a
simple matter for this note-taker to find the part in which a writer
agonizes for a week before putting a comma in a particular sentence,
and then for another week before taking it out again; just search for
occurrences of the word `comma' in the text. Don used this technique
to find quotations involving the word `expression' in {\sl Grimms'
Fairy Tales\/} and {\sl Wuthering Heights}, both of which are available
on SAIL.

If any member of the class would like to demonstrate virtuosity at
``playing the library,'' he could try to track down the quotation ``God
is in the details.'' Don rather identifies with God in this, but
hasn't been able to track down the reference. A~number of people have
assured him that it originated with Mies van der Rohe, but
despite reading all the works and contacting the two biographers of
this architect, he has not been able to find it. Someone told him
that Flaubert once wrote ``Le bon Dieu est dans le d\'etail.''
Don hasn't the patience to search exhaustively in Flaubert's voluminous
publications, but he did try French equivalents of Bartlett---finding
the two quotes above (which express the opposite sentiment).
The God-in-details aphorism  remains an orphan to this day. 

Don has found another quote that so well expresses his
philosophy on the subject of error that he is having it carved in
slate by English stonecutters, to occupy pride of place in his garden:

\smallskip\halign{\qquad\qquad #\hfil\cr
     The road to wisdom? Well it's plain\cr
     and simple to express:\cr
\qquad    err\cr
\qquad    and err\cr
\qquad    and err again\cr
\qquad    but less\cr
\qquad    and less\cr
\qquad    and less.\cr
}

\smallskip
Mention was also made of indexes for books. The Sears \& Roebuck
catalogue for 1897 contains the useful advice: ``If you don't find it
in the index, look very carefully through the entire catalogue.''
A~British judge named Lord Campbell  wanted legislation to
compel writers to index their work, but was unable to get round to
indexing his own.

Tangentially, Don mentioned that the designers had given his \TeX book
rather large paragraph indentations---perhaps it's the style of the~80s,
 he said. This meant that he sometimes had to add or subtract
words to ensure that the last line of each paragraph was at least as
long as the indentation on the following one. The page looks rather
strange if this isn't the case.

\beginsection 23. [Scientific American Saga (1)] \tll November 11

Today we heard war stories---stories of the wars between Don Knuth and
the {\sl Scientific American\/} editorial staff.

However, before we got completely on track, Don told us a little about
the book he is writing this quarter: {\sl Concrete Mathematics}.  

He said that this summer he went to see {\sl Snow White and the Seven Dwarfs\/}
and was very impressed. (``Who would have conceived, in 1937, that such a
work of art could be made?'') He said he was inspired; that he wanted to
produce a work of art as inspired as {\sl Snow White}, ``except that I~wanted to
finish it in three months.''

A book in three months: This means that Don has to 
``crank out''  four pages a day,
including Saturdays, Sundays, and holidays. Surprisingly, Don says, ``Here
it is November, and I am still happy.''  He says sometimes he gets up in
the morning and can't wait to get writing; at other times he just finds it
a chore that he has to do; ``but once I get started,
it's easy---starting is the hard part.''

At this point he delivered the punch line to his story on inspiration: We
have one more week to finish the first draft of our term papers.  We have
the good fortune to have two professional editors who have volunteered to
read our papers: Mary-Claire van Leunen and Rosalie Stemer.  

Moving immediately from his statement that we were lucky to have
professionals editing our work to the stories of his wars with a 
professional  editor,
Don showed us a quotation from {\sl The Plague}, by Camus [found by PMR].

{\narrower\smallskip\noindent
    ``What I really want, doctor, is this.  On the day when the manuscript
    reaches the publisher, I~want him to stand up---after he's read it
    through, of course---and say to his staff: `Gentlemen, hats off!'$\,$''
\smallskip}

Of course, the fictional character who made the above statement is
portrayed by Camus as being not only na\"\i ve but a bit mentally unstable.
This doesn't mean that a person couldn't harbor a healthy enmity for an
overzealous copy editor.

We now review the correspondence concerning one paper that Don eventually
had published in {\sl Scientific American\/} (henceforth known as SA):

In the Fall of 1975, Don received a letter from Dennis Flanagan (the editor).
The letter invited him to write a paper, of about
6000 words, on the topic of Algorithms, for SA's 600,000 readers.  It
offered him a \$500 honorarium for such a paper. (This means he got about
eight cents per word or eight cents per 100 readers---depending on how you
like to think of such things.)

After some correspondence concerning the date that the paper was to be
received (Don had been ill and the date needed to be pushed back), we came
to the cover letter for the original manuscript that Don submitted to SA.
He told Mr.~Flanagan that he understood that some editing would take
place, but that he had gone out of his way to try to imitate the
``{\sl Scientific American\/} style.''  Don told them, ``It will be interesting to
see what you do to this, my masterpiece.''

Don soon got a letter back from Mr.~Flanagan acknowledging Don's paper,
telling him that it might have to be ``slightly edited,'' and warning him that
it might take a while to give it the attention it deserves. (Don also got his
\$500 at this point.)  

Finally, 14 months later, Don 
received an edited copy of his paper together with a cover letter that
explained that it had been ``edited for the general reader.''  Don
was told to correct any errors that they might have inadvertently
introduced and exhorted to get back to them within the next two weeks.

To put it mildly, Don was not pleased with the results of this editing.
Every sentence had been rewritten.
He wrote a letter to Martin Gardner---a~letter
written more to vent frustration than in expectation of achieving any
result---in which he stated many of his grievances.  One of his comments
covers the general tone: ``I~was astonished to see how many editorial
changes were made that took perfectly good English and turned it into
something that would be worth no more than~B$^-$ on a high school term paper.''

In addition to showing us his letter to Gardner (and Mr.~Gardner's
sympathetic response) he showed the class the original and the edited
versions.  Among SA's changes: Changing all uses of `we',
transforming some long sentences to several short sentences,
transforming some short sentences into one long sentence, removing commas
(commas that Don found necessary), changing `which's to `that's, removing
technical jargon, changing `most common' to `commonest', and introducing a
few errors. (Don found many of the changes gratuitous, but the editorial
introduction of errors was useful because it meant that Don's exposition
had not been clear enough.)

The next letter we saw was the cover letter for the, now re-edited,
manuscript that Don sent back to Dennis Flanagan.  He mentioned his
extensive re-editing, stated that he appreciated some aspects of the
editing more than some others, and asked to see the galley proofs; he
said viewing the proofs was especially important since there is ``so much
technical material that is typographical in nature.''

Two weeks later
 Don got back the proofs and a letter.  The letter argued successfully with
some of Don's objections to the original editing job 
(they stuck by `that' instead of `which', hurray!); less successfully,
SA~refused to budge on `commonest' (boo). The letter also
 said that sending
proofs to an author was unprecedented.  But the printer was having a
terrible time with the mathematics, so they made an exception.
(Don pointed out that this was
largely dictated by the printing mechanisms they were using.)
But it was a good thing the proofs were sent, because important changes
were made during a 1.5-hour telephone conversation.

By the end of class, about the time that Don showed us his second letter
to Martin Gardner---the one in which he said he shouldn't have been so
frustrated in the first place---Don admitted that some of the disputed
changes really had been appropriate ones.  He said that the original copy
editor had improved his article in some ways,
 but that his further editing had improved
it still further. At the end everybody was happy. (Music~up.)

As a final parenthetical remark, he told us about the way that the (quite
long) captions for  illustrations in SA are typeset: The linebreaks are
determined by hand.  The final line always ends at the right margin
(there's no extra white space). To achieve this, the SA copy editors
must count letters and reword the
captions until they fit.  One of the methods that they use to make things
 fit nicely is to start at the end of the caption and start
removing `the's. (``It is not placed at root of tree because it is too far
from center of alphabet.'') At least, this was the system in~1977.

Among comments about how the SA editorial staff is overworked and how he
shouldn't have been so upset, he did get off a parting shot:
``After spending all this time doing crazy
stuff like caption filling, it's
no wonder the copy editor had no time for polishing
my article.''
\beginsection 24. [Scientific American Saga (2)] Friday the 13$^{\bf th}$,
 part 24: The Classnotes\hfill[notes by PMR]

{\bf The Story So Far.}  {\it Readers will recall that our hero,
`Prof'~Don, is locked in mortal combat with {\sl Scientific American},
a~journal whose global reach is exceeded only by its editorial hubris.
Will Don's definitive {\sl Algorithms\/} article reach the world
unscathed? Or will it suffer the death of a thousand `improvements' at
the hands of a horde of dyslexic copy-editors?}  {\bf Now Read On
.\thinspace.\thinspace.}

On March 25, Don received the page proofs for his article, which was to
appear in the April edition. (``Ever since Martin Gardner's famous
April Fool hoax, I~had wanted to get into an April issue,'' he mused.)
Don picked up the phone and spent the next hour and a half in
damage-limitation negotiations with an editor code-named {\tt TEB}.

\vfill\eject
Some straightforward errors were easily corrected: A~`1' had
metamorphosed into an~`l'  and an~`$\emptyset$' into a `$\varphi$'. Typesetters who
are unfamiliar with mathematics invariably find creative things to do
with this ``empty set'' symbol, Don said.  Many problems show up only at
the page-proof stage. For example, one page began with the solitary
last line of a paragraph and then broke with a new subheading. Since
the paragraph could just as easily introduce the new subsection as
conclude the previous one, Don just moved the subheading back one
paragraph, putting it
on the previous page.  Don also got his floor brackets restored
where square brackets appeared on the page proof.

He didn't get his way on everything, though.
 Brackets were used
interchangeably with parentheses in a mathematical formula, despite
Don's protest that the former have special meanings.

Neither was {\sl Scientific American\/} (`SA', hereinafter) able to get hold
of a photograph of a particular Mesopotamian clay tablet that is
housed in the Louvre. It is a table of reciprocals, and is probably
the earliest example of a large database that was sorted into order
for ease of retrieval.
Don thinks this object definitely
deserves a place in the hearts and minds of CS folk, being perhaps the
first ever significant piece of data processing. Even a modern
computer might need a second or so to do the work involved.

On the whole, Don was pretty happy with his article. It enjoys a
continuing success as an SA reprint; thousands of copies are still sold
to schools (with the page references carefully renumbered).  As far as
Don knows, it's the only one of his articles to have been translated
into Farsi (Persian). He showed us that in this 
language, as in others where the
text runs right to left across the page, mathematical formul{\ae} are not
reversed. The word `hashing' invariably gives translators pause; it
becomes 14~characters in Chinese, and a French translator 
of one of his books once put in a
call to the Acad\'emie Fran{\c c}aise to establish the authorized equivalent.

All the re-editing was painful at the time, admits Don, but in the
long run he has come to agree that this co\"operative 
 effort did much to remove the jargon and
make the paper accessible to a general audience. Martin Gardner, Don
told us, attributes his success as a mathematics writer to the fact
that he is not a mathematician.

Don's paper for the 
IEEE  {\sl Transactions on Information Theory\/}
makes for a sadder tale: They made
such a mess of it that Don decided the game was just not worth the candle,
and he advises everyone to read the Stanford CSD Report instead.  For
example, IEEE says `zero' and `one' instead of~`0' and~`1'. 
Don likes to use `lg' to mean `log to the base~2', but
they changed this without explanation to `log' despite the fact that
to most people this latter means `log to the base~10'; or to number
theorists, when it means the natural log (base~$e$). Not the greatest copy-editors,
Don sighed.

More recently, Don wrote for the October issue of the ACM {\sl Transactions on
Graphics}, and encountered some really shocking copy-editing. They changed
`\dots  data has to \dots' to `\dots data have to \dots'. Now long ago Don was
told that `data' is {\it really\/} plural, but everywhere it is used
both as a singular or a plural, even in the reliably conservative
(`antediluvian!' chimed Mary-Claire) {\sl New York Times}.
Don thought it quite right to use it as a singular when 
referring to data as some kind of collective stuff. Don wrote and
complained that the ACM should certainly know about data. In the end,
Don kept everyone happy by changing the sentence to read
`\dots data must \dots'.

Mary-Claire van Leunen sanctioned the term `Automata Theory',
although one would not normally incorporate a plural adjective into a
compound noun. But no-one has ever said `automaton theory', and no-one
ever will.  

ACM did gracefully admit to and correct some
straightforward mistakes, such as `this number plus that number are
equal to~63'. But where Don wrote $1000000$ they substituted
$1{,}000{,}000$. Don objected that although this might be justified in
text, his use is perfectly OK in a {\it formula}. Well then, they
replied, write $10^6$. Fine, said Don, but what do I do
when the number is 1234567? The IEEE standard here is to insert
spaces, thus: $1\,234\,567$. Don doesn't like this in formul\ae, but agrees
that it may be useful in a high-precision context, such as numerical
tables.

Don recalled a remark by George Forsythe  that every scientist should
try to write for a general audience---not just for other scientists---at
 least once in his life. Don has done this three times now, so feels
that he's done his bit! He gave his first such lecture to a
non-technical audience in Norway and found it surprisingly hard to
understand their `mind set'. The problem is to make the talk
interesting, but convey how it feels to a computer scientist to do
computer science. The public probably imagine that mathematicians sit
and factor polynomials all day, and that CS types design videogames.
How to convey the soul of the subject to them? 
In this lecture, Don presented a
sequence of algorithms for a search task. Since we all have to look up
information in large tables or indexes now and then, he hoped the
audience would have a clear intuition of the problem.  Brute force
searching is clearly too slow; binary search is natural and powerful;
hashing is better still, but very unintuitive to most people. Don was
asked to write up his talk for a Norwegian magazine called
{\sl Forskningsnytt}, 
`Research News' (a sort of {\sl
Scientific Norwegian}). In the course of doing so he learned enough of
the language to write~$v$ and~$h$ instead of $l$ and $r$ to designate
left and right sons in a tree structure. 
Dr.~Ole Amble, a~numerical analyst who was one of Norway's
computer pioneers, helped Don with Norwegian style on this
article, and got interested in search algorithms as a result. He 
asked Don whether there mightn't be a way to combine the advantages of
binary search and hashing? Don at first 
told him ``obviously not,'' but then realized what Amble meant \dots
alas, too late to include in the just-published Volume~3
of {\sl ACP}. But this combination of methods made a nice conclusion
to his SA paper, which was based on this Norwegian prototype.

It was in April of 1977 that Don's travails with SA prompted him
to investigate typesetting for himself; in May of that year he
designed the first draft of \TeX\ and spent his sabbatical 
(and ten more years) perfecting
it, putting Volume~4 of {\sl ACP\/} on the back burner.

We had a few minutes left to look at other changes that
SA made to Don's original manuscript. In one case, there
seemed to be no reason for restructuring a sentence to put Amble's
name first instead of the motivation of his discovery. But
Mary-Claire noted that SA always tries 
 to stress the human contributions in 
science, sometimes at the expense of the ideas. 
Don also mentioned another surprising thing he learned about SA's
editorial policy: They never
display equations. (PMR knows at least one scientist who refuses to
read SA for this very reason---``How can you explain science without
equations?---Pah!'')
\beginsection 25. [Examples of good style] \tll November 16

After a brief (but charming) musical prelude, Don demonstrated to us
that we are not alone in being concerned with the mechanics of writing. He
showed us four small publications that touched on some of the humorous
aspects of written rhetoric.  

We briefly viewed a Russell Baker column entitled ``Block That That  Cursor'';
a~``Peanuts'' comic strip with a punch line concerning comma placement;
a~quotation from the {\sl New York Times\/} (``Plagiarize creatively,
but quotes can be dangerous if you don't
acknowledge the source''); and an article by Richard Feynman in the Caltech
{\sl Alumni\/} magazine.  Feynman discussed his disappointment with his experience
of serving on the Challenger Disaster Panel; he complained that instead
of discussing ideas, the panel spent all their time ``word-smithing''
(deciding how to reword or re-punctuate sentences in the committee's
report).

Feynman's dismay at the amount of time he spent dealing with commas,
wicked-whiches, and typographic presentation is not unique.  Don said,
``Word-smithing is a much greater percentage of what I am supposed to be
doing in my life than I would have ever thought. That's one of the
main reasons I~am teaching this course.''

Don also showed us what he thinks is a wonderful piece of writing: a~spoof
on the Sam Spade genre, full of detectives, blondes, .38's, and the
`sweet smell of greenbacks'. It turned out to be a passage from
``Getting Even'' by Woody Allen. Likewise for the term papers, he said,
try to have a genre in mind (though perhaps not this one) and do a
good job in that genre.

To help prepare us for the guest speakers coming up soon, Don
handed out copies of several of their works, encouraging us to read them
as examples of good practice. First he handed out
an ``Editor's Corner'' article published by Herb Wilf last January:
\smallskip
\display 20pt::
  This issue marks another changing-of-the-guard for the {\sc Monthly}.
  Paul Halmos' act will be a tough one to follow \dots
\smallskip
Wilf's article contains a nice exposition of problems related to
Riemann's famous unproved Hypothesis.

Don also showed us
another draft of a paper by Herb: `$n$~coins in a fountain'. This
title, he said, was just too good to pass up, even though it includes
a formula. But Don would have capitalized the~$n$, because it comes
first. As for the objection about starting a title with a symbol,
why shouldn't we regard~$N$ as simply another English word? (After all,
it appears in most dictionaries as the first entry under~`N'.)
But this approach would make it necessary to capitalize $N$
throughout the article.\footnote*{Herb eventually solved the problem
by calling his paper `The Editor's Corner: $n$ Coins in a Fountain',
in {\sl American Mathematical Monthly\/ \bf95} (1988), 840--843.}

The next guest speaker after Wilf will be Jeff Ullman, who will tell us
how to become rich by writing textbooks. Don recommended that we
look closely at Chapter~11 of Jeff's book {\sl Principles of Database
Systems\/} (second edition), which shows ``excellent simplification of
subtle problems and algorithms.''

Don handed out two examples by the third guest speaker Leslie Lamport.
One, from {\sl Notices of the American Mathematical Society\/ \bf34}
(June 1987), is entitled ``Document Production: Visual or Logical?'' and
Don said ``It's a `flame' but very well written so I wanted you all to
read it. It's a nice polemic that takes the `WYSIWYG versus Markup'
controversy and reformulates the problem along more fruitful lines.''
The other Lamport article is entitled ``A simple approach to specifying
concurrent systems''; it will soon be published in {\sl Communications
of the ACM}.

Don says the latter paper is the best technical
report he has seen in the last year or so.  The paper is unusual because
of its question-and-answer format. While dialogs have been used 
effectively by experts
in other fields (such as Socrates, Galileo, George Dantzig,
and Alfred R\'enyi), this is the first time, as far as Don knows, that such a format
has been used in computer science.

\medskip
    Before moving on to the next handout, Don told us about writing his book
{\sl Surreal Numbers}.  Like Leslie Lamport's paper, Don's book is presented
    as a dialog.
 Don's dialog presents some ideas that John Conway told him
 at lunch one day (Don wrote the ideas down on a napkin and then lost
    the napkin).  The most extraordinary aspect of this book is that Don wrote
    it in six days (``And then I rested'').  That week was very special for Don.
    (``It was the most exciting week in my life.  I~don't think I~can ever
    recapture it.'')
    
    When Don wrote the book he was in Norway.  He was in the middle of writing
    one of the volumes of {\sl The Art of Computer Programming\/} (isn't he
    always?), and he did not expect Jill (his wife) to be sympathetic when he
    told her that he wanted to write yet another book---even if he did think he
    could write it in a week.  Perhaps Jill knows more about Don than Don
    knows about Jill, because she  not only didn't complain but she got
    quite into the spirit of the thing.
    
    Just what was the spirit of the thing?  ``Intellectual whimsey'' probably
    isn't far off.  Don rented a hotel room (``near where Ibsen wrote'') and
    spent his week writing, taking long walks (``to get my head clear''),
    eavesdropping on his fellow hotel guests at breakfast
(``so I~could hear what 
    dialog really sounds like''), and pretending that Jill's visits were 
    clandestine (``we had always read about people having affairs in 
    hotels \dots'').
    
    Don said he wrote ``with a muse on my shoulder.''  Every night's sleep was
    filled with ideas and solutions; before dozing off he would have to get 
up and write down the first
    letter of every word of the ideas he had (and he would
    spend the morning decoding these cryptic scribbles).  He told us that he
    was more perceptive during this week---his description of the King's
    Garden during an evening  walk was worthy of Timothy Leary.  
    
    All this prolific word production must have left him in verbal debt: 
    When he finished the book he tried to write a letter to Phyllis telling 
    her how to type the book.  He couldn't.  Except he must have 
eventually---the 
book is still in print and sells several hundred copies a year (in seven languages).
    
\goodbreak\medskip
Still another handout was part of a chapter written by Nils
Nilsson and Mike Genesereth for their new book {\sl Logical Foundations of
Artificial Intelligence}.  Chapter~6, entitled ``Nonmonotonic reasoning,''
presents a new area of research at the level of a graduate student.  Don
says that the chapter has an excellent blend of formal and
informal discussion, with well-chosen examples; this subject
had never been ``popularized'' before, so the task of writing a good
exposition was especially challenging. Don also praised the authors'
typographic conventions (for example,
logic is presented using a ``typewriter'' font).

\smallskip
Don said that we already have Mary-Claire's book, so he didn't have to
introduce her to us. But he ran across some electronic mail she had
written recently, and thought it was a particularly elegant essay, so
he passed it along (see \S{26} below). Computer scientists and mathematicians
are way behind real writers when it comes to exquisite style.

Finally, just in case we still craved more good examples to read,
he handed out some
excerpts from a paper written by Garey, Graham,
Johnson, and Knuth.  Don says that he included it because it has two
proofs of difficult theorems: proofs that are not, and probably could not be,
trivial.  

Don tried to interest his readers in the first proof (and algorithm) by
presenting an example as a mathematical puzzle.  He says that by solving the
puzzle the reader can see that the problem is not simplistic---but that an
algorithm might be possible. 
(``This builds exactly the right mental structures in the reader's
mind for this particular problem, I~think.
The algorithm itself is the worst algorithm
I~have ever had to present---but there is probably 
no simpler one.'')  While
flashing us part of the algorithm---complete with more cases than could fit
on the monitor---Don said, ``The ability to handle lots of cases is Computer
Science's strength and weakness.  We are good at dealing with such
complexity, but we sometimes don't try for unity when there is unity.''

The second proof involves the reduction of one problem to another.  The
reduction requires a very complicated system---a~system that Don found was
well served by an extended biological metaphor and some involved
terminology.  As his metaphor, he chose the jellyfish (``an unrooted,
free-floating tree''); he named pieces of the data structure stems, polyps,
tentacles, heads, and nematocysts (the biological term for stingers).  

Mary-Claire asked, ``If that structure turns out to be generally
useful, are you going to be sad that you called it a nematocyst rather
than a stinger?''  Don said No, but he has been sorry about names he has
chosen in the past. (He wishes he had called LR$(k)$ grammars L$(k)$
grammars.)  When he was writing {\sl The Art of Computer Programming}, Volume
three, he used the word ``Daemon''  to refer to what are
now called ``Oracles,'' but the Oracle replaced the Daemon before
it was too late.

Another last minute terminology substitution happened when Aho, Hopcroft,
and Ullman substituted ``NP-complete'' for ``Polynomially-complete'' in their
text on Algorithms---even though they had already gotten  galley proofs using
the original name.  The name was changed at that late date as the result
of a poll conducted throughout the Theoretical Computer Science community
(suggested names were NP-Hard, Herculean Problem, and Augean Problem).

\vfill\eject
\beginsection 26. [Mary-Claire van Leunen on `hopefully']

\vfill\eject
\null\vfill\eject
\beginsection 27. [Herb Wilf on Mathematical Writing] \tll October 28

Class opened as Don introduced today's guest speaker: Professor Herbert
Wilf.  Professor Wilf is on the faculty at the University of Pennsylvania
but is spending his sabbatical year at Stanford.  

As Wilf took the dais he pronounced this ``a marvelous course.'' (``Taken
earlier in my career it would have saved me and the world a lot of
grief---mostly me.'')  The course topic is one of daily concern for him;
apart from writing his own papers he edits two very different journals:
the {\sl American Mathematical Monthly\/} (``The {\sc Monthly}'') 
and the {\sl Journal of Algorithms}.

The {\sl Journal of Algorithms\/} was founded in 1980 by Wilf and Knuth and is a
research journal.  Results are reported there if they are new, if they are
important, and if they are significant contributions to the field. If these
conditions are met, a little leeway can be given in the area of beautiful
presentation.  But the {\sc Monthly} is an expository
journal.  It is a home for excellent mathematical exposition.  (It also
seems to be a popular place to send ``proofs'' of Fermat's Last Theorem.)  

Though he told us that he feels ``older without feeling wiser'' and is
uncomfortable setting down rules for a human interaction that ``involves
part brain and part hormone system'' he gave us several pointers.

\smallskip
\display 30pt::
{\cm Get the attention of your readers immediately.}  Snappy titles, arresting
    first sentences, and lucid initial paragraphs are all methods of doing
    this.  
    
\smallskip
\display 30pt::
    As examples, he showed us a paper by Andrew M. Gleason with the title
   ``Trisecting the Angle, the Heptagon, and the Triskaidecagon''; a~paper by
    Hugh Thurston that began ``Can a graph be continuous and discontinuous?'';
    and the first paragraph of an 
autobiographical piece by Olga Taussky-Todd
    that started with some insight into the author's fascination with matrices.
Gleason's paper was attention-getting mostly because Gleason is famous---``that
helps.''
    
\smallskip
\display 30pt::
{\cm Get everything up front.}  Tell your readers in plain English what you are
    going to write about and let them decide for themselves whether or not
    they are interested.  (``You can quintuple your readership if you will let
    them in on what it is that you are doing.'')
    
\smallskip
\display 30pt::
{\cm Remember that people scan papers when they read them.}  Potential readers
    will skim looking for statements of theorems; if all of your text is
    discursive they will have nothing to latch onto.  Summarize your results
    using bold face (``or neon'') so that the page flippers can make an informed
    decision.  Similarly, drop notational abbreviations and convoluted
    references in the statements of theorems.
    
\smallskip
\display 30pt::
{\cm A little motivation is good, but readers don't like too much.}
    Presenting examples that do not yield desired results can be quite useful,
    but the technique loses its charm after a  small number of such examples.
    (Far from overdoing this technique, many writers will introduce 
    mysteriously convenient starting points for their theorems.  ``Whenever 
    I see `Consider the following \dots' I~know the author really means
    to say `Here comes something from the left field bleachers.'$\,$'')

\smallskip
He gave us the name of three books (not written by anyone in the room)
that he considers superb books of mathematics:
    
\smallskip
\display 30pt::
{\sl Problems and Theorems in Analysis}, by P\'olya and Szeg\H o.  It has a
``Problems'' section and an ``Answers'' section.  The problems are
    self-contained, digestible pieces of more complex problems.  The 
    answers are on the spare side and have been the cause of much 
    head-scratching over the years.  By solving several of these 
    self-contained problems, a reader can arrive at an understanding of
    major results in the field.
    
\smallskip
\display 30pt::
{\sl An Introduction to the Theory of Numbers}, by Hardy and Wright.
    This book is ``short on
    motivation.''  Theorems are stated and proved concisely and precisely.  In
    the preface the authors claim that ``the subject matter is so attractive
    that only extravagant incompetence could make it dull.''
    
\smallskip
\display 30pt::
{\sl Principles of Mathematical Analysis}, by Rudin.  This book is rigorous.  It teaches
    the reader what is and is not a proof.  A reader who survives this book 
    feels strong.

\smallskip
Wilf commented that all three of these books are quite dry, but Knuth
objected (along the same lines as those used by Hardy and Wright in their
preface) and Wilf amended his statement: Each of these books is very lean.

Discussing the change of his own writing style over time, he told us that
when he was younger he didn't have much self esteem and stuck to
established forms. Now that he feels better about himself he has developed
his own, much chattier,  style. (Speaking of chattiness, he is also a fan
of the use of the first-person in technical writing.)  He says he aims to
be chatty leading up to a proof, prove it in the ``lean and mean'' style
that Rudin would use, and then be chatty again after he finishes the
proof.

The last things that Wilf discussed were two handouts (\S{28} and \S{29} below):
``Enumeration of orbits of mappings under action
of~$Cn$, the cyclic group,'' and ``Counting necklaces.'' Each handout discusses
the same mathematical problem,
solved the same way.
 ``Enumeration of \dots'' takes a half page;
``Counting Necklaces'' takes four pages.

Some audience members will appreciate the half page of exposition that is
condensed to the word ``evidently'' in the shorter paper; some will
merely be annoyed by it.  As the {\sl Monthly\/} editor he gets letters from
people who complain about the informal style
creeping into recent publications.  ``Mathematics is a serious business,
not a comic pursuit,'' said one such letter.

Finally, Wilf doesn't mean to say that either of the two approaches is
superior (``They are the two sides of the coin''); he means for us to
examine each and decide what techniques we want from each.

\vfill\eject
\beginsection 28. [Wilf's first extreme] From Acta Hypermathica

\vfill\eject
\beginsection 29. [Wilf's other extreme] From MathWorld

\null\vfill\eject
\null\vfill\eject
\null\vfill\eject
\null\vfill\eject
\beginsection 30. [Jeff Ullman on Getting Rich]	\pmr November 18

{\narrower\smallskip\noindent
   ``No man but a blockhead ever wrote except for money.''
\smallskip}
\line{\hfill --- Samuel Johnson, quoted in Boswell's {\sl Life of Samuel Johnson\/}}
\line{\hfill         April 5th, 1776}

\medskip
Today we got an entirely different perspective on the whole ball of
wax. Don began his fortnight's sabbatical by turning the stage over to
one of Computer Science's most prolific authors: Professor Jeff
Ullman. A~large crowd had gathered to hear Jeff's advice on ``How to
get rich by writing books''---an illustration of one of the principles
of cover design, he said: Attract people with something that isn't in
the book at all.

Jeff started by talking a bit about the pragmatics of publishing---how
the money flows. He kicked off with a back-of-an-envelope calculation.
A~book is a megabyte of text. Jeff can write perhaps two or three
kilobytes of first draft per hour---say one kilobyte per hour of
finished text. We can all train ourselves up to much the same
performance, he asserted. So it takes around a thousand hours of
labour to write a book. Now then, a typical CS text might sell for
\$40. A~good book on a specialized topic, or a mediocre book on a
general topic, might well sell 1500 copies in the US and 500 copies
abroad. (These figures put the 200,000 copies of Don's {\sl ACP\/} sold
in the USSR into some perspective.) A~15\% royalty is standard on
domestic sales, a rather lower rate for foreign sales.  All in all,
our talented specialist or so-so generalist can expect to net maybe
\$8000 over his book's lifetime of perhaps five years. Of course, fame
as well as fortune is to be gained through publication, but Jeff
dismissed such non-financial motivations as being beyond the scope of
his talk.

``I {\it told\/} you to be a lawyer. Or a doctor,'' someone's mother was
heard to whisper. But Jeff forestalled a mass exodus to the GSB by
going on to tell us how to make book-writing a going concern. Firstly,
he said, it's quite feasible to double the royalty rate. CS authors
have some leverage with publishers in that their books sell quite 
well---a~publisher's
costs are very sublinear in the number of copies sold,
so he can afford to pay a lot more for a book that will sell 5000,
instead of 2000, copies. What's more, a computer scientist often keeps
his publisher's costs down by preparing his own camera-ready copies.
Jeff is happy to tell you more about how to drive a hard bargain with
your publisher---go and talk to him about it! He sees an upward trend
setting in, with royalties exceeding 30\%.

Secondly, you need to aim for ten thousand domestic sales; say two
thousand a year for five years. That's 5--10\% of the entire market in a
topic like compilers or operating systems. There's nothing
off-the-wall about this, provided you find the right niche: Let yours
be the hardest book on the subject, or the easiest. Or the best. This
wasn't so hard to do in the early days of~CS, when there was a big
demand for textbooks but only a few authors; it's certainly going to
get harder as the field matures. If you're going for the big bucks,
advised Jeff, choose a young and booming field---biogenetics perhaps.

Increase your royalties and sales, and your efforts can net you as
much as a medium-grade hooker's: say \$100 per hour. Top-notch computer
scientists should aspire to no less.

\vfill\eject

{\bf A miscellany of tips}

\display 20pt:$\bullet$:
Find a coauthor or two. Coauthors won't save you any time, but they
do help filter out your idiosyncrasies. Jeff said that when he writes
alone ``my own craziness takes over'' and the book turns out a dud. He
was just ``too weird'' in {\sl Principles of Programming Systems\/}---although
 not too weird for the Japanese, who continue to buy it. His
Database book went well, though probably because Chris Date's book
provided a framework and the necessary ``dose of reality.''  Filtering
out oddball stuff has a big effect on quality. And since a textbook that
is only marginally better than the competition will nevertheless grab
the lion's share of sales, any small improvement is well worth having.

\display 20pt:$\bullet$:
Jeff never saw a book with too many examples. Use lots. Even a very
simple example will get three-quarters of an idea across. A~page or
two later you can refine it with a complex example that illustrates
all the ``grubbies.'' But finding good examples---examples that 
illustrate  all and only the points you are concerned with---is not
easy; Jeff has no recipe. You must be prepared to spend a lot of time
on~it.

\display 20pt:$\bullet$:
Jeff endorsed Don's exhortation: ``Put yourself in the reader's place!''
If Mary-Claire concurs, we may even be convinced.

\display 20pt:$\bullet$:
Spend the day reading about a topic, and write it up in the evening.
That way, you'll get the expository order right. You have an advantage
over the experts because you can still remember what was hard to learn.

\display 20pt:$\bullet$:
Jeff often sees a definition in Chapter 2 and its use in Chapter~5.
This just isn't the way readers work; it's essential to keep
definitions and uses close together. Don't be ashamed to repeat
yourself if that's what it takes.

\display 20pt:$\bullet$:
Those who can, do; those who can't, teach; those who can't teach,
show off.  Remember that the object of exposition is education, not
showmanship.

\display 20pt:$\bullet$:
There is a tradeoff in using powerful mechanics to justify your
methods; they may be too opaque. Jeff had to decide whether to spend
20~pages teaching asymptotic analysis in order to spend 5~pages
applying its theorems, or whether just to say ``It can be shown that
\dots'' and refer his readers to another text. In the end he got around
the dilemma by doing only the most basic calculations and proving
nothing deep. In general, keep the level of your exposition down so
that you can rely on your readers understanding it.

\smallskip
{\bf A couple of tactical remarks:}

State the {\it types\/} of your variables. Talk about `\dots  the set~$S$
\dots', not about `\dots ~$S$ \dots'. 

Jeff's English professor, now a leading poet, told him never to use the
non-referential `this'. Recognizing the dearth of poetry in~CS, Jeff
now forbids his students to use it either. 90\% of the time it doesn't
matter; the other 10\% leaves your readers bewildered. One book
presents four ideas in a row and then says ``This leads us to
consider \dots''.  {\it What\/} leads us to consider?

\vfill\eject

{\bf Coping with the competition}

Like it or not, book-writing is an increasingly competitive sport.
But just because every other introductory Pascal text starts with
`{\bf write}' statements doesn't mean that yours has to start with `{\bf while}',
just to be different.  Don't slavishly imitate another's style, but
don't avoid it either. Know the market, know thyself, and work out a
compromise of your own. Don't hesitate to follow the crowd when they
are all going in the right direction.

This last remark brought Jeff (``I am not a lawyer'') Ullman round to
the tricky subject of plagiarism. According to Prentice-Hall's
{\sl Guide to Authors},
imitation ceases to be the sincerest form of flattery and becomes
something much more culpable if a reasonable person could not believe
that you didn't have the other chap's book open in front of you as you
wrote yours. That said, remember that you can't copyright ideas as such, but
only ways of expressing them. Jeff shamelessly admits that his {\sl
Compilers\/} book borrowed another's table of contents  and the
general front-to-back expository scheme.

Jeff showed us a suspicious case in which an author had written ``Knuth
has shown \dots'' and then went on to quote more-or-less verbatim from
{\sl ACP}. The coincidence of notation is hardly conclusive, he said,
but the identical use of italics is pretty damning.

Don here pointed out that his disciple had actually corrected a typo,
 for one sentence was in fact the exact logical negation
of the other. But this book contained much worse examples of plagiarism:
A~dozen or so
 successive equations lifted straight from elsewhere. In these
notes, names have been suppressed to protect the guilty.

Someone asked about second and subsequent editions. Jeff said that these
will still consume a kilohour or so, although they'll go faster if you
can use your earlier examples. But the financial advantages are very
real: People stop buying a book when it has been out for five years,
so publish a new edition and start the clock ticking again!

One person asked about writing survey papers---surely they will
contain a lot of verbatim quotes? There's no problem since the writer
is not presenting the work as his own, Jeff said.  Besides,
accusations of plagiarism hinge on financial loss, and no one writes
technical papers to make money.  But be explicit in your quotation if
you feel more comfortable doing so.

Why don't expositions of CS make more use of analogy, asked someone,
drawing an analogy with physics texts (which are planted thick with
analogy, metaphor, and simile). Jeff thought it partly due to the
nature of the subject, but encouraged us to use analogy where we are
sure that the reader will get the point.

Asked about progress on {\sl Parallel Computation}, Jeff confessed
that it may never be finished: ``That's another point about
coauthors \dots''.  Jeff left us, and Don, to reflect on his maxim:
$$\hbox{``Never spend more than a year on anything.''}$$
\beginsection 31. [Leslie Lamport on Writing Papers] \tll November 20

Today's special guest lecturer was Leslie Lamport of DECSRC.
Leslie, sporting a {\it Mama's Barbeque\/} T-shirt (``WALK
IN --- PIG OUT''), took the stage and gave us a very active lecture. (He
clearly believes in one of his own maxims: ``You've got to be excited about
what you are doing.'')

The first thing  Leslie told us was that he would restrict his
advice to the writing of papers (not books).  ``I~have one
thing to say about writing a paper for publication: Don't. The market
is flooded.  Why add to the detritus?''  After the appropriate dramatic
pause, he continued with, ``But seriously folks, somebody has to write
papers.''

While we are asking ourselves if our own papers are worth writing, Leslie
asks that we keep in mind two bad reasons for writing a paper:

The first bad reason is ``to have a long publications list.''  Leslie says
he would like to think that the people who are supposed to be impressed by
a long publications list would be more impressed with quality than
quantity. Admitting that this might not always be the case, he appealed to
our own sense of integrity to police us where others' standards do not. 

The second bad reason is ``to have a paper published in a specific
conference.''  Leslie has known people whose need to insert papers in
specific proceedings is greater than their need to disseminate accurate
information.  This approach ``sometimes leads to pretty sloppy papers.''
He told us that he knows of one case where the authors of a conference
paper promised to send a correction, once they figured it out, to each
conference participant.

Leslie recognizes {\it one\/} good reason to publish a paper: ``You have done
something that you are excited about.''

Just how excited can you be and yet not publish a paper? Leslie was once
told: ``Judge an artist not by the quality of what is framed and hanging on
the walls, but by the quality of what's in the wastebasket.'' Similarly,
Leslie thinks that we should be judged on the ``best thing that we have
done that we decided not to publish.''

Moving on to how we learn to write well, Leslie told us that learning to
write is more like learning to play the piano than like learning to type.
While both typing and piano-playing involve motor skills, a good pianist
must spend much time studying music in its entirety; he must spend more
time away from the piano than in front of it.  Correspondingly, we should
learn to write by reading.  Leslie payed homage to Halmos and Knuth, but
said that they can not match Fowles and Eliot: We should read great
literature in order to learn how to write good mathematical literature.

We must know what we want to present before we can present it well.
As Leslie said, ``Bad writing comes from bad thinking, and bad thinking
never produces good writing.'' We must keep in mind what we are
writing---and to whom.

The question of audience is closely related to where a paper, once
written, should be published.  Appropriate places may be a Tech
Report, a letter, a Journal, or the bottom drawer of your desk. (Don't
really throw anything out: it is good to have the record, even if you don't
publish your work.)  How do we choose?

Journal articles should be polished and timeless.  Conference papers
can be a little rougher.  Conference papers are appropriate for work
that is ``not yet ready for the archives.''  Technical reports (usually
distributed by an institution) are good for work that is not even ready
for the general world but still should be written up.

Leslie asks us to remember that ``in each case, you still have readers.
That tech report may some day turn into a Journal article. You've got
to be excited about your writing.''

As for the central theme of a paper, Leslie told us that he enjoys the
Elizabethan use of the word ``conceit'' to denote a fanciful or cute idea
around which a paper can be built.  While such an idea can be a good
catalyst as we begin to write, we should be willing to abandon it.
After all, we use such metaphors or themes in order to present
ideas---we should not allow them to intrude.  The line between what
 can be called a conceit and the merely cute is a fine one.  Beware
of jokes.  Just how funny will a joke be ten years after it was
included in a Journal paper?

While jokes should be left out, examples are welcome additions to most
papers.  Leslie said, ``It is better to have one solid example than to have
a dry, abstract, academic paper.''  He also said that it is never a mistake
to have too simple an example (``at~least not for a lecture'').  Demonstrating
that ``examples keep you honest,'' Leslie told us about a major revision of
one of his published theories upon discovering that his original draft
of the theory was not
powerful enough to deal with the example that he wanted to use in his
paper.

Expressing concern that people often ``fix 
the sentence and not the idea,'' Leslie
told us that we can be too concerned with details.  For example, he tells
us not to think about formatting when we are writing.  (``Don't think about
format.  Do think about structure.'')  He suggests that whenever we have
some detail, such as complex notation, we shouldn't write it out: We
should use a macro.

Leslie discussed trends in notation, showing us a translation of
Newton's {\sl Principia Mathematica}.  Newton stated his mathematical
theorems in
non-mathematical language that was  very difficult to read. 
Instead of saying that something is inversely proportional to the
square of the distance, we can get the point across better by saying that it
is $c/d^2$.

Thus algebra has provided us with a tool for presenting the structure
of a formula. But can't we improve present practice by making the
structure of an entire discourse more clear?
Leslie gave us a handout demonstrating two forms of a proof: a paragraph
form and a form that looks like the tabular proofs that high-school
students  produce in Plane
Geometry homework. (See \S{33} below.)
Pointing out that the tabular proof is much easier to read,
Leslie cautioned us that he was not talking about formatting, but the
structure that the tabular form enforces.  He says that writing proofs in
such tabular ``statement-reason'' forms will help us clarify proofs that are
to be presented in paragraph style. (The flip side of the
handout also contains an example
that Leslie did not have time to explain.  The example shows some
``bloated prose'' that Leslie trimmed down by half.)

In discussing writing itself, Leslie said, ``You should be excited about
what you are writing and that excitement should show.''  Saying that this
principle can especially be applied to first sentences (``You want
something that leaps out at you''), he read us several first sentences from
various compositions.  
The first sentence can be expected to be nontechnical and to represent
an author's best effort.
He was pleased with some of the first sentences
from his own work and less pleased with others, but he was ecstatic about
some of the first sentences he read us by T.~S. Eliot or Allen Ginsberg.
Thus it might be a good idea to ask ourselves: ``What would T.~S.
have written, if he were writing this paper?''

What characterizes a good first sentence?  Leslie says to ``avoid  passive
wimpiness,'' but to be simple and direct.  ``Get right down to
business.''  Of course, once you have hit your readers in the gut with your
first sentence, you can't let them down with your second. Continuing in
this vein, by induction, ``When you come to sentence number 2079, you've
got to keep socking it to them.''  (He illustrated this by reading an
arresting sentence from the middle of {\sl The Four Quartets\/} by T.~S. Eliot,
choosing the sentence at random.)

Leslie finished his lecture by saying, ``I~am not T.~S. Eliot.  I~need to
pay more attention to my writing.  As do we all.''

\beginsection 32. [Lamport's handout on unnecessary prose] How I changed
my coauthor's draft

\vfill\eject
\beginsection 33. [Lamport's handout on styles of proof] Toward structured proof\kern1pts

\vfill\eject
\beginsection 34. [Nils Nilsson on Art and Writing] \pmr November 23

Nils Nilsson, latest in our line-up of megastar guest speakers, spoke
on the subject of ``Art and Writing.'' He began by showing us two
photographs: Edward Weston's print of a snail-shell (strangely
reminiscent of a human form), and Ansel Adams's ``Aspens in New Mexico.''
Having thus set the artistic mood, Nils went on to talk about what
this has to do with writing. Novels and plays are recognised as art;
mathematical writing should also qualify, he said. Writing can be both
art and communication; indeed, {\it real\/} communication happens only
when writing is charged with artistic passion. 

For Nils, a key word is
{\it Composition}. Nils once took a course in photography from a
teacher who declared that:
$$\hbox{{\sl Composition $=$ Organisation $+$ Simplification}.}$$
This formulation made a lasting impression on Nils. It applies equally
to writing as to  photography. A~quote from Edward Weston:
``Composition is the strongest way of seeing.'' A~typical artistic
phrase, said Nils, but what does it mean?  Some might say that Weston
anticipated the findings of recent research in computer vision: The
viewer must participate, construct models, form hypotheses. There are
no spectator sports! Likewise, a photographer sees best when a scene
is well-composed.

{\obeylines
\qquad\qquad\qquad ``Life is very nice, but it lacks form.
\vskip -5pt
\qquad\qquad\qquad  \phantom{''}It is the aim of art to give it some.''
}
\line{\hfill--- Jean Anouilh.\qquad}

But like all art, said Nils, writing should be fun. Just as the
painter takes pleasure in the smell of his paints, so should the
writer feel good when surrounded by the tools of his art: paper, ink,
typewriter, wordprocessor, whatever. He must feel a thrill, as Don
does  when pinpointing a reference. Another key word, then, is {\it Joy}.

But if writing is to be art, we must first master the craft. Only when
our grasp of the minuti{\ae} is perfect can we transcend technique and
aspire to genius. Nils gave us a ``broad brush'' overview of some
important points, along with some autobiographical tales.

\smallskip
\display 20pt:{\bf 1.}:
{\bf Start early.} \quad Impressionable minds are best. Some people find
that writing becomes a real compulsion; if this happens to you, then
let the urge take over!

\smallskip
   Way back in 1954 Nils took a Stanford course on ``Scientific Writing.'' Writing
   an essay or two a week, he learned to become clear and organised; and
   he got an A$^-$ for his paper on ``Ionic Oscillations.'' Nils was pretty
   pleased, and thus began his career as a writer.  In the Air Force,
   he   discovered a growing urge to write a book about radar;  he
   realises now that this was mainly a compulsion to get the material
   organised. By 1960 he had an outline of the book, but it
   never saw the light of day; after leaving the Air Force he joined SRI
   and got deeply involved with something else entirely (Neural Nets, as
   it happens).

\smallskip
\display 20pt:{\bf 2.}:
%{\bf Write, rewrite, rewrite, rewrite \dots\thinspace .} \quad
{\bf Write, rewrite, rewrite, rewrite .\thinspace.\thinspace.  \thinspace .} \quad
This dictum  really is true,
said Nils.  It is the extremely rare artist who does not need to
labour over and over on his work. Mozart was said to be an exception;
his first draft was his final version. Beethoven, on the other hand,
rewrote his work over and over, and even then was never satisfied. As
someone once remarked: ``A work of art is never completed, only
abandoned.'' 

A~member of the class quoted Robert Heinlein as saying
that a writer must resist the urge to rewrite. (But then, Heinlein writes
great thick books and pretty poor ones at that.  Likewise, Barbara
Cartland is said to wander about the house dictating her novels into
a tape-recorder, whence they are transcribed and published. Of the
literary qualities of her work, the less said the better.)

\smallskip
``Easy writing makes damned hard reading.'' Nils couldn't remember the
source of this quote.\footnote*{``Easy reading is damned hard writing.''
--- Nathaniel Hawthorne. (``Just lucky to
find it.'' --- DEK.)}
Hemingway put it rather more colourfully, which
we blush to repeat here. Think of your early drafts as being like an
artist's sketches, urged Nils: Be prepared to throw away nearly all of
them.  Neither are you done when the book is finished and on the
shelves. Maurice Karnaugh (inventor of Karnaugh Maps) wrote to Nils
after {\sl Principles of Artificial Intelligence\/} was published and
pointed out that the $A^{\ast}$~algorithm as Nils had defined it would fail on
a certain graph. This led to a correction in a subsequent edition.

Never let anything you write be published without having had others
critique it. A~university is a good environment in which to get
feedback on your work, though you may need to give some thought to the
timing of your requests for comments (unless you have infinite
resources of willing readers). Nils told us about the time that he
thought he had a neat result in non-monotonic reasoning and
circumscription. He wrote it up and sent it to John McCarthy, who
passed it on to Vladimir Lifschitz, who discovered that Nils's
derivation ``appeared to contain an oversight \dots''.

Nils always tries to teach a course on a topic at the time that he is
writing it up---it's ridiculous to inflict your ramblings on the world
unless you are prepared to do this, he said.

   Nils decided that since he had the whole book online, he would take a
   crack at publishing it himself. He and his wife Karen set up the Tioga
   Publishing Company. One big advantage about this cottage-industry approach 
   is the ease with which the author can make changes in subsequent editions. 
   Karen went on to become a full-time publisher; Tioga's theme has now
changed from AI to nature and the environment. 
So Nils considers himself pretty well
   ``vertically integrated'' in the world of books.

\smallskip
\display 20pt:{\bf 3.}:
{\bf  Read.} \quad Read a great deal; it'll sharpen your style and get your
critical faculties working.

\smallskip
\display 20pt:{\bf 4.}:
{\bf Model the Reader.}\quad  D\'ej\`a vu. This should be obvious, said
Nils, but there's really a lot to it.  Ask yourself what the reader's
primitives are, and write with them in mind. In fact, the whole issue
is so complex and important that Nils likes to operationalise it with
AI-type ``d{\ae}mons.'' Any number of these have to be running in the
background as you write, catching errors and providing constructive
criticism. You have to be asking all the time; ``How is the reader
going to misunderstand me here?'' You must automatically insert forests of
guidelines to keep him on track. You develop these d{\ae}mons by 
practice---it's a kind of motor skill, like playing tennis or riding a bicycle.
A~split infinitive should really jar, Nils said: ``It's got to light
up in red!'' The d{\ae}mons have to run automatically; you can't be
consciously checking a list of rules all the time. Besides, if writing
is to be fun, it can't be compulsive!

\smallskip
\display 20pt:{\bf 5.}:
{\bf Master the Medium.}\quad You need a good vocabulary, though this
needn't mean a huge list of big words. There are issues other than
pure language: indexes, tables, graphs, and how to use them to best
effect. As Don pointed out earlier, we can use typography to make
important distinctions, as with the typewriter font for logical
formul{\ae}.

In the future, said Nils, it's clear that reading and writing will be
far more interactive processes---{\sl The Media Lab\/} is not all
hype.  It's not clear yet what will prove necessary or useful; just as
it took several centuries to invent the index, it will probably take
us a long time to identify the ``stable points'' offered by our new
technology. We in the audience are at the cutting edge of these experiments.

\smallskip
\display 20pt:{\bf 6.}:
{\bf Master the Material.}\quad There's a lot of internal feedback
involved in writing; one comes to understand the material in a new way
on trying to organise it for publication. Nils drew this diagram :

\bigskip
$$\vcenter{\halign{\hfil#\hfil\qquad&#\hfil\cr
Internal\cr
Model\cr
\noalign{\bigskip\bigskip}
&re-organisation\cr
\noalign{\bigskip}
\qquad\quad writing\cr
\noalign{\bigskip\bigskip}
Text\cr
}}$$

\bigskip

As Mary-Claire said on Wednesday,``How do I know what I mean until I
hear what I~say?'' Even Nils sometimes finds himself thinking 
``I~don't believe that!'' when he hears himself lecture. I~am reminded of
the (true) story of a professor who was always seen to take a pad of
blank paper with him when he delivered a talk. When asked what it was
for, he replied: ``Why, if I say anything good I'll want to write it
down!'' So go to lectures and classes, give talks. All these things
help modify your internal model and get things into shape.\looseness=-1

{\narrower\smallskip\noindent
``In a very real sense, the writer writes in order to teach himself,
to understand himself; the publishing of his ideas, though it
brings gratifications, is a curious anticlimax.''  
\smallskip}
\line{\hfill--- Alfred Kazin\qquad}

\smallskip
\display 20pt:{\bf 7.}:
{\bf Simplify.} \quad Lie, if it helps. You can add the correct details later
on, but it is essential to present the reader with something
straightforward to start off with. So don't be afraid to bend the
facts initially where this leads to a useful simplification and then
pay back the debt to truth later by gradual elaborations.

{\narrower\smallskip\noindent
``Another noteworthy characteristic of this manual is that it doesn't
always tell the truth.''
%\dots Deliberate lying will actually make
%it easier for you to learn the ideas.''
\smallskip}
\line{\hfill--- Don Knuth, {\TeXbook} (page vii)\qquad}

{\narrower\smallskip\noindent
``Everything should always be made as simple as possible, but not simpler.''
\smallskip}
\line{\hfill--- Albert Einstein\qquad}

Ted Shortliffe  did a great job with Mycin, Nils
said. But with 20/20 hindsight he might have done better to invent a
simplified system for expository purposes. For example, he could have
demonstrated the backward-chaining techniques and only later dealt
with ``certainty factors.''

By using simple examples we can get ourselves on the winning side of
the 80-20 rule: we can convey 80\% of the truth with only 20\% of the
difficulty. Mathematicians, of course, like to go the other way: They
never state a theorem in three dimensions if it can be generalised 
to~$n$. Such terse elegance can be painful for the reader.

\smallskip
\display 20pt:{\bf 8.}:
{\bf  Avoid Recycling.}\quad With online text and sophisticated
editors (I~refer to software, not the mandarins behind {\sl Scientific
American\/}) it is very tempting to re-use portions of old material.
Resist the temptation. Almost certainly you are writing in a new
context, with a new emphasis. Hopefully you are older and wiser, and
perhaps even a better writer than you were when the old material was
written. So do rewrite it, it's worth the extra effort.

\smallskip
\display 20pt:{\bf 9.}:
{\bf Aim for Excellence.}\quad You've got to keep shooting for perfection,
even if you'll never get there. What the Great have said on this:

{\narrower\smallskip\noindent
 ``We are all apprentices in a craft where no-one ever becomes a master.''
\smallskip}
\line{\hfill --- Ernest Hemingway\qquad}

{\narrower\smallskip\noindent
 ``Someday I'll build the perfect birch-bark canoe.''
\smallskip}
\line{\hfill --- John McPhee\qquad}

{\narrower\smallskip\noindent
 ``Someday I'll write the perfect AI textbook.''
\smallskip}
\line{\hfill --- Nils Nilsson\qquad}

{\narrower\smallskip\noindent
 ``Ah, but a man's reach should exceed his grasp, or what's a heaven for?''  
\smallskip}
\line{\hfill --- Robert Browning\qquad}

{\narrower\smallskip\noindent
``The message of these books is that, here in the 80s, `good' is no
  longer good enough. In today's business environment, `good' is a word
  we use to describe an employee whom we are about to transfer to a
  urinal-storage facility in the Aleutian Islands. What we want, in our
  80s business executive, is somebody who demands the best in
  everything; somebody who is never satisfied; somebody who, if he had
  been in charge of decorating the Sistine Chapel, would have said:
       ``That is a good fresco, Michelangelo, but I want a
        better fresco, and I want it by tomorrow morning.''
\smallskip}
\line{\hfill --- Dave Barry\qquad}
\beginsection 35. [Mary-Claire van Leunen on Calisthenics (1)] \tll November 25

Don opened class by introducing guest lecturer
Mary-Claire van Leunen and by giving us
the title of her talk: ``Calisthenics.''  Mary-Claire opened her talk by
telling us a story.

Many years ago Mary-Claire was a frequent passenger on the Chicago bus
system.  The neighborhood where she boarded her \#5 bus was a gathering
spot for ``bummy guys.''  All of these guys were interested in money: Some
begged, others peddled.  Among the peddlers---hawking wares ranging from
trenchcoats full of watches to freedom from the peddler's presence---was a
man whom Mary-Claire patronized quite regularly.  
He sold pencil stubs (obviously collected from trash bins); but
Mary-Claire said his  patter was charming enough to rate one or two
purchases a week.

{\narrower\smallskip\noindent
    ``These pencils are magic pencils,'' he would say.  ``Buy a magic pencil.  
    Only 25 cents.''
    
    ``What's a magic pencil?'' would come the expected response.
    
    ``With this pencil, you can write {\it the truth}.''
    
    Inevitably, someone would pipe up, ``But I can write lies with it.''
    
    ``Oh, you can break the magic.  But if you really believe, you can write
   {\it the truth}.''
\smallskip}

Mary-Claire sees this as the wonder and the motivation behind the craft of
writing: If you work hard, you can explain a new truth to someone you
will never meet---perhaps to someone who will live after you are dead. 

Such a vocation requires preparation.  The Composition Exercises that
Mary-Claire has given us (see \S{36} below) were designed to help us become as
strong as we can.  Our readers are more likely to be tolerant of a few
weaknesses if they are surrounded and supported by strength.

Mary-Claire has given these exercises to students before, but preparing
this draft for our class pushed her to really {\it write\/} the exercises.  The
copy that she referred to over the TV monitors was slightly different than
the copies that we have been given.  Mary-Claire, hoping that these
differences represent improvements, invites us to suggest further
improvements to the draft. (She says that she might publish something that
evolves from this draft---but probably not soon: She is not a fast writer.)

The first set of exercises, labeled ``Vocabulary,'' is designed to increase
our command of just that.

The first of the pair is an exercise that was done by little Greek boys:
taking a composition and swapping all the old words (nouns, verbs,
adjectives, and adverbs) for new ones. What is the effect of these
changes?  What happens when a vulgarism is used?  When a hoity-toity word
is used?

The second vocabulary exercise, the writing of a thesaurus entry, is best
done over a week.  After several days of slowly adding to our set of
synonyms, we should compare our entry with an entry in our thesaurus.
(Everyone needs at least one thesaurus and a good unabridged dictionary.
In addition to more than one kind of dictionary, Mary-Claire recommends
Sidney Landau's book {\sl Dictionaries\/} to help us understand how to best use
our dictionaries.)

``Syntax,'' the next set of exercises, deals with syntactic mastery.
Mary-Claire says even though vocabulary improvement is more often
considered than increasing our command of syntax, syntactic armory
improvement is more important.  She says that most of the time we will use
our basic three to four thousand words; we must use them in the most
interesting way possible.

Speaking of using words in interesting ways, Mary-Claire has been reading
the first draft of our term papers. There must be room for some
improvement there:  Her first comment  was, ``Nobody sits down to
write a boring paper.''  How can we tell when something we write is
``syntactically impoverished''?  She gathered  some statistics that might
help us get the right idea.

One of her tricks was to study
 the first 10 complete sentences on the third page of every paper.
First she charted the
average length of the 10 sentences: They varied from 15.6 words per
sentence to 24.4 words per sentence.  Mary-Claire says that any of us with
averages under 20 words per sentence are in the correct range for adult
writing. (But the writer with the 24.4 average had better have
results pretty wonderful, to compensate for the
extra work that it takes to read his
paper.)

Sheer variation in sentence length is one indication of syntactic
variation and appropriate pacing. With 10~sentences we should be aiming
for 9 or~10 different lengths.  The samples from our papers yielded 6 to~9
different lengths.  The difference between the word count on the shortest
sentence to the word count on the longest varied between 17 words to 37
words.  The ideal chart of sentence lengths should look like a bell-curve
centered around 15 to~18 words per sentence.   

She asks us to note that we did not have enough short (``and punchy'')
sentences.  A~few long sentences are also important.  She said, ``A~well
constructed 46-word sentence is not a difficult beast, but it had better
not be your crucial point.''  We should remember that we have a
responsibility to emphasize and deemphasize our points to the reader; long
sentences are one method of deemphasizing a point.

Beyond the word counts, she looked at the templates used to construct
our sentences.  For example, she found two writers who would appear to be 
similar if we just looked at their sentence length average and variation,
but who had quite different methods of constructing their sentences. One of
these writers used the same sentence construction for almost every sentence
(adverbial $+$ subject $+$ transitive-verb $+$ object),
 and the other used many different styles of
construction.  But the second writer was not free of flaws.  He had two
sentences in a row with a full independent clause followed by a full
parenthetical independent clause. Mary-Claire says that we must learn what
syntactic usage is unusual so that we do not overuse it.

One syntactic trick not normally thought unusual was startlingly absent
from our papers: tight parallels.  The use of two adjacent sentences with
exactly the same construction is an effective way to communicate
similarity to our readers.  Mary-Claire is curious how we could all avoid
this technique.  Perhaps it is an artifact of the way that students of 
our  generation were taught?

Given this motivation to ``increase our syntactic muscle,'' Mary-Claire led
us back to discussing the ``Syntax'' exercises.  She passed over the first
two exercises as obvious, but a few comments were made on ``periodic''
sentences.

We who have had mathematical training might be tempted to guess that a
periodic sentence is one that repeats cyclically, but we would be wrong.
Periodic sentences are those whose grammatical and physical ends coincide:
We must get adverbials out of the final position. For example, a~verb that
is intransitive must end the sentence.  Period.

Periodic sentences are not really appropriate in our kind of writing; they
are a high literary form.  Even though such a sentence form is more
frequently encountered in church than in conference papers, the use of
periodic sentences will heighten our awareness that we can control 
sentence structure.  

The next exercise has us recast a sentence so as to change emphasis.  What
are the emphatic positions? The front of the sentence and the back. (``The
middle of a sentence is sort of a slum.'')   But she says not to take her
word for it; we should write sentences with varying emphasis and find out
for ourselves.

Mary-Claire says that the last syntactic exercise is ``incredibly
wonderful'': Write nonsense.  Write a completely unrelated stream of
thoughts with the correct glue: words like ``thus,'' ``therefore,'' and ``as we
can see.''  She says this is a fun exercise to do after a couple of drinks.
(Maybe we need a class lab?)

Moving on to exercises labeled ``Manual labor,'' Mary-Claire told us that
these should logically come first, but she wanted to woo us with the
logical stuff before we ran into the weird stuff.  Why is it important to
use different methods to copy other people's writing? Because writing---and
even reading---is partly a manual process.  Mary-Claire typed out large
sections of our papers as she was analyzing them. She said that if you
tie a baby's hands behind his back, but give him otherwise adequate
mental stimulation, he will not learn to speak well.

When we want to  read a passage of text seriously, such manual labor can
help us slow our brains down until we can give the passage the
consideration it deserves.  (W.~H. Auden said that the proper way to show
contempt for a poem is to copy it on a typewriter;  the proper way to
show admiration is to copy it in longhand.)  Memorization and recitation
can also help us to be able to read {\it word\/} by {\it word}. 
(``Make yourself into
a book that you can take to prison should worse come to worst.'')

Mary-Claire is aware that we may not buy this ``manual labor'' technique at
first, but she asks us to take it on faith. She took the technique on
faith for ten years and then wrote a sincere Thank-You to
the person who told her about~it.
Ten years from now we can write her and tell us our opinions.

Most writers are aware of how important the manual part of composition is:
They have very rigid restrictions on how they compose. (``Oh, I~can only
write on yellow pads with a fountain pen.'')  Mary-Claire
says we should be able to compose on a cocktail napkin.  

While discussing the section labeled ``Frozen sounds,'' Mary-Claire told us
about reading aloud to her students their own writing. Some students were
chagrined; others glowed.  She says we should form partnerships with other
novice writers: Read and listen to each other.  But she cautions  that a
little goes a long way.  If the writing is good, we can live on that joy
for quite a while; if the writing is bad, we won't be able to stand it
for very long.

At this point in the lecture, Mary-Claire noticed that very few
minutes remained. So her comments on the final exercises were limited to
 those that she thought were the most important.

Concerning the ``Marks on paper'' exercises, Mary-Claire quoted from 
E.~M. Forster: ``How do I~know what I~mean  till I~see what I~say?''  We need to
remember that writing is ``the most forgiving medium known to man.'' We can
work on it until we get it right.

 Rushing past the ``Stance, voice, and tone'' section, she told us that
she borrowed techniques from speech therapists---who ask patients to
exaggerate their defects until they understand just exactly what
characterizes their defects.  For example, she says, ``If any one has ever
told you that you are `breezy,' write something truly off the wall.''

She told us that the sections labeled ``Observation,'' ``Same as and different
from,'' and ``Invention'' are less important for us than for pure writers.
Our discipline provides the glue that writers with more freedom have to
manufacture from scratch.

She warns us that the ``Scansion'' exercises are hard, but very important.
She realizes that she may have trouble convincing us that we need to write
verse in order to learn to write mathematics, but once again she says,
``Trust me.''

She reminded us that the ``Precis'' exercises were touched upon by Leslie
Lamport in his talk.  At some point we cannot reduce the word count of a
piece of prose without changing the structure of that prose. (We should
never change the meaning, but we will have to dispense with some details.)
This point comes at different percentages of decrease---depending on the
flabbiness of the original text.

The final exercises she discussed, ``Nearly real,'' are aptly named.
They really are very much like real writing.  For instance, Mary-Claire
says that  ``Writing a joke is exposition at its purest.  Things aren't
funny unless they are well written.''

She suggests that we try ``Ben Franklin's exercise,'' rewriting a passage of
someone else's from memory and limited written hints---but that we try it
with Don's writing. When we have finished, what do we like better about
Don's version?  What do we like better about our own?

Before the cameraman could shoo us out of the room, Mary-Claire reminded
us once again that these exercises are ``very hard work.''  She closed with,
``I~hope they will serve you as well as they have served me.''

Some of us surely hope the same.

\beginsection 36. [Mary-Claire's handout on Composition Exercises]

\vbox to 4in{}
\vfill\eject
\null\vfill\eject
\null\vfill\eject
\null\vfill\eject
\null\vfill\eject
\null\vfill\eject
\null\vfill\eject
\null\vfill\eject
\beginsection 37. [Comments on student work] \pmr November 30

{\narrower\smallskip\noindent
{\it During the whole of a dull, dark, and soundless day in the autumn
of the year, when the clouds hung oppressively low in the heavens, 
I~had been passing alone, on horseback, through a singularly dreary
tract of country; and at length found myself, as the shades of evening
drew on, within view of the melancholy Terman Engineering building.}
\smallskip}
\line{\hfil --- E. A. Poe (amended)\qquad}

Don, like Mary-Claire, scans the pages of {\sl The New Yorker\/} for
choice malapropisms to entertain us. In its columns the law firm of
Choate, Hall, \& Stewart had been rendered as Choate, Hall, Ampersand,
and Stewart, presumably by a journalist receiving dictation over the
telephone.

We also saw a splendid dangling participle from the same source:

{\narrower\smallskip\noindent
	      ``Flavor and texture of cooked okra are
		different from other vegetables. We 
		usually don't eat it raw, but in judging
		at fairs, I frequently taste a slice of
		a pod to check maturity and condition.
		In soups, it is used as a thickening
		agent. When fried, I love okra.''
\smallskip}
\line{\hfil [When sober, can't stand the stuff. --- {\sl The New Yorker\/}]\qquad}

Don announced that he had good news and bad news for us. He gave us
the good news first. Mary-Claire is to speak again on Wednesday.
Also, Don finally got up the courage to ask Paul
Halmos to appear in our guest spot; he readily agreed and will speak
next Wednesday (9$^{\rm th}$~December). This talk should be a
fitting climax to the course.  And a week from today (Monday,
7$^{\rm th}$~December) we will hear from Rosalie Stemer, 
a~copy-editor for {\sl The San Francisco Chronicle\/}.

Having thus softened us up with these cheerful tidings, Don delivered
the Bad News: The first drafts of the term papers were, well \dots
``their content was not one hundred percent pleasing to your
instructor.'' What makes a professor's life worthwhile? The knowledge
that he has succeeded in teaching something. In particular, there's a
joy in the thought that a student was able to do something that he
couldn't have managed without the professor's help. Don confessed that
this joy did not run through him as he read our drafts.  Indeed, he
could almost think that many of them were written before the class
started. Have we been relaxing too much, he wondered?  Has our writing
in fact changed at all? Have we learnt nothing? Disturbing thoughts,
he said.

Of the thirteen papers submitted, eleven were sprinkled with wicked
whiches---at least two in each. Don himself has been guilty of
these in his time, and of course there is no-one like a convert for
rooting out heresy.
  But these are the 80s and we are supposed to be sensitised
to these things. And heaven knows, we've talked about this issue enough
in class!  So what is he to think about this landslide of
carelessness?  Shaking his head, Don declared that we left him with no
alternative---he would have to resort to the ultimate sanction: 
a~quiz. 

In keeping with Honor Code protocols, Don left the room while we
each wrote a sentence that used a `that' correctly where a `which'
would have been wrong, and another complementary
sentence---which used a `which'
correctly where a `that' would have been incorrect. A~minute passed.
And then another minute.  
 There was little to hear but the scratching of pencils
and the beating of hearts. 

Don returned. We spent a few minutes
looking at the various examples that the class had come up with, some
correct and some incorrect.  
By and large, the class redeemed itself by the creative solutions
that were submitted:

{\narrower\smallskip\noindent
All the students that know when to use `which' and `that' will pass the
quiz. The exam, which took place at the beginning of class, was not
difficult. 

A paper that uses two whiches improperly does not demonstrate that
the author hasn't learned anything. My first draft, which was written
this summer, had a million of them.

Beware of examples that are misleading. My term paper, which contains
many wicked whiches, is otherwise not too bad.
\smallskip}

CS-types
just love self-reference, it seems. 

Is it not true, TLL asked of
Mary-Claire, that people invariably get their whiches and thats
right when they speak?  Mary-Claire replied that people 
almost never say `which' improperly
 in general speech---it's only when they
feel under pressure that they resort to this unnatural diction.  So
unnecessary use of `which' really conveys a bad tone in your writing;
it makes you sound nervous.  (Conversely, on paper we can often fool
our audience into thinking that we are a lot more comfortable than we
really are).

Don observed that all translations of the Bible are strewn with
erroneous whiches. (``Thou shalt not suffer a wicked which to live,''
he might have said.) A~clamour of voices pointed out that Fowler is
quite clear about the rule. True, but it was never enforced until the
late 70s, Don countered. It seems particularly strange, he said, that
the {\sl New English Bible\/} should commit this error, as its editors
take great pride in the literary qualities of their text. Mary-Claire
resolved this mystery: Apparently our ``oldest and closest allies''
on that far-off island
regard this whole issue as unmitigated nonsense!

Don made a final plea to us: ``You all keep your text online, so it's
very {\it very\/} easy to locate all your whiches and check them.
Please don't cause your instructor any more pain on this score!''

Sneaky Don had saved one more item of good news to lighten our spirits
after this depressing interlude: A~letter from Leonard Gillman, editor
of the Sierpi\'nski  proof over which
we had laboured many moons ago.  Professor Gillman was lavish [not `fulsome',
according to page 23 of his book] in his
praise of our suggestions, and is now working on an improved write-up.
Particular credit went to Student~B, of course.  Gillman is an Emeritus
Professor as of three months ago---Don drools to think of all the free
time he must have.

\vskip\parskip
\vglue 5\baselineskip
\goodbreak

We moved on. Don claimed to have discovered a new (?) rule only by seeing it
broken in three of the papers he read. It is this: The text
should make sense if we read through it omitting the titles of
subsections. So, for example, don't say:

{\narrower\smallskip\noindent
{\bf 2. Contour Integration.} This technique, invented by
Cauchy, is used \dots
\smallskip}

Rather, say:

{\narrower\smallskip\noindent
{\bf 2. Contour Integration.} The technique of integrating along curves
in the complex plane, 
invented by Cauchy, is used \dots
\smallskip}

The point is that the subheading should not be referred to explicitly
or regarded as  an ``integral''
part of the text. Think of it as some kind of marginal note or
meta-level comment.

We spent the rest of the time looking at the draft of
a paper about graph theory
written by Ramsey Haddad and Alex Sch\"affer. Firstly, Don pointed to
their careful attention to definitions. This is particularly important
in graph theory as various authors use terms inconsistently.  A~{\it
path\/} may or may not be the same thing as a {\it simple path}.  At
least one writer uses the term {\it walk\/} to make a distinction here.
So it is necessary to define one's graph theory
terms clearly right at the start, even the most basic ones.
Remarkably, there was a time when the symbol `$=$' was not in
general use. Fermat never used it, preferring always to write 
`{\ae}q' or `ad{\ae}q' or fuller Latin words 
like `ad{\ae}quibantur'
that these terms abbreviate.
So in those days you would have to define the symbol `$=$'
at the beginning of your article if you intended to use it. 
(The equals sign was invented by Robert Recorde in his {\sl Whetstone
of Witte}, 1557, but it did not come into general use until more
than a hundred years later. Descartes used `$=$' to mean something
 completely different.)
The moral: Ask yourself what background your
readers share, and what they may or may not have in common.  ``Be
aware of what's diverse in your readership.''

\def\ato{\buildrel \ast\over\to}%asterisk over 
\def\pto{\buildrel +\over\to}%asterisk over 

We saw a somewhat intimidating multi-part definition. It would become
less formidable to the reader if shortened. In this particular
case, the expression
$$W1W2\ato W3$$
could have been condensed to
$$W1\pto W3$$
since $W2$ is used nowhere else. (In the Haddad-Sch\"affer paper,
`$\ast$'~means `zero or more' and 
`$+$' means `one or more', but let's not worry about that here.) Try to be
succinct, said Don: ``Less is more.''

It is important to be consistent in your use of terms, and you need to
be especially careful about this when working with coauthors. In this
paper, one writer talked about `dominators' and the other about
`parents', referring to the same concept. (Freudian slip?)
A~related issue: Don't
define terms that you never use. Don recalled Feynman's complaint
about New Maths: you are taught the symbols $\cap$ and
$\cup$ in second grade, but you don't use them in any nontrivial way
 for seven years.

Next came a tricky question of tenses. ``Gabow and Tarjan[Gab83] show
that for many algorithms that had such a multiplicative factor in their
worst-case complexities, the multiplicative term can be removed.''
Here `had' should be `have'; an algorithm lives forever, and its worst-case
complexity is a timeless fact about~it. However, the {\it problem\/} solved
by an algorithm can have different known complexities at different times;
therefore `had' would be okay if `algorithms' were `problems'.
(The quoted sentence also exhibits other anomalies. 
A~`multiplicative factor' is not also a `multiplicative term'; factors are
multiplied, terms are added. Also the logic of the sentence can be
unwound
to make the point clearer: ``Gabow and Tarjan have shown how to improve
the algorithms by removing such a multiplicative factor from the worst-case
complexities in many cases [Gab83].'')

We talked about abbreviations for bibliographic references.  Don
didn't like the lack of space before the bracket in 
``\dots Tarjan[Gab83] \dots''; 
neither does he like this kind of thing: ``In
[Smith{\thinspace}80] it was shown \dots''.  References should ideally
be parenthetical; we
should be able to read the sentence ignoring them and still have it
make sense (cf.~subheadings). Some citation styles
 write up names and
dates in full, but this can get repetitious: ``\dots  Knuth [Knuth{\thinspace}83]
has shown that \dots''.  Don's paper on {\bf goto}'s was published
first in ACM {\sl Computing Surveys\/}
 and later incorporated into a book. For this
second printing he had to make numerous changes to the sentences containing
citations,
 because the originals would look strange in the different
context and format of the book. Oren Patashnik pointed out that the
{\sl Chicago Manual of Style\/} recommends leaving references unnumbered,
lest you have to make changes all through the text every
time you insert a new one.  This is less of an issue when a system
like \TeX\ handles such things automatically, of course. The {\sl CMS\/} is
full of such efficiency tips.

Too many commas can be a bad thing (bad things?). For example, 
consider this sentence: ``Our algorithm to recognize and label the
graphs when given a directed graph,~$G$, with distinguished vertex~$s$,
can be summarized as follows.''
Remove the commas around~`$G$' and put one after `graphs'. As a
rough guide, put a comma where a speaker would pause to draw breath.

The word `loop' was ambiguous when first used; Don replaced it by
`self-loop'.

A sentence in the paper began ``If any $Hj$ ($j>0$) 
has \dots''.  In fact it was known that $H0$ satisfies the stated
condition, so Don suggested that the authors simplify the statement by
omitting the $j>0$ condition. Moral: Give a simple rule
rather than an optimal one.

Elsewhere we saw ``\dots all the $Hj$'s \dots''. This is of course
the standard way to form the plural of a symbol, but you are going to
get into trouble when you start also using the construct 
$Hj'$ (that is, $Hj$ primed). A~simple way to avoid the
problem is just to say: ``\dots each $Hj$ is \dots''.
Alternatively, you might want to invent another name for the concept,
particularly if you are going to be using it time and time again. It's
just not elegant to have too many symbols crowded on the page. At
one point the authors wrote ``\dots of $Hj$s descendants \dots''.
This doesn't work at all; you do need an apostrophe for the genitive
(possessive) case.

Three small points:

{\narrower\smallskip\noindent
Instead of ``\dots the one vertex path \dots'', write ``\dots the one-vertex
path \dots''.

The preposition `at' would be better than `of' in ``\dots vertex of
distance $<d$ \dots''.

We certainly need a space here: ``\dots using 4.3(below) we derive \dots''.
\smallskip}

Some authors have a disconcerting habit of  using a lemma
or theorem that is not proved until later on in the book. This can
leave the reader wondering whether someone hasn't pulled a fast one on
him (essentially by using a result to prove that same result). So make
it quite clear to the reader that your proof structures do respect the
necessary partial ordering.

Using ties: \TeX\ and other systems allow you to specify that certain
blanks are not to be used as line breaks. 
For example, put ties after the word `dominates' in the phrase 
`$v$~dominates~$x$ and $x$ dominates~$w$'.
In ``\dots if $e$~is \dots''
it is best to put such a tie between~$e$ and~`is'. 
The idea is to keep line breaks from interrupting or distorting the message.

Beware of the unfortunate co-incidence!  Sometimes we cannot use an
idiom because some word is also being used in another sense. For
example: ``\dots $n$~vertices have been deleted by this point.'' In one
of the term papers, someone was using contour integration to study
aerodynamics. There were airplanes and complex planes all over the place,
 much to Don's
confusion. Another example that came up was the word `left'. This can
be either left as against right (in a tree structure, say), or a past
tense of the verb `to leave'. So `the node~$x$ is left' might be
ambiguous.

In a final remark for today, Don suggested that a sequence of examples
that build upon one another is much more useful than a number of
unrelated ones.
The paper by Haddad and Sch\"affer has a particularly nice sequence of
illustrations demonstrating this point.

After class, everyone got back two independently
annotated copies of their term papers.
\beginsection 38. [Mary-Claire van Leunen on Which vs.\ That] \tll December 2

Don welcomed Mary-Claire van Leunen to her encore lecture by pointing out
the intriguing books that Mary-Claire had placed on the desk; he said
he hoped that she could now tell us all the things that the clock had
prevented her from telling us last lecture.  Mary-Claire countered by
saying that Don had only invited her back ``on the theory that no one could
possibly be that nervous two weeks in a row.''

Leaving the books alone for the moment, Mary-Claire told us the tale of
{\it which\/} and {\it that}.

The story opens in the 17th century, when speakers of English have two
relative pronouns: {\it which\/} and {\it that}.  
What are relative pronouns?  Here are
some sentences (examples taken from the {\sl Concise Oxford Dictionary\/}) 
where
{\it which\/} and {\it that\/} are used as relative pronouns, both singular
and plural:

\smallskip
{\obeylines\parskip=0pt
\qquad\qquad\qquad Our Father which art in Heaven \dots
\qquad\qquad\qquad These are the ones which I want to learn.
\medskip
\qquad\qquad\qquad \dots the one that I mean \dots
\qquad\qquad\qquad These are the ones that I want to learn.
}

\goodbreak
\smallskip
{\it Which\/} and {\it that\/} 
are not always used as relative pronouns.  Here are some
sentences (again from the {\sl Concise Oxford\/}) where they serve
other functions (along with the technical term for the function 
they are serving):

\smallskip
\halign{\qquad\qquad#\hfil\qquad&#\hfil\cr
Which?  Say which.&(interrogative pronoun)\cr
Which one?  Say which one.&(interrogative adjective)\cr
\dots during which time we can \dots&(relative adjective)\cr
\noalign{\medskip}
I like that.&(demonstrative pronoun)\cr
I like that thing.&(demonstrative adjective)\cr
\dots not all that wonderful \dots&(adverb)\cr
\dots no doubt that he can \dots&(subordinating conjunction)\cr
O that we could!&(particle)\cr}

\medskip
We have no spoken evidence from the 17th century, but Language Theorists
believe that writing and speech were very far apart. That is, 
they believe
that no
one's ideal was to write the way that he sounded. Theorists cite two
pieces of evidence to support this claim: The first is that the Theorists
themselves find it difficult to believe that, in the last three centuries,
spoken English has evolved as fast as it must have if the  written
language and the spoken language originally matched.  The second piece of
evidence comes from examining extant 17th century guidelines on writing or
speaking effectively.

By examining samples of writing from the 19th century (particularly the
everyday writing that was used for communication rather than as examples
of great literature), we can see that the written language has evolved
into one much closer to the spoken language.  Language Theoreticians of
that time said that this evolution was good, but their admonitions came
after the direction of evolution was already evident.  (We should remember
that our language belongs to millions of people.  It cannot be controlled
by the decrees of any one person or group.)  We now move on to our
own century.

In 1906 H. W. Fowler and F. G. Fowler published {\sl The King's English\/}
(Oxford University Press still has it in print).  Here
the brothers Fowler
write down for the first time that conversational rhythms are to be
reflected in written English.  

In 1926 H. W. Fowler published {\sl Modern English Usage\/} (also still
available from Oxford University Press).  In this book the surviving
brother continues the explanation of the relationship between spoken and
written English---but he does so much more clearly.  

While we are following the hot trail of our current subject, we should not
lose sight of the vast range of the contributions that Fowler made in
this landmark book. Mary-Claire calls Fowler the ``great theoretician of
the semicolon.'' Fowler saw the semicolon, which has no spoken
equivalent, as a structuring device that operates between the levels of
the sentence and the paragraph.  This is just one example of how Fowler
tried to utilize the graphical nature of print to the advantage of written
English.

Returning to the evolution of written English toward  spoken
English, let's examine how people use {\it which\/} and {\it that\/} 
when they talk.

Speakers do not use {\it which\/} as a relative pronoun because speakers do not
normally express thoughts that are long enough to contain non-restrictive
clauses: Our spoken sentences are shorter than our written ones. People do
use {\it which\/} when they talk, but they use non-referential whiches to
introduce new thoughts that are tacked on to old thoughts.  Examples of
this kind of usage seem strange when written down (because we don't use
non-referential whiches in written English), but they sound perfectly normal
when heard on the street.  Here is one:

\nobreak\smallskip
{\obeylines
\qquad\qquad\qquad I went sailing this weekend; which tells you why my nose is pink.
}

\smallskip
Fowler realized that written English would sound more like speech if the
choice of relative pronoun was uniquely determined by whether or not the
clause it introduced was restrictive or non-restrictive.  He wrote several
thousand words on this subject; here are a few of them: 

{\narrower\smallskip\noindent
    A supposed, and misleading, distinction is that `that' is the
    colloquial and `which' the literary relative.  That is a false
    inference from an actual but misinterpreted fact.  It is a fact
    that the proportion of `that's to `which's is far higher in speech
    than in writing; but the reason is not that the spoken `that's
    are properly converted into written `which's.  It is that the
    kind of clause properly begun with `which' is rare in speech with
    its short detached sentences, but very common in the more complex
    and continuous structure of writing, while the kind properly begun
    with `that' is equally necessary in both.  This false inference,
    however, tends to verify itself by persuading the writers who
    follow rules of thumb actually to change the original `that' of
    their thoughts into a `which' for presentation in print.

    The two kinds of relative clause, to one of which `that' and to
    the other of which `which' is appropriate, are the defining and
    the non-defining; and if writers would agree to regard `that'
    as the defining relative pronoun, and `which' as the non-defining,
    there would be much gain both in lucidity and in ease.  Some there
    are who follow this principle now; but it would be idle to pretend
    that it is the practice either of most or of the best writers.
\smallskip}

There is no doubt that Fowler has had a significant influence on the
English language, but why is it that his effect on American English has
been greater than on British English?  To answer that question, we move
our focus to New York in the year 1925: Harold Ross has just founded the
{\sl New Yorker\/} magazine.

Ross was a man who liked things to be clearly defined.  He took {\sl Modern
English Usage\/} as gospel.  For decades the {\sl New Yorker\/} had reliably
influential prose, and for decades H.~W. Fowler's dictums were applied
blindly to that prose.  Mary-Claire was nearly nonplused as she
mentioned reading a collection of letters from a {\sl New Yorker\/} editor to
various literary luminaries. (``I'm sorry, but we had to change all your
{\it which\/}es to {\it that\/}s,'' sounds rather presumptuous when addressed to
John Updike.) The {\sl New Yorker\/} no longer treats Fowler as divinely
inspired, and they haven't since the 1950s, but that leaves close to
three decades of blind obedience to consider.

According to Mary-Claire, Harold Ross's attachment to obedience is not
unusual---for Americans.  She says that Americans look in a reference work,
see what it says, and then decide to  obey  or to disobey; Britishers look
in the same reference, see what it says, and then formulate new and
different ways of treating the same questions. Why is it that Britishers
feel that they are entitled to an opinion but Americans do not?  Two
partial answers might combine to give us a single satisfactory one.

Most British people have been English speakers for generations, but most
Americans are descendants of recent immigrants.  Immigrants are told, ``These
are the facts.  If you want to speak English, follow the rules.''  Perhaps
more important, educated British people are taught to write from day one.
Many of the exercises that Mary-Claire gave us in her lecture on
Calisthenics are actually used in British Grammar Schools.  British
University students discuss their weekly writing with their tutors---and
they regularly write about 2000 words a week.

In 1957, we Americans acquired a new source of authority on writing
English---this time {\it American\/} English.  An old classmate of E.~B. White's
sent him a copy of the book that they had received from their English
professor, Will Strunk.  The decision-makers at Macmillan decided to
publish a book that contained Strunk's monograph plus an extra chapter by
White on ``spiritual things.''  The combination of Strunk's clear and simple
instructions and White's beautiful prose made {\sl The Elements of Style}, by
William Strunk, Jr., and E.~B. White, the landmark of written style for
our generation.  Here is their entire essay on {\it which\/} and {\it that\/}:

{\narrower\smallskip\noindent
``That'' is the defining or restrictive pronoun, ``which'' the non-defining
    or non-restrictive.  See under Rule 3.
\smallskip}

{\narrower\narrower\smallskip\noindent
        The lawn mower that is broken is in the garage.  (Tells which one.)

        The lawn mower, which is broken, is in the garage.  (Adds
        a fact about the only mower in question.)
\smallskip}

Rule 3 says ``Enclose parenthetic expressions between commas.''

Mary-Claire has a copy of the first edition of {\sl The Elements of Style}, in
which White uses a `which' for a `that' 
(this has been changed in later
editions).  The line originally read:

{\narrower\smallskip\noindent
\dots a coinage of his own which he felt was similar to \dots
\smallskip}

{\sl The Elements of Style\/} has many departures from guidelines presented by
Fowler.  It was written for the American audience, and it was written for
an audience without a high level of grammatical sophistication.  In
contrast, as Mary-Claire said, ``Fowler is rough going for those of us
whose Latin is weak and whose Greek is non-existent.''  Future editions of
Fowler may need prefaces explaining what adjectives, adverbs, and the like
are. It is most common for people to learn those terms when they learn
their first non-native language (though Latin is the only language to
which the terms are perfectly suited).

Fortunately for native English speakers, there is a rule completely
lacking in jargon that we can use to determine whether a `which' should be
a `that': 

{\narrower\smallskip\noindent
   If you can substitute `that' for `which', do it.  
\smallskip}

Mary-Claire attributed this rule to Leslie Lamport.  Leslie says that his
version of the rule is actually: 

{\narrower\smallskip\noindent
   If it sounds all right to replace a `which' by a `that',   
   then Strunk \& White say replace it.
\smallskip}

Which brings us to our next issue: Are {\it which\/}es that could be 
{\it that\/}s always
wrong?  Don said that now that he knows the rules, he finds every ``wicked
which'' an irritating distraction from his reading
enjoyment.  He seemed to imply that {\it which\/}es in restrictive
clauses are always wrong.  

Mary-Claire said that the rules given hold for ``everyday, expository
prose---shirtsleeve prose, not literary prose.''  (She did not tell us how
to decide when the everyday rules should be violated.)  As for Don's
discomfort on finding {\it which\/}es in 
well-beloved authors, she said ``I~believe we
are encountering obedience here.''  After class, Leslie Lamport had this to
say on the same subject:

{\narrower\smallskip\noindent
    I unfortunately have somewhat the same reaction as Don to ``incorrect''
    uses of {\it which}, for which I curse the evil influence of Strunk
\&  White.  When I observed that writers such as Dickens and Fowles
    are ``incorrect,'' I~quickly lost my desire to be ``correct.''  However,
    I~can't completely unlearn the reflex of being bothered by the
    ``incorrect'' usage.

    I still try to use {\it that\/}s  when Strunk \&  White tells me to,
    because I know that many of my readers will have been similarly
    indoctrinated.  But I will throw in an occasional wicked {\it which\/}
    to avoid a string of {\it that\/}s.
\smallskip}

Mary-Claire's final word on the subject: 

{\narrower\smallskip\noindent
``{\it Which\/} and {\it that\/} are not in themselves very important.  But tone
    is important, and tone consists entirely of making these tiny, tiny
    choices.  If you make enough of them wrong---choices like {\it which\/}
    versus {\it that\/}---then you won't get your maximum readership.  The
    reader who has to read the stuff will go on reading it, but with less
    attention, less commitment than you want.  And the reader who doesn't
    have to read will stop.''
\smallskip}
\beginsection 39. [Mary-Claire van Leunen on Calisthenics (2)] \tll December {2 (continued)}

During the final moments of class, Mary-Claire finally got to those
intriguing books.  They were worth waiting for.

First, she reminded us of ``Franklin's Exercise'' from her previous lecture.
She read us a passage from {\sl The Autobiography of Benjamin Franklin\/} where
Franklin mentions it:

{\narrower\smallskip\noindent
   A question was once, somehow or other, started between
   Collins and me, of the propriety of educating the female
   sex in learning, and their abilities for study. He was of
   opinion that it was improper, and that they were naturally
   unequal to it.  I~took the contrary side, perhaps a little for
   dispute's sake.  He was naturally more eloquent, had a
   ready plenty of words; and sometimes, as I thought, bore me
   down more by his fluency than by the strength of his
   reasons.  As we parted without settling the point, and were
   not to see one another again for some time, I~sat down to
   put my arguments in writing, which I copied fair and sent
   to him.  He answered, and I replied.  Three or four
   letters of a side had passed, when my father happened to
   find my papers and read them.  Without entering into the
   discussion, he took occasion to talk to me about the manner
   of my writing; observed that, though I had the advantage of
   my antagonist in correct spelling and pointing (which I
   ow'd to the printing-house), I~fell far short in elegance
   of expression, in method and in perspicuity, of which he
   convinced me by several instances.  I~saw the justice of
   his remarks, and thence grew more attentive to the manner in
   writing, and determined to endeavor an improvement.
   
   About this time I met with an odd volume of the
{\sl Spectator}.  It was the third.  I~had never before seen
   any of them.  I~bought it, read it over and over, and was
   much delighted with it.  I~thought the writing excellent,
   and wished, if possible, to imitate it.  With this view 
I~took some of the papers, and, making short hints of the
   sentiment in each sentence, laid them by a few days, and
   then, without looking at the book, try'd to compleat the
   papers again, by expressing each hinted sentiment at
   length, and as fully as it had been expressed before, in
   any suitable words that should come to hand.  Then 
 I~compared my {\sl Spectator\/} with the original, discovered some
   of my faults, and corrected them.  But I found I wanted a
   stock of words, or a readiness in recollecting and using
   them, which I thought I should have acquired before that
   time if I had gone on making verses; since the continual
   occasion for words of the same import, but of different
   length, to suit the measure, or of different sound for
   rhyme, would have laid me under a constant necessity of
   searching for variety, and also have tended to fix that
   variety in my mind, and make me master of it.  Therefore 
 I~took some of the tales and turned them into verse; and,
   after a time, when I had pretty well forgotten the prose,
   turned them back again.  I~also sometimes jumbled my
   collections of hints into confusion, and after some weeks
   endeavored to reduce them into the best order, before I
   began to form the full sentences and compleat the paper.
   This was to teach me method in the arrangement of thoughts.
   By comparing my work afterwards with the original, 
 I~discovered many faults and amended them; but I sometimes
   had the pleasure of fancying that, in certain particulars
   of small import, I~had been lucky enough to improve the
   method or the language, and this encouraged me to think I
   might possibly in time come to be a tolerable English
   writer, of which I was extreamly ambitious.
\smallskip}

Next, she reminded us of the verse-writing exercises that she had so
highly recommended during the previous lecture.  She showed us the book
that she and her student, Steven Astrachan, had worked through together:
{\sl A~Prosody Handbook}, by Shapiro and Beum.  She said what would have been
even better was {\sl Rhyme's Reason\/} by John Hollander.
Mary-Claire said we
should all buy Hollander's book, and then she tried to make sure we would
by playing on the Computer Scientist's love of self-reference. This is
Hollander's description of one particular poetic form:

\smallskip
{\obeylines\parskip=0pt
\qquad\qquad\qquad This form with two refrains in parallel?
\qquad\qquad\qquad (Just watch the opening and the third line.)
\qquad\qquad\qquad The repetitions build the villanelle.
\smallskip
\qquad\qquad\qquad The subject established, it can swell
\qquad\qquad\qquad across the poet-architect's design:
\qquad\qquad\qquad This form with two refrains in parallel
\smallskip
\qquad\qquad\qquad Must never make them jingle like a bell,
\qquad\qquad\qquad Tuneful but empty, boring and benign;
\qquad\qquad\qquad The repetitions build the villanelle
\smallskip
\qquad\qquad\qquad By moving out beyond the tercet's cell
\qquad\qquad\qquad (Though having two lone rhyme-sounds can confine
\qquad\qquad\qquad This form).  With two refrains in parallel
\smallskip
\qquad\qquad\qquad A poem can find its way into a hell
\qquad\qquad\qquad Of ingenuity to redesign
\qquad\qquad\qquad the repetitions.  Build the villanelle
\smallskip
\qquad\qquad\qquad Till it has told the tale it has to tell;
\qquad\qquad\qquad Then two refrains will finally intertwine.
\qquad\qquad\qquad This form with two refrains in parallel
\qquad\qquad\qquad The repetitions build: The Villanelle.
}  
                                                                               
\smallskip
Mary-Claire told us that she once wrote out a recipe for making bagels in
Alexandrine couplets.  It was a good exercise, and it was hard.
She says that it was so hard that she actually began to believe that the
results would be intelligible (and interesting) to someone else.  She sent
the recipe off to a food magazine and received ``a~truly astounded letter
of rejection.''  She cautioned us again that the verse exercises, useful as
they are, ``really are only exercises.''

The final book that she showed us was {\sl A~Handlist of Rhetorical Terms}, by
Richard A. Lanham.  She said Lanham is the source for many of the great
words she dazzles people with.  Some of the terms in Lanham's book are
more useful than others; there are some terms in the book that can only be
represented in Greek syllabic verse.

Mary-Claire and Steven wrote out examples of each term 
in Lanham's book.   Having performed
the exercise, Mary-Claire confidently told us that it was not profitable.
She said that her warning us not to try exercises that won't do us any
good proves that she isn't totally crazy---and that the exercises that she
did give us are worth doing.

\beginsection 40. [Computer aids to writing] \pmr December 4

{\narrower\smallskip\noindent
\llap{``}All the officer patients in the ward were forced to censor letters
  written by all the enlisted-men patients, who were kept in residence
  in wards of their own. It was a monotonous job, and Yossarian was
  disappointed to learn that the lives of enlisted men were only
  slightly more interesting than the lives of officers. After the first
  day he had no curiosity at all.  To break the monotony, he invented
  games. Death~to all modifiers, he declared one day, and out of every
  letter that passed through his hands went every adverb and every
  adjective.  The next day he made war on articles. He reached a much
  higher plane of creativity the following day when he blacked
  everything in the letters but {\it a, an\/} and {\it the}.  That erected
  more dynamic intralinear tensions, he felt, and in just about every
  case left a message far more universal.''
\smallskip}
\line{\hfill --- Joseph Heller, {\sl Catch-22\/}\qquad}

Don rewarded today's early birds with the chance to participate in a
referendum. We voted to decide the due-date for the term papers,
Monday ${14^{\rm th}}$ or Wednesday ${16^{\rm th}}$. UN observers were not
surprised to find the latter date was favoured by the populace; the
only surprise was that the vote was not quite unanimous. Very well
then, said Don: All papers to be handed to him, his secretary or TAs,
by 5pm (Pacific Standard Time) on Wednesday ${16^{\rm th}}$ December.  (The
{\it real\/} early birds were rewarded with some cookies that Sherry was
handing round. And very good they were, too.)

It was of course too much to hope that we could get through the whole
of a CS class without computers rearing their ugly heads; today they
did. Don's topic was computer programs that are supposed to help us
with our writing. Two such---{\tt style} and {\tt diction}---are available on
Navajo (a CSD Unix machine). These are relatively old programs.
State-of-the-art systems cost a lot of money, and so naturally
Stanford doesn't have them. There is a program called {\tt sexist}, for
example, which attempts to alert us to controversial word usage. Don
recalled the occasion when the (London) {\sl Times\/} quoted him as
saying that it wasn't appropriate to talk about `mother and daughter'
nodes in a tree structure, and he received a lot of irate mail as a
result. People seem to be less uptight about such things these days,
he said.

The {\tt style} program  takes a piece of text and scores it according to
`readability'. The analysis is very superficial---way below the
level of human critiquing. However, said Don, these programs are kind of fun.
And they do provide an excuse to read the document from another point
of view. Even if the analysis is wrong it does prompt you to re-read
your prose, and this has to be a good thing.  Don recalled Richard
Feynman's anecdote about his first day at Oak Ridge Laboratories:
Having no idea what he was supposed to be doing, Feynman pointed to a
random symbol in the blueprints and said, ``What about this then?''  
A~technician immediately agreed that Feynman had spotted a significant
and potentially dangerous oversight in the design.

To illustrate the programs, Don had run them on a dozen or so sample
texts. For instance, he used a passage from the rather ponderous
introduction to a book by Alonzo Church; samples of PMR's and TLL's
notes for CS~209;
versions of his own exposition of  binomial coefficients, vintage
 1965 and 1985;
{\sl Wuthering Heights\/}; {\sl Grimms'
Fairy Tales\/}; and part of a book about the Bible that Don is 
writing on weekends. The
{\tt style} routine produces four different readability grades
for any piece of text. Each is literally a ``grade'' in that it
indicates what level of education the piece suggests. The basis of the
grading is very straightforward; it's a linear formula whose variables
are the average number of syllables (or letters) per word and the
average number of words per sentence (or sometimes the reciprocal
of this value). For example, there are constants $\alpha$, $\beta$,
$\gamma$ such that
$$\hbox{grade $= \alpha$ (words/sentence) $\null + \beta$ (syllables/word)
$\null+ \gamma$.}$$

How were $\alpha$, $\beta$, and
$\gamma$ determined? The authors of each readability
 index simply look at a large number
of pieces of writing and assign them a grade-level `by eye'---that is,
they estimate the age of the intended reader.  Each piece of text is then
characterised by three real numbers: 
the average number of words per sentence, 
the average number of syllables per word, 
and the subjective grade level.
So each piece determines a single point in
3-space (plotted against three orthogonal axes); the set of pieces
determines a scatter of points in 3-space. Standard linear regression
techniques are used to find the plane that is the ``best fit'' for these
points. The three parameters above define this plane.

Someone asked whether we should be shooting for some specific grade
level, and if so, what level? Don replied that his usual aim is to
minimise the level, although overdoing this will defeat the purpose.

In addition to the raw scores,  a variety of other
parameters come out of a {\tt style} analysis:
  average length of sentences, percentage of sentences that
 are much shorter or longer than the average, 
percentage of sentences that begin with various parts of speech,
etc. The program also
attempts to classify sentences into types and tabulate their
frequencies, as well as telling us the percentages of nouns,
adjectives, verbs (active or passive), etc. 
 A~sentence  is considered
``passive'' if a passive verb
appears in it anywhere, even in a subclause.
Curiously, {\tt style} classifies any
sentence that begins `It \dots' or `There \dots' as an ``expletive.''
This seems a little strange to those of us who are old enough to
remember Watergate. We always thought that it was quite a different
class of words that the transcribers of Tricky Dicky's tapes felt the
need to delete. 

Don's theological piece stood out as being pitched at a significantly
lower grade level than the other specimens. 
He was initially surprised by this, and double checked the data
to make sure there was no mistake. But on reflection he concluded
that we usually 
 write more obscurely when writing about our own field.
The two versions of his binomial chapter had very similar scores,
despite their having been written twenty years apart. Church's piece
scored high. Don said that the statistics were misleading here;
although Church's sentences are quite long, they are not ugly but musical.
Still they were not a special joy for the reader.

The {\tt 
style} output also noted a lot of passive voice in Church (perhaps not
surprising in a technical work) and a paucity of adjectives in {\sl
Grimms' Fairy Tales}. Don noted that Mark Twain didn't think much of
adjectives either.

A companion program called {\tt
diction}  operates on different lines. It has an
internal dictionary of 450 words and phrases that it deems `questionable' and
flags them, inviting the writer to find an alternative way to express
himself.  For example, {\tt diction} 
doesn't like the word `gratuitous', and flags its use
as an error. Neither does it like the phrases `number of' or `due to'.
Don noted that  copy editors generally prefer `because of' to `due to' in ordinary
writing, and perhaps {\tt diction} is overlooking the mathematical usage:
``This theorem, due to Cauchy, is used \dots''. In Don's book {\sl  \TeX: The
Program}, the copy-editor changed all Don's `due to's to `owing to';
Don changed them all back again. 
But he searched unsuccessfully for a reference to
the mathematical usage in his
dictionaries, so he wondered aloud if he was completely out of line
with the rest of the world. The class unanimously reassured him that `due to'
was quite the elegant way to give credit for a scientific innovation.
Lexicographers are out of touch here.

The word `very' is also on {\tt diction}'s list of suspects.
 Don recalled that someone had once advised him thus: ``Try
changing all your `very's to `damn's and see what results.  Don't use
`very' unless you would happily use `damn' in its place.''  Damn good
advice! 

The {\tt diction} filter also objected to `literally' and `in fact', but
partially redeemed itself by catching a wicked `which'. A~sister
program, {\tt explain}, expands on {\tt diction}'s objections and recommends
improvements. For example, {\tt explain} suggests that we write `if' instead
of `assuming that', and `really' instead of `actually'. In practice,
users  reportedly
accept about 50\% of {\tt diction}'s suggestions. And that's as it
should be---we've got to keep these machines in their place.

\beginsection 41. [Rosalie Stemer on Copy Editing] \pmr December 7

Today we heard from our penultimate guest speaker, Rosalie Stemer.
Rosalie is a wire features editor at the {\sl San Francisco Chronicle}, 
teaches copy editing at Berkeley, and has worked 
as a copy editor for the {\sl San Francisco Chronicle},
 the {\sl Kansas City Star}, and {\sl Chicago Daily News}. 
So she wields an ultimate pen.

It's a sad truth, Rosalie said, that people who should be able to
write well often can't. She illustrated with a newspaper headline:
$$\advance\abovedisplayskip-3pt
\advance\belowdisplayskip-3pt
\hbox{\tt DISABLED FLY TO SEE CARTER}$$
and a story that began: ``Doing what he loved best, golf pro John
Smith died while \dots''. She told us
about the occasion when a newspaper was having trouble fitting the
word `psychiatrist' into a headline, and resolved the problem simply
by writing `dentist' instead.

Rosalie went through a story filed by an experienced journalist,
pointing out its good and bad features and the changes she had made.

{\narrower\smallskip\noindent
\llap{``}Nine out of ten books bought in this century by
the U.S. Library of Congress, one of the great research libraries of
the world, will self-destruct in 30 to 50 years.'' 
\smallskip}

She faulted this sentence on a
number of counts. The `great research libraries' phrase puts the wrong
focus on the sentence---we are not really concerned about the status
of the Library of Congress in this article. `Nine of ten \dots' would be
better, she said; the word `out' is superfluous. And does the Library
of Congress buy the books that it houses?  No.
Publishers {\it give\/} books to the Library of Congress, as required by law.

The second
sentence noted that the problem is plaguing fine book collectors,
among others.  What does `fine' modify here, Rosalie asked: the books
or their collectors?  

{\narrower\smallskip\noindent
\llap{``}Yet many books several hundred years old are
in excellent condition, Dr.~Norman Shaffer, the congressional
library's director of preservation, said yesterday.''  
\smallskip}

Rosalie thought the subject and verb too far apart.
Moreover, she
said, it raises the question: ``{\it Why\/} are they in excellent
condition?'' 

Other points: ``\dots the cheaper process of making their
products \dots''---cheaper than what?  ``Another solution lies in
persuading \dots''---a~solution to what? And who should be doing the
persuading?  Rosalie saw a systematic error here. A hallmark of good
writing is that it answers more questions than it raises, she said.

Someone asked whether reporters perhaps write a little sloppily in the
knowledge that the copy-editor will go over their copy and clean it
up? Rosalie said that they certainly {\it shouldn't\/} do this. Someone
else pointed out that it's probably a bad idea to start talking about
solutions (to problems) in the same breath as {\it chemical\/}
solutions. This is the ``unfortunate coincidence'' problem that Don
talked about recently. 

Rosalie shuddered over an extremely awkward
sentence about acids and alkalis---fortunate\-ly someone in the class
was able to decipher and explain it.  ``One hopeful sign \dots'' was
another problem---the sign is not full of hope. Over and over again we
saw sentences in which the subject and verb are separated by many
words---these are hard to read. Rosalie pointed out a number of
places in which whole phrases could be dropped without any loss of
meaning: ``Dr.~Shaffer said one of the most encouraging signs is the
fact he has heard one of the largest paper manufacturers \dots'' can be
reduced to ``Dr.~Shaffer said one of the largest paper
manufacturers \dots''.  Whenever you see ``he has heard'' you can
often improve or delete it, Rosalie said. A~similar case: ``The
reason for removing the spaces from the list is that \dots'' can be
(better) written ``We remove the spaces because \dots''.

Rosalie conceded that good writing is very difficult. We must strive
to be clear, coherent, accurate, and concise. This last is especially
important, she said, and quoted Pascal: ``I~have made this letter
longer than usual because I lack the time to make it shorter.''

Rosalie was pleased to note that  the first drafts of our term
papers were quite a bit better than something else she had read
recently---a~report by a local software company. After just a few
weeks we are pushing out the envelope of Silicon Valley literacy!
But many of our sentences could be improved, she said,  by cutting
them shorter. Out with the semicolons, in with the periods. Don't
write one long sentence if you can say the same thing in two short
ones. A~semicolon should be used only where the separated clauses
have a very close relationship, and even there a period is often
better. 
She quoted William Zinsser in his book {\sl On Writing Well\/}
as saying ``The semicolon all too easily conveys `a~certain
19th-century mustiness' and slows the pace of the writing.'' Another
common error was the frequent repetition of a word like `this',
`they', `just', or `then'. Reading the piece aloud will often help you
spot such over-uses.

Rosalie wasn't too keen on some of the conversational idioms that
crept into our writing: sentences that begin `Anyway, \dots'; an
algorithm described as `pretty straightforward' (perhaps a bad idea
in a paper on pretty-printers?). Neither did she like the phrase
`again iterate through'---this sounds awkward; surely the same point
could be put more smoothly? A~lot of sentences suffered from not
having their subject near the beginning: ``If \dots, the
graphic interface \dots''. Someone suggested that these kinds of
problems---as well as over-use of the passive voice---are easily
avoided if we stick to a subject-verb-object style.

Someone asked whether `in the context of' wasn't a ``noise-phrase''---one
 that could be deleted without any loss of meaning?  Rosalie said
that this was often so, but that sometimes it can mean something. In
the example we were looking at, the phrase had been used early on, so
it seemed reasonable to repeat it somewhere else to make it clear to
the reader that we were  once again talking about the same thing as before.

A student asked whether perhaps there aren't different styles of
writing appropriate to scientific journals and to newspapers?
Newspapers would probably put a greater premium on simple, direct
sentences, for example. Rosalie said that there might be something in
this, but that clear writing was always good.

The sentence ``Each of the $\Delta z$'s are then multiplied by this
factor'' can be improved on two counts: Change
`are' to `is' and eliminate the
passive voice. How about: ``Multiply each $\Delta z$
by this factor''?

If you are using commas to insert a parenthetical note, you must put a
comma on {\it both\/} sides: ``\dots this node,~$b$, is thus \dots''.

Rosalie didn't like a sentence that began `It is unlikely \dots'.  The
pronoun reference problem appears: {\it What\/} is unlikely?  There
is no uniform rewrite-rule for this, but we can usually find an
alternative construction that conveys the same meaning. Like Jeff Ullman,
Rosalie wasn't enthusiastic about `This is done by \dots'; better, she
said, to say ``This procedure (process, step, etc.) is done by \dots''.

A very common error is the misplaced `only'. To illustrate, Rosalie
took the sentence ``I~hit him in the eye yesterday'' and inserted
`only' in each of the eight possible positions. Sure enough, each
resulting sentence carries a somewhat different force: 

\smallskip
{\obeylines\parskip 0pt
\qquad\qquad\qquad Only I hit him in the eye yesterday.
\qquad\qquad\qquad I only hit him in the eye yesterday.
\qquad\qquad\qquad I hit only him in the eye yesterday.
\qquad\qquad\qquad I hit him only in the eye yesterday.
\qquad\qquad\qquad I hit him in only the eye yesterday.
\qquad\qquad\qquad I hit him in the only eye yesterday.
\qquad\qquad\qquad I hit him in the eye only yesterday.
\qquad\qquad\qquad I hit him in the eye yesterday only.
}

If
you say ``Here we only calculate the position of two vertices'' you
probably mean ``Here we calculate the position of only two vertices.''

We saw a sentence that contained four or five occurrences of the word
`then'---surely a trifle excessive? Someone remarked that the sentence
was probably an anglicised version of a line of computer code, which
abounds in `{\bf if} \dots {\bf then} \dots's, sometimes deeply nested.

Another line that caught Rosalie's eye was: ``\dots saving the
computation for the place where it is really needed.'' 
The word `really' was used again the same paragraph.
She thought
this sounded altogether too vague for a piece of technical writing: Is
the computation needed or not? What is this ``really needed''? There
was a definite difference of opinion over this question: Some people
in the class couldn't see any objection to this usage. Someone argued
that the ``really'' amounted to stylistic advice (as in ``when
painting a house, be especially careful on the window-frames, where
precision is really important''), but it is  by no means superfluous---the
 word makes a substantive contribution to the meaning of the
sentence. Rosalie's objection stemmed mainly from the fact that the
word ``really'' is much over-used in colloquial speech. In the end we
agreed that it would probably be better to say something like: ``\dots
saving the computation for those vertices where the additional work
contributes more to the visual quality''.

Can an object witness a property? To Rosalie's ear this was a strange
construction. But the class assured her that this is common usage in
computer science. Technical terms take on an anarchic life of their
own!

In the last minute, Rosalie showed us a list of pairs of words that are
frequently---and sometimes amusingly---conflated. For example, `prostrate'
and `prostate'.  
One common confusion is `alternately' vs.\
`alternatively'. These are not synonyms. (Alternately, Tracy and I take
notes in class. You could read them, or alternatively you could take
your own notes.)
\beginsection 42. [Paul Halmos on Mathematical Writing] \tll December 9

Don started class by introducing Paul Halmos.  Paul is a distinguished
 author, a
professor of mathematics at the University of Santa Clara, and a spicy and
entertaining lecturer.  As Don said, ``He brings our program of guest
speakers to a triumphant conclusion.''

%Looking very much like a man who does not flinch from a confrontation,
Paul started his lecture by wondering why we had called him here.
``I~don't have anything new to say,'' he said. ``What I had to say has already
been majorized by Don and Mary-Claire.'' He said that even the act of
talking about mathematical
writing was difficult, by comparison with the act of talking about
mathematics itself. We don't have to remember 
much about math, because we know its structure; we can
develop and discover the material as we talk about it. The structure of
mathematical writing is much more elusive,
so how do we know what to say about it? Sure, Paul brought
several pages of prepared notes to class, but he claims that even those
won't help him much.  

Not that the subject of
mathematical writing isn't important.  Some mathematicians have disdain
for anything other than great theorems. ``Anything else is beneath them.''
But they are wrong. Mathematicians who merely {\it think\/} great theorems
have no more
done their job than painters who merely {\it think\/} great paintings.

Paul has read our handouts, and he wants to make a few comments.
He wants to have a dialog with us; he admonished us to break in
whenever we feel the urge.

He is going to drift in and out of many different topics but only after
he has given us an anchor and a rough outline.  The anchor?  Two basic rules:

\nobreak\smallskip
\halign{\qquad\qquad\qquad#\hfil\cr
	Do organize material. \cr
	Do not distract the reader.\cr}

The outline? Four aspects of good mathematical communication:

\nobreak\smallskip
\halign{\qquad\qquad\qquad#\hfil\cr
	Semantics (words, and the job they do);\cr
	Syntax (also known as grammar);\cr
	Symbols (very meaningful to mathematical writers);\cr
	Style (synthesis of the above).\cr}

Turning first to Semantics, Paul spoke to us about the natural process of
change
inherent in language and how it affects our word usage.  Some changes
are good---some changes are bad.  According to Paul, one of the most often
discussed symptoms of that change is the word `hopefully'.

The most recent literary tradition, handed down to us by our
grandparents, tells us that
`hopefully' means the exact opposite of `hopelessly':
\nobreak\smallskip
\halign{\qquad#\hfil\cr
	``I don't have a chance in the world to be promoted,''
		he growled hopelessly.\cr
	``My chances look good,'' his colleague grinned hopefully.\cr}
But another, impersonal use of `hopefully' has become popular---an evasive
form in which one can say ``Hopefully he won't be re-elected'' instead
of ``I hope he won't.'' This conflicts with the normal usage of other words that
can end both -fully and -lessly. Although we may
think that interest rates will rise, we don't say ``Thoughtfully interest
rates will rise.'' Although we may fear that muggers are in the street,
we don't say ``Fearfully those muggers are still out there.'' Consistent
English usage would prohibit

\nobreak\smallskip
\halign{\qquad\qquad\qquad#\hfil\cr
	Hopefully I'll visit you again next year,\cr}
as much as it prohibits

\nobreak\smallskip
\halign{\qquad\qquad\qquad#\hfil\cr
	Hopelessly I'll not be able to come.\cr}

Paul doesn't like the new usage, which he calls ``illogical and ugly.''
The mere fact of change is bothersome.
But he realizes that
his is only one vote, and he seems
to be outnumbered.  On  balance, it is perhaps a good change, one that
might even make communication easier. ``The English language won't
collapse if the other side wins.'' In fact, Paul says,
\nobreak\smallskip
\halign{\qquad\qquad\qquad#\hfil\cr
	Arguably the change is a needed one.\cr}
But he is surprised to hear himself saying that.

Paul sees other changes as needless and careless.
 It grates on his ears when he hears,
\nobreak\smallskip
\halign{\qquad\qquad\qquad#\hfil\cr
	The earthquake decimated more than half the houses.\cr}
Of course some would say, Why do we need to reserve
 a special word for the random destruction of one out of
ten? Paul thinks muddying the meaning of the word is bad, but he admits
that it is harmless.

Other 
unneeded, and  harmful,  obfuscations should be discouraged.
`Imply' does not mean `infer', and `disinterested' does not mean
`uninterested'.  To confuse these words is to lose valuable distinctions.
Tragically, the differences between these words are becoming so confused
that if we are writing for a large audience, and if we need to make use of
the distinctions, we probably shouldn't.

Evidence of bad changes can even be found in our handouts.  In \S4,
one of the TAs (not the one with the charming British Accent) used
`reference' as a verb.  Paul's response: ``There is no such verb, and if
there were, it sure as hell wouldn't be transitive.''
How would it sound to say ``I quotationed the author''?

Barry Hayes pointed out that in Computer Science, `reference' is a technical
term  used as a verb. Technical terms like `majorize' sometimes
creep into our vocabularies. Don supported him by saying that computer programmers
``reference and de-reference things all the time.''  Paul's response: ``My
condolences.  You know, the French say English is ruining their language.
How the French feel about English is how I feel about that.''

We moved on to Syntax.
``Obviously,'' said Paul, ``people approve of it; nobody uses ungrammatical
English on purpose.'' Syntax changes more slowly than semantics.
However, he once heard the following lovely sentence:
\nobreak\smallskip
\halign{\qquad\qquad\qquad#\hfil\cr
	If I'da knowed I coulda rode, I woulda went.\cr}
This has rhythm, it's communicative, it's personal; but of course it's not
grammatical English. Therefore it distracts the reader from what is
actually being said. Here's another non-made-up example:
\par\nobreak\smallskip
\halign{\qquad\qquad\qquad#\hfil\cr
	Us'll go along with she if her'll go along with we.\cr}
If we are trying to communicate with people who use such grammar, we
should use their language so as not to distract {\it them\/} from what we're
saying. But as technical writers we are presumably not addressing that
audience, certainly not in print.

Paul would
like to advance the thesis that grammar is logic.  This notion is abhorrent
to linguists, who see grammar as illogical, inconsistent, and contradictory;
and they are right. Nevertheless, grammar is the organizational principle
that lies behind linguistic communication. A typical English sentence
like `He saw her' contains case, tense, and gender; such things give
a tremendous amount of information in condensed form, and they can
be seen as logic. To identify grammar with logic is less of an error
than to reject logic altogether.

Speaking of case, Paul says, ``Cases are good things, even though in
English by now they are vestigial.''  They do exist, and they must be
treated with respect. We say
\par\nobreak\smallskip
\halign{\qquad\qquad\qquad#\hfil\cr
	I don't know him,\cr}
but we wouldn't be caught dead saying
\par\nobreak\smallskip
\halign{\qquad\qquad\qquad#\hfil\cr
	I don't know he.\cr}
Similarly, we say
\par\nobreak\smallskip
\halign{\qquad\qquad\qquad#\hfil\cr
	He is the President of France,\cr}
but never
\par\nobreak\smallskip
\halign{\qquad\qquad\qquad#\hfil\cr
	Him is the President of France.\cr}
Therefore we would not logically ask,
\par\nobreak\smallskip
\halign{\qquad\qquad\qquad#\hfil\cr
	Whom is the President of France?\cr}
Simple, right? Well, there are
more confusing cases too:
\par\nobreak\smallskip
\halign{\qquad\qquad\qquad#\hfil\cr
	I don't know who is the President of France.\cr}
Or should it be `I don't know
whom is the President of France'?  A grammatical push-pull
is involved here. (The nominative wins, and
`Who' is correct.)

Paul would like to stamp out abuses such as `I hate whomever said that'.
An attention to logical rules of grammar helps us to clarify our own
thinking in general.

Taking issue with part of our first handout, Paul says the rule ``A~preposition is
a bad word to end a sentence with'' is ``reactionary grammarian balderdash.''
Consider:
\par\nobreak\smallskip
\halign{\qquad\qquad\qquad#\hfil\cr
	Palo Alto is a good place to live in.\cr
	Don Knuth is fun to have a drink with.\cr
	There aren't many people I would say that to.\cr}
All of these are examples of prepositions in ``post position'' that could
only be ruined by being made grammatically pure. (We
have all heard Winston Churchill's famous
statement about ``the sort of nonsense up with which I will not
put.'')  Why should we do gymnastics for sentences with only one
preposition at the end?  Paul gave us a famous sentence ending with five
prepositions:

{\narrower\par\nobreak\smallskip\noindent
	What did you want to bring that book I didn't 
	want to be read to out of up for?
\smallskip}

On the discussed and re-discussed subject of `which' and `that', he says
that Mary-Claire stole his thunder. It is worthwhile to get it right, but
it is not terribly important.

We began discussing Symbols by discussing punctuation.  Paul urges everyone
(contrary to rule \#25 in \S1) to place quotation marks logically,
every time.  He gave us what he sees as a ridiculous example from
Kate Turabian, whom he calls ``The Antichrist,'' in {\sl The Chicago Manual of
Style\/}:

{\narrower\par\nobreak\smallskip\noindent
	See the section on ``Quotations,'' which may be found elsewhere in 
	this volume.
\smallskip}

Paul was incensed.  ``Horrors'', he said. ``You see the illogic,
don't you? There's no reason for it. It's not
a grammatical convention---it's a totally arbitrary typographer's
convention. The battle against this sort of stupidity can be won.''
He has succeeded in getting his own books punctuated logically.
Bob Floyd gave support by mentioning how deadly such 
conventions are in a book about computer programming.

But then Don remarked that one of Paul's two main points was not
to distract the reader. Paul said,
``And your implied, snide, argument?'' ``Well,'' said Don, ``I guess I'm
implying that you think you're distracting only the copy editors and not
the readers.'' ``Yes, I believe that's right, with respect to commas
and quotation marks.''

Mary-Claire asked, ``Just how far are you willing to go in the direction of
logic?  Are you willing to place periods outside the quotation marks in
actual dialog that already has its own punctuation?''. Her example:
\par\nobreak\smallskip
\halign{\qquad\qquad\qquad#\hfil\cr
	He said, ``No.''.\cr}
Paul said that if you push him in a corner he might go so far as to say ``Yes.''.
And Mary-Claire responded,
``That's what I thought.  Luckily there's not much dialog in the sort of stuff
you write.''.  (Paul conceded that he doesn't really have an ear for
dialog and doesn't have immediate
plans to break into the world of fiction. He would love to write a novel,
some piece of literature that isn't expository, but 
he's not being held back by an inability to punctuate.)

The second Symbols-related point that Paul wanted to bring to our
attention was the subject of written versus symbolic numerals.  He gave us
an examples where `one' could either be a pronoun or a numeral,
depending on the context:
\par\nobreak\smallskip
\halign{\qquad\qquad\qquad#\hfil\cr
	What are we to do when $x$ is one?\cr}
The sentence preceding that one may have been
\par\nobreak\smallskip
\halign{\qquad\qquad\qquad#\hfil\cr
	The solutions of the equation are the singularities of the\cr
	function we are studying.\cr}
Or it may have been
\par\nobreak\smallskip
\halign{\qquad\qquad\qquad#\hfil\cr
	Everything is clear when $x$ is 2 or greater.\cr}

Another example (this time from Birkhoff \& MacLane's classic text):
\par\nobreak\smallskip
\halign{\qquad\qquad\qquad#\hfil\cr
	The first few positive primes are\cr
\noalign{\par\nobreak\smallskip}
	\qquad2, 3, 5, 7, 11, \dots\ .\cr
\noalign{\par\nobreak\smallskip}
	Any positive integer which is not one\cr
	or a prime can be factored \dots\cr}
  He urges us to remove such ambiguities by 
using~`1' when we want to speak of the numeral.
\par\nobreak\smallskip
\halign{\qquad\qquad\qquad#\hfil\cr
	The number of solutions is either two or three.\cr
	The only solutions are 2 and 3.\cr}

Paul now moved on to the final area of discussion: Style. 

Rule \#6 in \S1 suggests that we use `we' to avoid passive voice.
This use of `we' is equivalent to `the reader and~I'.  Paul says that even
better is to avoid both passive voice and the use of `we' through judicial
use of imperative and indicative moods along with an outlying kind of
non-sentential phrase.
For example,
\par\nobreak\smallskip
\halign{\qquad\qquad\qquad#\hfil\cr
	We can now prove the following result:\cr}
becomes
\par\nobreak\smallskip
\halign{\qquad\qquad\qquad#\hfil\cr
	A consequence of all this is the following result.\cr}
Or,
\halign{\qquad\qquad\qquad#\hfil\cr
	Consequence: $A$ implies $B$.\cr}
The latter technique can occasionally be used in a sequential manner,
\par\nobreak\smallskip
\halign{\qquad\qquad\qquad#\hfil\cr
	Consequence 1: $X$. \ Consequence 2: $Y$.\cr}
ending with a final blaze of glory,
\par\nobreak\smallskip
\halign{\qquad\qquad\qquad#\hfil\cr
	Conclusion: $Z$.\cr}

Alternatively, here's an example of imperative mood:
\par\nobreak\smallskip
\halign{\qquad\qquad\qquad#\hfil\cr
	All we need to do to get the answer is to replace $x$ by 7 throughout.\cr}
Just say
\halign{\qquad\qquad\qquad#\hfil\cr
	Replace $x$ by 7 throughout.\cr}
Paul finds this less distracting.  Using `we' is not a crime, but it adds
an irrelevant dimension that can often
 be replaced by something clearer and smoother.

\def\\{$\langle$something$\rangle$}
He gave a lengthier example of a typical passage that shows how both `we'
and passive voice can be avoided without sounding artificial:
\par\nobreak\smallskip{\narrower \narrower \narrower \raggedright
	If $U$ is \\, then the spectral theorem justifies the assumption \\.
	If $f$ is \\, then \\ equals \\. Since, however, $y$ is \\, it
	follows that\footnote*{Passive, God forgive me, or at least not
	active; but this phrase
	is standard and inoffensive to my ear.} \\. Since, moreover,
	the assumption \\ implies that \\, the Le\-besgue theorem is applicable.
	This completes the proof of convergence.\par}
(The example would be more effective, of course, if the \\s were replaced
by meaty concepts, but that would distract us from the point at issue.)

An audience member asked if using `we' introduced a light tone that imperative
doesn't have.  Paul agreed that it does, and stated that he isn't sure he
wants that tone in his writing.

Another questioner asked
about first person singular?  Paul likes it, but he admits that it
can be disturbing: ``Who does that jerk think he is?'' He
reluctantly agrees that the first person singular should be avoided in
formal technical writing.

Leslie Lamport asked at this point if the use of `we' could not be avoided
by avoiding prose proofs in favor of tabular proofs.  Paul didn't like
that idea at all: He finds symbols insidious and much prefers prose proofs.
But then he had second thoughts, saying that he and Lamport might not disagree
too much on the need to rethink the techniques of proof presentation.
Outline form (not too heavily symbolic) might be advantageous.

Paul said that he casts all possible votes in favor
of Rule \#9 in \S1: Do not
echo unusual words.  We had been told that this
is a good idea because it avoids monotony.  Paul says that it is a
good idea because two uses of the same word in unrelated passages
will be associated in the reader's mind and cause unwarranted connections.
(Bob Floyd says that it will also cause technical typists to omit all
words between the two occurrences.)

Paul's next bugaboo (``Do I dare do this thing?'')\
was the phrase `he or she' when he feels the traditional neuter
pronoun `he' would
be sufficient.  As soon as he brought this up, Mary-Claire disagreed, but
Paul held the floor and quoted from authority by reading Mary-Claire's words
from page~4 of her own book:

{\narrower\par\nobreak\smallskip\noindent
	This `his' is generic, not gendered.  `His or her' becomes clumsy 
	with repetition and suggests that `his' alone elsewhere is masculine, 
	which it isn't.   `Her' alone draws attention to itself and distracts 
	from the topic at hand. 
\smallskip}

Mary-Claire responded, ``Deeply moving quotation, but it is not true that
the traditional solution to this problem in English is `he'.  The
traditional solution is `they'.''

Many people in the audience stated pieces of opinions, but time was nearly up.
``To each their own.''
Paul moved to
the next topic: Proof by contradiction.  He emphasized that proofs
by contradiction should not be used if a direct proof is available. 
For example, he noted that proofs of linear independence often say, 
``Suppose the variables are linearly dependent. Then there are coefficients,
not all zero, such that \dots contradicting the assumption that the
coefficients are nonzero.''
This circuitous route can usually be replaced by a direct argument:
``If the linear relation \dots \  holds, the coefficients are all zero.
Hence the variables are linearly independent.''

Don pointed out that proof by
contradiction is often the easiest way to prove something when you're
first solving a problem for yourself, but such stream-of-consciousness
proofs don't usually lead to the best exposition.

Paul wound up his speech by repeating his opening rules: ``Do organize,'' and
``Do not distract.''

The trouble is that it is hard to say what organization is.  But
we recognize it when we see it. ``Give me a
book, or a paper or a manuscript,
and I'll tell you if it is organized,'' said~Paul. The material is in
linear order, but organization means much more than that. ``The plot
of an exposition is rarely a straight line.'' Branches and alternative
threads must be woven together. Paul says he spends most of his writing time
working on organization of the material.
He suggests that we
look at Roget's {\sl Thesaurus}, an encyclopedia, a do-it-yourself article, and
a good textbook, for increasingly complex examples of 
 non-linearly-organized
presentations.

``Do organize,'' and ``Do not distract.'' Except that all rules are made to be
broken.  When you want to jar your readers, Paul suggests that you distract
them by changing your notation, screaming ungrammatical sentences, or being
awkwardly repetitious.

His final words to the class were, ``Anything that helps communication
is good.  Anything that hurts is bad.  And that's all I have to say.''
\beginsection 43. [Final truths] \tll December 11

The final lecture of CS 209 was partially devoted to course evaluation.
(We were, no doubt, harsh but fair.)  Don told us that we would spend the
last 40~minutes of class  looking at the notes of people who have
been going over our handouts but haven't had a chance to speak. (More
course evaluations, perhaps?)  Don said that he wanted to ``end on a note
of honesty and truth.''

The first comments that he addressed were from Nelson Blachman (father of
course member Nancy Blachman). Nelson is very interested in writing (he
writes papers frequently), and he took the time to suggest improvements to
the first few handouts.  

Don liked some of these suggestions, but he found others incompatible
with his personal style.  He said, ``The main thing that I get from this is
that the style has to be your own.  You will write things that someone
else will never write.''  Don says he has learned this lesson well
 by writing an annual Christmas letter with his wife, Jill.
``We get along 364 days of the year,'' he said, ``but there is no way that we
can write a sentence acceptable to both of us.''  (They
have solved the problem by writing alternate paragraphs.)

Among Nelson's suggestions were:

{\narrower\smallskip\noindent
	Changing `the above proof' to `the proof above'.  Don agrees with
	this change mostly because editors are forever calling him on it,
	but the original usage doesn't sound terribly
odd to him.  Nelson says that 
	`above' and `below' are two adjectives that never precede the things 
	they modify. Don thinks `above' has become an adjective, but
`below' hasn't (yet).
	
	Changing `, i.e.' to `; i.e.'.  Don says that that is a matter of
	taste and pacing.
	
	Changing the spelling of `hiccups' to `hiccoughs'.  Don's
	dictionaries preferred the shorter spelling.
	
	Changing `depending on the usage, the terms this, that, or the
	other might be used' to `depending on the usage, the term this,
	that, or the other might be used'.  Don didn't see this as
an improvement.
	
	Changing `programming language notation' to `programming-language
	notation'.  Don said that the suggestion might be appropriate for
readers in 
	other disciplines, but in our field the hyphenation  would
	become annoying.  Analogous cases are `random number generator' and
	`floating point arithmetic', each of which is potentially
	ambiguous, but  so familiar in computer science that a hyphen looks wrong.
\smallskip}

%Don next showed us some of Leslie Lamport's comments in support of proof
%by contradiction.  Leslie's note began, "I had to leave the tube just as
%Halmos was beginning to diatribe against proofs by contradiction."  Don
%noted the, to him at least, new usage of the word "diatribe" as a verb,
%and then went on to give us his own feelings about proof by contradiction.

%Don does not feel that proof by contradiction is always bad.  When he
%tries to prove something new, he finds proof by contradiction is often
%the method he chooses. ("It's not just that I'm a contrary person.")
%After all, a proof by contradiction allows more hypotheses because we can
%assume the contradiction.

%However, the method that we use when we first prove something is not
%necessarily the method of choice for the final, formal presentation.
%Proofs by contradiction can limit the applicability of a theorem and can
%confuse readers. Finding an elegant direct proof can both increase the
%range of the theorem and aid others who are trying to follow our proof.

Then Don briefly showed us an example of a problem that often
occurs when mathematicians are allowed to typeset their own text.
A~novice typesetter tends to make fractions like 
${n(n+1)(2n+1)\over 3}$ instead of using the more readable
slashed form $n(n+1)(2n+1)/3$.

Next, we returned to Mary-Claire's essay on `hopefully' (see \S{26}
above).  Don says that he passed it out to us more for the style
of the essay than the content, but it does make good technical points
as well.  To
his surprise, Mary-Claire said that after re-reading it she actually
wanted to improve the style. (This proves once again that nothing is
perfect.) Here is what Mary-Claire wrote to him:

\display 30pt:
1): The dates should be expressed in the same terms.  Given that I'm
    going to need to say `1637', I~have to say `late in the 1500s',
    not `late in the 16th century'.

\display 30pt:
2): The sentence

{\narrower\narrower\smallskip\noindent
        Impersonal substantives, on the other hand, serve less often
        than personal ones at the head of the kind of active verbs
        we modify with adverbs of manner
\smallskip}

\display 30pt::
    is so horrid I'd prefer to think I was drunk when I wrote it.
    To fix it I have to rewrite the whole paragraph, sliding `adverb
    of manner' up earlier:
   
\def\|{\advance\abovedisplayskip-6pt\advance\belowdisplayskip-6pt}
{\narrower\narrower\smallskip\noindent
        As with most adjectives, both of these `hopeful's regularly
        produced \hbox{`-ly'} adverbs of manner.  The kind of hopefulness
        that means expectant and eager produced adverbs more readily
        than the kind that means promising and bright.  There's nothing
        mysterious about that difference in frequency.  The pattern
$$\|\hbox{$\langle$personal noun$\rangle\,\langle$active 
verb$\rangle\,\langle$adverb of manner$\rangle$}$$
        is very common.  People can carry themselves hopefully or eye
        a desirable object hopefully or prepare themselves hopefully
        for a possible future.  The pattern
$$\|\hbox{$\langle$impersonal noun$\rangle\langle$active 
verb$\rangle\langle$adverb of manner$\rangle$}$$
        is less common.  Impersonal nouns serve less often than personal
        ones as subjects of the kind of active verbs that we modify
        with adverbs of manner.  Nonetheless, a wager can be shaping
        up hopefully, a day can begin hopefully, \dots  etc.
\smallskip}

Bob Floyd sent a few comments to Don,  beginning with his opinion of
the usage of hopefully.  First, he reports that only 44\% of the American
Heritage Usage Panel found the use of hopefully as a sentential adverb
acceptable.  Bob 
also provided several authoritative quotations to support his objection to
its use.  (Don said 
that this is the main concern: Using `hopefully' raises hackles in many
people, distracting them from what you're trying to say;  that's
why he doesn't use it. But he thinks some of the documents
 that Bob uses to support his
position were probably written by the
people that Mary-Claire was calling ignorant in her essay.)

Tom Henzinger, who is Austrian, observed that the German language
has a common word `hoffentlich' that corresponds precisely to
the new English usage of `hopefully'. This reminded
Don  that he often needs
words  that the English language just doesn't have.  For
example, we have hundreds of ways
 to say that Jane beat Jim, but we have few ways to
say that Jim lost to Jane. (And we have to use two words in the latter case where
only one is needed in the former.) Don said:

{\narrower\smallskip\noindent
    Our language often lacks verbs that correspond to ``reflexive''
    relations. We have an abundance of words like `dominate' but none like
    `dominate or equal to'.  So we must use long-winded phrases like `less 
    than or equal to'; sometimes, but not often enough, we can say `at most'.
\smallskip}

Returning to Bob Floyd's comments, Bob sent Don several citations to
support his claim that exclamation points
 should be used only with actual exclamations or interjections.
Some examples: Ouch! Stop! Thief! Well, I'll be!
To Don's surprise, none of the authorities even mention that exclamation
points can indicate surprise! Paul Halmos, speaking from the peanut
gallery today, told about a trick he has to get around this: You can
put the exclamation point in parentheses(!).\footnote*{Don was able
to use that trick the next day in Chapter~8 of his book. (Who said
this course wasn't practical?) But he found that it was like an
unusual word: You can't easily repeat it again in the same chapter.}
Then everybody is happy, because you've made an exclamation of surprise.

Bob said, ``Advice to always avoid splitting infinitives is unwise.''  Don
agreed that split infinitives can provide good emphasis and that rewrites
can sound forced or awkward.

About not ending sentences with prepositions, Bob said, ``You have no case,
give up.''  Don agreed, saying that he had not understood the issue.
``Coming from Milwaukee, where half the people speak English with a
heavy dose of German, has made me oversensitive to sentences that end
funny.'' However, there is a problem with sentences ending with
prep\-ositions, namely when they already have a structure that accommodates
the preposition in the middle:

{\narrower\smallskip\noindent
Avoid such prepositions, which such sentences end with. The people who
don't like the rule against prepositions in post position would never
think of writing such sentences, so they probably have forgotten
why the overly restrictive rule was first formulated.
\smallskip}


Bob next objected to Don's suggestion not to omit `that's.  Don admitted
that there are cases when leaving out a `that' produces a better sentence.
For example, `He said he was going' is a better sentence than `He said
that he was going.'  But, in this example `that' is not needed as a
grammatical help because the pronoun (in nominative case) keeps the syntax
clear. 
In technical writing
we often have more complicated sentences, which can benefit from the extra
information that `that' provides.

Someone in the class mentioned a related issue: Should the word `then' be
used in sentences like ``If I get there early enough, $\langle$then$\rangle$ 
I~will save
you a seat.''  (Rosalie had suggested that it should not.)  Don says that
there is a difference between technical writing and newspaper writing, and
he believes that well placed `then's can make a paper more easily
understood.  In that particular sentence he would definitely leave
out `then'; but in mathematical contexts (where the phrase after the
comma is likely a mathematical statement) he would definitely leave it~in.
Don says that our brains only have time to do simple parsing when we are reading
for speed and comprehension.  As Paul Halmos said, ``Anything that helps 
communication is good.''

The final subject that Don introduced was a 
behind-the-scenes discussion between
Mary-Claire, Don, and one of the class members: Dan Schroeder.  Dan
received the comments on his term paper and objected to the claim that he
had ``wicked-whiches''; he gave involved logical reasoning in support of
why his whiches really should be whiches.  Don said, ``If you have to think
that long about the sentence, it is probably wrong.''
Mary-Claire said that writers have to contend with overly-sensitive
readers like Don, who wince at  all whiches that aren't preceded
by commas or prepositions.

In one place Dan did not place a comma before a `which' because he was
concerned about coincident commas.  This is what Mary-Claire has to say
about coincident commas:

{\narrower\smallskip\noindent
   Coincident commas are not a sign of bad construction, any more than
   the coincidence of a final comma and a period, or a final comma and
   a semicolon, or any other two marks of punctuation.  Where two commas
   coincide, we write only one.  Where a comma and a period coincide,
   we write the period.  Etc.  Truly, coincident punctuation is not a
   problem.
\smallskip}

(Did you catch the coincident periods there?)

After this comment we were thrown from the room in order to make way for
another class.  As always in this course, there was more to say than there
was time to say it in.

\medskip
Postscript: The instructor received an anonymous contribution after class,
in response to his request for a poetically stated computer program:

\halign{\qquad\qquad{\tt{#}}\cr
This algorithm to count bits\cr
Rotates VALUE one left and sums its\cr
\qquad two's-comp negation\cr
\qquad in a zeroed location\cr
Repeats WORDLENGTH times, then exits.\cr
}

(Not only does this rhyme and scan, it also works. In fact, it may be
the fastest way to do sideways addition on the GE635 and similar machines.)

\vfill\eject
Postscript about ``God is in the details'' (see page 48):
William Safire's column on Language in the {\sl New York Times\/} and the
{\sl International Herald Tribune}, July 31, 1989, discusses this mysterious
phrase as well as its counterpart, ``The Devil is in the details.'' Nobody
has been able to trace either one to a definite source. Safire cites
Shapiro who claims to cite Nietzsche---but without chapter and verse.
Safire also says that Justin Kaplan, editor of {\sl Bartlett's Familiar
Quotations}, is searching too. According to Kaplan, ``Flaubert has been
suggested, but nobody can find it in his writings.''

Perhaps the following facts will be helpful. A biography of Mies by Franz
Schulze (Chicago, 1985) has a relevant footnote on page 281:

{\narrower\smallskip\noindent
The aphorism, ``God is in the details,'' has been endlessly attributed to
Mies, though I have found no one who ever heard him say it. In {\sl
Meaning in the Visual Arts\/} (New York, 1955, p.~5), Erwin Panofsky
quotes Flaubert: ``Le bon Dieu est dans le d\'etail.''
\smallskip}
\noindent Actually Schulze should have referred not to page `5', but to
page~`v'---where Panofsky drops Flaubert's name but gives no hint of location.

David Spaeth, another Mies biographer, answered a query from Don as follows
on December 6, 1985:

{\narrower\smallskip\noindent
The statement ``God is in the details'' was made by Mies. It [only]
appeared in print in an article by Peter Blake entitled ``The difficult
art of simplicity'' which was published in {\sl Architecture Forum\/}
vol.~108, May 1958, pages 126--131. However, it was a statement Mies made
in class many times. George Danforth, one of Mies' very first students,
remembers Mies saying it in the early 1940s.
\smallskip}

{\sl Simpson's Contemporary Quotations}, by James B. Simpson (1988),
traces the phrase to Mies, citing an article entitled ``On restraint
in design'' in the {\sl New York Herald Tribune\/} for 28 June 1959.
Don hasn't yet had a chance to verify this citation, which may well have
simply been derived from {\sl Architecture Forum}.

Paul Roberts (PMR) wrote to Nigel Rees, who produces a BBC radio show based
on quotation identification, asking for his opinion. Rees found the phrase
mentioned in the {\sl New York Herald Tribune\/} obituary of Mies, 1969,
and said that Mies ``certainly popularized it even if he didn't originate it.''

PMR also turned up a significant clue that might in fact be the true origin
of the saying. At least it carries things back further than anyone else has
been able to do so far: On pages 13, 14, 229, and 286 of E.~H. Gombrich's
1970 biography of Aby Warburg, a prominent historian of Renaissance art,
Gombrich states that Warburg used the phrase ``Der liebe Gott steckt im
Detail'' as the motto of a seminar series at Hamburg University in the
fall of 1925; he also says that Warburg often repeated this motto.

Thus, Mies probably learned the phrase in Germany. But was it original
with Warburg?  We can't rule out Flaubert and Nietzsche until their
complete works have been made available in electronic form.

{\narrower\smallskip\noindent
Particulars, as every one knows, make for virtue and happiness; generalities
are intellectually \hbox{necessary} evils.\par}
\rightline{--- Aldous Huxley, {\sl Brave New World}.}

\bye
